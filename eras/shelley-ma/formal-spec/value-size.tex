\section{Output Size}
\label{sec:value-size}

In Shelley, the protocol parameter $\var{minUTxOValue}$ is used to
disincentivize attacking nodes by putting many small outputs on the
UTxO, permanently blocking memory. In the Mary era
and onwards, the $\ValMonoid$ is $\Value$, rather than $\Coin$ (as in Allegra). Elements of $\Value$
can potentially be arbitrarily large, so the ledger requires each $\UTxO$ entry to
contain a minimum amount of Ada, proportional to its size.

There is also another $\Value$ size consideration with respect to spendability
of an output. The restriction on the total serialized size of the transaction (set
by the parameter $\var{maxTxSize}$) serves as an implicit upper bound on the
size of a $\Value$ contained in an output of a transaction. Without tighter
limits on the output $\Value$ size, one of the following situations could arise,
causing the output to become unspendable (these are just a two examples):

\begin{itemize}
  \item The script locking the very large $\Value$-containing UTxO is too large
  to fit inside the transaction alongside the $\Value$ itself while still respecting
  the max transaction size
  \item The large $\Value$ cannot be split into several outputs, because the
  outputs are impossible to fit inside a single transaction
\end{itemize}

The same considerations apply for any underlying $\ValMonoid$ we choose to fix.
In the ShelleyMA eras, the two types that are used to define concrete ledgers are $\Coin$ and $\Value$.
The size calculations for $\Coin$, in practice,
result in either trivial restrictions in the ledger rules,
or ones that align with Shelley (as discussed in Section \ref{sec:utxo}).

\subsection{Value Size}

Figure \ref{fig:size-helper} contains abstract and helper functions
used in calculating the in-memory and serialized representation
sizes of $\Value$ elements.

\begin{figure*}[h]
  \emph{Abstract Functions}
  %
  \begin{align*}
    & \fun{serialize} \in \wcard \to \ByteString \\
    & \text{Serialization function on an arbitrary (serializable) type}
    \nextdef
    & \fun{anameLen} \in \AssetName \to \MemoryEstimate \\
    & \text{Returns the length (in bytes) of an asset name}
  \end{align*}
  %
  \emph{Helper Functions}
  \begin{align*}
    & \fun{serSize} \in \ValMonoid \to \MemoryEstimate \\
    & \fun{serSize}~v=\lvert \fun{serialize}~{v} \rvert \\
    & \text{Gives the size of the serialized representation of a $\ValMonoid$}
    & \nextdef
    & \fun{numAssets} \in \Value \to \N \\
    & \fun{numAssets}~{vl}=\lvert \{~(pid, an)~\vert~ pid \mapsto (an \mapsto \wcard) \in vl~\} \rvert \\
    & \text{Returns the number of distinct asset IDs in a $\Value$}
    & \nextdef
    & \fun{sumALs} \in \Value \to \N \\
    & \fun{sumALs}~{vl}= \sum_{\{~an~\vert~\wcard~\mapsto~(an~\mapsto~\wcard)~\in~vl~\}} \fun{anameLen}~an \\
    & \text{Returns the sum of the lengths (in bytes) of distinct asset names in a $\Value$}
    & \nextdef
    & \fun{numPids} \in \Value \to \N \\
    & \fun{numPids}~{vl} = \lvert \fun{pids}~{vl} \rvert \\
    & \text{The number of policy IDs in a $\Value$}
  \end{align*}
  \caption{Value Size Helper Functions}
  \label{fig:size-helper}
\end{figure*}

For a serializable type, the function $\fun{serialize}$ is defined on that type. We do not
specify serialization functions here, but the CDDL specification gives the encoding
details (see \cite{alonzoCDDL}). We give this function here because it is the
first time it is used in the specification, but it is not specific to
multi-assets, Mary, or Allegra in any way.

The function $\fun{serSize}$ returns the actual number of bytes in the bytestring that is the
    serialized representation of a $\ValMonoid$ element. The specific underlying $\ValMonoid$
    is required to be serializable in every era.

    The $\fun{serSize}$ function is used to constrain
    the serialized representation of the transaction (in particular, the size
    of $\Value$ elements in outputs), whereas the min-Ada requirement is a calculation based on
    the in-memory representation size. A transparently-calculated size estimate
    is not necessary for limiting the size of values in outputs, since this size-bound
    check does not place any additional accounting/monetary constraints on transaction construction,
    unlike the min-Ada requirement.

\subsection{Min UTxO Value}
\label{sec:min-value}

Figure \ref{fig:min-val-calc} gives the types of constants used in the estimation
of the size of a UTxO entry, and the associated min-Ada-value.

\begin{figure*}[h]
  \emph{Constants}
  \begin{equation*}
    \begin{array}{lcl}
      (k_0, k_1, k_2, k_3, k_4) & \in & \N \times \N \times \N \times \N \times \N \\
      \mathsf{UtxoEntrySizeWithoutVal} & \in & \MemoryEstimate \\
      \mathsf{AdaOnlyUTxOSize} & \in & \MemoryEstimate \\
      \mathsf{MaxValSize} & \in & \MemoryEstimate \\
    \end{array}
  \end{equation*}
  %
  \emph{Size and Min-Ada Functions}
  \begin{align*}
    & \fun{size} \in \Value \to \MemoryEstimate \\
    & \fun{size}~\var{vl} =
    \begin{cases}
      k_0 & \fun{isAdaOnly}~vl\\
      k_1 + \lfloor~ \fun{numAssets}~vl * k_2 + \fun{sumALs}~vl & \\
      ~~~~~~ + \fun{numPids}~vl * k_3 + k_4 - 1 /~ k_4~\rfloor & \text{otherwise} \\
    \end{cases} \\
    & \text{Calculate the size of a $\Value$}
    \nextdef
    & \fun{coinsPerUTxOWord}\in \Coin \to \Coin \\
    & \fun{coinsPerUTxOWord}~\var{mv} = \lfloor~ \var{mv}~/~ \mathsf{adaOnlyUTxOSize}~ \rfloor \\
    & \text{Calculate the cost of storing a memory unit of data as a UTxO entry}
    \nextdef
    & \fun{utxoEntrySize} \in \ValMonoid \to \MemoryEstimate \\
    & \fun{utxoEntrySize}~\var{v} = \mathsf{utxoEntrySizeWithoutVal} + \fun{size}~\var{v} \\
    & \text{Calculate the size of a UTxO entry}
\end{align*}
\caption{Value Size Calculation}
\label{fig:min-val-calc}
\end{figure*}

The $\fun{size}$ function returns the estimated size of a $\Value$ element. The size
function on $\Value$ is defined via the isomorphism in Section \ref{sec:coin-value},

\[ \fun{size}_{\Value}~v=\fun{size}~(\fun{iso}_{v}~v) \]

The size of a $\Value$ element is constant in the case when it contains only Ada.
If there are other types of assets contained in it, the size depends on

\begin{itemize}
  \item the number of distinct asset types (asset IDs)
  \item the number of distinct policy IDs, and
  \item the sum of the lengths of distinct asset names.
\end{itemize}

The parameter $\fun{minUTxOValue}$ specifies the min-Ada value for a UTxO containing
only Ada. This type of UTxO varies in size somewhat (eg. Byron style addresses
may be a different length than Shelley ones), but we estimate the size of the most commonly
used type of Ada-only UTxO as the constant value $\mathsf{adaOnlyUTxOSize}$.
This constant is in fact an upper bound on UTxOs which have have only Shelley credentials.

We use this size estimate
to calculate what $\fun{minUTxOValue}$ implies to be the min-Ada value requirement
\emph{per word} of UTxO data.
The function $\fun{coinsPerUTxOWord}$ performs this calculation by dividing the
min-Ada value by the Ada-only UTxO size (and taking the floor).

The $\mathsf{utxoEntrySizeWithoutVal}$ is the constant representing
the size of a UTxO entry, not counting the size of the $\ValMonoid$ element it contains.
Here, again, the actual size of a UTxO (excluding the $\ValMonoid$ element) can vary, but
we use an upper bound on the size of Shelley-credential UTxOs.

The function $\fun{utxoEntrySize}$ estimates the size of an arbitrary ShelleyMA-era
UTxOs. It adds the size estimate of the $\ValMonoid$ element in a UTxO and the
$\mathsf{utxoEntrySizeWithoutVal}$ constant.

The constants used in the implementation of the ShelleyMA eras are as follows :

\begin{itemize}
  \item $(k_0, k_1, k_2, k_3, k_4) = (1, 6, 12, 28, 8)$
  \item $\mathsf{utxoEntrySizeWithoutVal} = 27$ words (8 bytes)
  \item $\mathsf{adaOnlyUTxOSize} = 27$ words (8 bytes)
  \item $\mathsf{MaxValSize} = 4000$ bytes, ie. 500 words.
\end{itemize}
