\label{sec:crypto-details}

\subsection{Hashing}
The hashing algorithm for all verification keys and multi-signature scripts is BLAKE2b-224.
Explicitly, this is the payment and stake credentials (Figure~\ref{fig:defs:addresses}),
the genesis keys and their delegates (Figure~\ref{fig:ts-types:pp-update}),
stake pool verification keys (Figure~\ref{fig:delegation-transitions}),
and VRF verification keys (Figure~\ref{fig:delegation-defs}).

Everywhere else we use BLAKE2b-256.
In the CDDL specification in Appendix~\ref{sec:cddl},
$\mathsf{hash28}$ refers to BLAKE2b-224 and
and $\mathsf{hash32}$ refers to BLAKE2b-256.
BLAKE2 is specified in RFC 7693 \cite{rfcBLAKE2}.

\subsection{Addresses}
The \fun{sign} and \fun{verify} functions from Figure~\ref{fig:crypto-defs-shelley}
use Ed25519. See \cite{rfcEdDSA}.

\subsection{KES}
The \fun{sign_{ev}} and \fun{verify_{ev}} functions from Figure~\ref{fig:kes-defs-shelley}
use the iterated sum construction from Section 3.1 of \cite{cryptoeprint:2001:034}.
We allow up to $2^7$ key evolutions, which is larger than the maximum number
of evolutions allow by the spec, \MaxKESEvo, which will be set to $90$.
See Figure~\ref{fig:rules:ocert}.

\subsection{VRF}
The \fun{verifyVrf} function from Figure~\ref{fig:defs-vrf}
uses ECVRF-ED25519-SHA512-Elligator2 as described in the draft IETF specification
\cite{rfcVRFDraft}.

\subsection{Multi-Signatures}
As presented in Figure~\ref{fig:types-msig}, Shelley realizes multi-signatures 
in a native way, via a script-like DSL. One defines the conditions required to 
validate a multi-signature, and the script takes care of verifying the 
correctness of the request. It does so in a na\"ive way, i.e. checking every 
signature individually. For instance, if the requirement is to have $n$ valid 
signatures out of some set $\mathcal{V}$ of public keys, the na\"ive 
script-based solution checks if: (i) the number of submitted signatures is 
greater or equal to $n$, (ii) the distinct verification keys are part of the set 
$\mathcal{V}$, and (iii) at least $n$ signatures are valid. However, there are 
more efficient ways to achieve this using more advanced multi-signature schemes, 
that allow for aggregating both the signatures and the verification procedure. 
This results in less data to be stored on-chain, and a cheaper verification 
procedure. Several schemes provide these properties~\cite{musigBoneh, musig, 
musig2, pixel}, and we are currently investigating which would be the best fit 
for the Cardano ecosystem. We formally introduce multi-signature schemes.

\sloppy
A multi-signature scheme~\cite{musigs} is defined as a tuple of algorithms 
$\textsf{MS} = (\textsf{MS-Setup}, \textsf{MS-KG}, \textsf{MS-AVK}, 
\allowbreak\textsf{MS-Sign}, \textsf{MS-ASign}, \textsf{MS-Verify)}$ such that 
$\Pi\leftarrow\textsf{MS-Setup}(1^k)$ generates public parameters---where $k$ is 
the security parameter. Given the public parameters, one can generate a 
verification-signing key pair calling, 
$(\mathsf{vk,sk})\leftarrow\textsf{MS-KG}(\Pi)$. A multi-signature scheme 
provides a signing and an aggregate functionality. Mainly
\begin{itemize}
\item $\sigma\leftarrow\textsf{MS-Sign}(\Pi, \mathsf{sk}, m)$, produces a 
signature, $\sigma$, over a message $m$ using signing key $\mathsf{sk}$, and
\item $\tilde{\sigma}\leftarrow\textsf{MS-ASig}(\Pi, m, \mathcal{V, S)}$, where 
given a message $m$, a set of signatures, $\mathcal{S}$, over the message $m$, 
and the corresponding set of verification keys, $\mathcal{V}$, aggregates all 
signatures into a single, aggregate signature $\tilde{\sigma}$. 
\end{itemize}

To generate an aggregate verification key, the verifier calls the function 
$\mathsf{avk}\leftarrow\textsf{MS-AVK}(\Pi, \mathcal{V})$, which given input a 
set of verification keys, $\mathcal{V}$, returns an aggregate verification key 
that can be used for the verification of the aggregate siganture: 
$\textsf{MS-Verify}(\Pi, \mathsf{avk}, m, 
\tilde{\sigma})\in\{\textsf{true},\textsf{false}\}$. 

Note the distinction between multi-signature schemes (as described above) and 
multi-signatures as defined in Figure~\ref{fig:types-msig}. In 
Figure~\ref{fig:types-msig} we allow scripts to consider valid the 
\type{RequireAllOf}, \type{RequireAnyOf} or \type{RequireMOfN} typed signatures. 
In the definition above, given a set of public keys $\mathcal{V}$, a signature 
is considered valid if \textit{all} key owners participate. However, such 
multi-signature schemes together with a simple script-like DSL can achieve the 
properties defined in Figure~\ref{fig:types-msig} while still providing the 
benefits of a simple verification procedure, and a smaller signature size. 
