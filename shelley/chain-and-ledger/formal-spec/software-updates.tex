\section{Software Update Adoption}
\label{sec:software-updates}

This section describes how software updates are performed.
The process of upgrading the system a new version consists of:

\begin{enumerate}
  \item New software is ready for downloading.
  \item The core nodes propose and vote on changing the
    \textit{application version}.
    The distinction between protocol and software updates, as well
    as the voting and proposal of AV updates, is discussed
    in Section \ref{sec:update}.
  \item If accepted, the proposal is stored on the blockchain as the future AV
    $\var{favs} = s \mapsto ap$,
    indexed by the slot number $s$ in which $ap$
    is scheduled to become the official version.
  \item The users can now download the applications listed in $ap$
  \item In slot $s$, $ap$ becomes the official version, stored as the value
  $\var{avs}$ on the blockchain
  \item All nodes running official Cardano software update their software
  automatically, as soon as they see the updated official version on the ledger.
  \item This new software should be implementing the ledger rules in the
  PV currently in the PP's on the ledger
  \item There may be PP update proposals with new versions of PV's on the ledger
  (the AV $ap$ must be able to implement all these proposed PV updates)
  \item Every block producer includes the PV they are ready to move to
  in the block header (this would be a PV in an update proposals on the ledger)
  \item Core nodes monitor the PV's in the block headers of blocks being produced
  \item When enough blocks have the same future PV, core nodes vote to make
  it the current PV at some upcoming epoch boundary, see Section \ref{sec:ledger-trans}
  \item When the PV version switch is made on the ledger, the applications $ap$
  now begin implementing the ledger rules of the new PV
\end{enumerate}
