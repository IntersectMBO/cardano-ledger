
\subsection{Aspects that we need to model}
\label{sec:aspects-to-model}

\begin{description}
\item[Authentication] Update proposals and votes are authenticated (properly
  signed).
\item[Authorization] Only genesis keys (via deleg certs) can post update
  proposals.
  \begin{itemize}
  \item Only then they can vote on them.
  \end{itemize}
\item[Voting deadlines] voting end when a majority of the voters (4/7 if we
assume no stake) agree on the proposal.
\item[Block versions] (= protocol versions)
\item[Soft-forks] a protocol version changes according to the fork-resolution
  rule (75\% of stake create blocks with new-version).
\item[{Hard-forks}] ??? Do we need to model anything here?.
\end{description}

\subsection{Information in the ledger state}
\label{sec:information-in-ledger-state}

The ledger state has to expose these parameters of the protocol version to its
clients:

\begin{description}
\item[Slot duration] 
\end{description}

At the moment we don't know whether we need these:

\begin{description}
\item[Script Version] 
\end{description}


\lstset{ frame=tb,
       , language=Haskell
       , basicstyle=\footnotesize\ttfamily,
       , keywordstyle=\color{blue},
       , commentstyle=\itshape\color{purple!40!black},
       , identifierstyle=\bfseries\color{green!40!black},
       , stringstyle=\color{orange},       
       }

       
\begin{lstlisting}
  data BlockVersionData = BlockVersionData
\end{lstlisting}
