%% \begin{note}
%%   If you are reviewing this document, please have a look at the TODO
%%   file. It mentions some things that are still up for discussion and
%%   we'd like your opinion.
%% \end{note}

\begin{figure*}[t!]
  %
  \emph{Abstract Types}
  %
  \begin{equation*}
    \begin{array}{r@{~\in~}l@{~}lr}
      \var{s_{mc}} & \ScriptMPS & \text{monetary policy script}
     \end{array}
  \end{equation*}
  %
  \emph{Derived types}
  %
  \begin{equation*}
    \begin{array}{r@{~\in~}l@{\qquad=\qquad}lr}
      \var{lng} & \Language & \{\mathsf{nativeMSigTag}, \mathsf{nativeMATag}, \cdots\} \\
      \var{scr} & \Script & \ScriptMSig \uniondistinct \ScriptMPS \\
      \var{txout}
      & \TxOut
      & \Addr \times \Value
%      & \text{tx outputs}
      \\
      \var{utxoout}
      & \UTxOOut
      & \Addr \times \Value \\
%      & \text{utxo outputs}
      \var{utxo}
      & \UTxO
      & \TxIn \to \UTxOOut
%      & \text{unspent tx outputs}
    \end{array}
  \end{equation*}
  %
  \emph{Abstract functions}
  %
  \begin{align*}
    \fun{language} ~\in~    & \Script \to \Language \\
                            & \text{returns the language tag, e.g. $\mathsf{nativeMATag}$ for the MPS language} \\
    \fun{evalMPSScript}~\in~& \ScriptMPS\to\PolicyID\to\Slot\to\powerset\KeyHash \\
    %
  \end{align*}

  \emph{Transaction Type}
  %
  \begin{equation*}
    \begin{array}{r@{~~}l@{~~}l@{\qquad}l}
      \var{txbody} ~\in~ \TxBody ~=~
      & \powerset{\TxIn} & \fun{txinputs}& \text{inputs}\\
      &\times ~(\Ix \mapsto \TxOut) & \fun{txouts}& \text{outputs}\\
      & \times~ \seqof{\DCert} & \fun{txcerts}& \text{certificates}\\
       & \times ~\Value  & \fun{forge} &\text{value forged}\\
       & \times ~\Coin & \fun{txfee} &\text{non-script fee}\\
       & \times ~\Slot & \fun{txttl} & \text{time to live}\\
       & \times~ \Wdrl  & \fun{txwdrls} &\text{reward withdrawals}\\
       & \times ~\Update  & \fun{txUpdates} & \text{update proposals}\\
       & \times ~\MetaDataHash^? & \fun{txMDhash} & \text{metadata hash}\\
    \end{array}
  \end{equation*}
  %
  \emph{Accessor Functions}
  \begin{equation*}
    \begin{array}{r@{~\in~}lr}
      \fun{getValue} & \TxOut \uniondistinct \UTxOOut \to \Value & \text{output value} \\
      \fun{getAddr} & \TxOut \uniondistinct \UTxOOut \to \Addr & \text{output address} \\
    \end{array}
  \end{equation*}
  \caption{Type Definitions used in the UTxO transition system}
  \label{fig:defs:utxo-shelley}
\end{figure*}

\section{Transactions}
\label{sec:transactions}

This section describes the changes that are necessary to the transaction and
UTxO type structure to support native multi-asset functionality
in Cardano.

\begin{itemize}

  \item $\Language$ is a type of labels for scripting languages, e.g.
    $\mathsf{nativeMSigTag}$ or $\mathsf{nativeMATag}$.

  \item $\Script$ is the type of scripts allowed on the ledger. These
    include the multi-signature scripts from the Shelley era, as well
    as monetary policy scripts. The monetary policy scripting language
    is a basic scripting language that allows for expressing some of
    the most common monetary policies. Note however, that there are no
    restrictions on using a certain kind of script for a certain
    purpose, meaning that any script the ledger can be used
    anywhere. A suggestion for $\ScriptMPS$ and the implementation of
    the function $\fun{evalMPSScript}$, which evaluates MPS scripts,
    is given in Appendix~\ref{sec:mps-lang}.
    
\item $\TxOut$ is the type of outputs that are carried by a transaction. This differs from the base Shelley
  $\TxOut$ type in that it contains a $\Value$ rather than a $\Coin$.

  \item $\UTxOOut$ is the type of UTxO entry that is created when a transaction
  output is processed. This has the same structure as
  the transaction output $\TxOut$, but is given a different name to
  account for the fact that $\Value$ is stored differently in the outputs of $\UTxO$ and $\Tx$
  (due to optimization in the $\UTxO$).

  \item $\UTxO$ entries are stored in the finite map $\TxIn\mapsto \UTxOOut$.
  This type also differs from the Shelley $\UTxO$ type only in that $\Coin$ is replaced by $\Value$.

\end{itemize}

\subsection*{The Forge Field}

The body of a transaction with multi-asset support contains one additional
field, the $\fun{forge}$ field.
The $\fun{forge}$ field is a term of type $\Value$, which contains
tokens the transaction is putting into or taking out of
circulation. Here, by "circulation", we mean specifically "the UTxO on the
ledger". Since the administrative fields cannot contain tokens other than Ada,
and Ada cannot be forged (this is enforced by the UTxO rule, see Figure~\ref{fig:rules:utxo-shelley}),
they are not affected in any way by forging.

Putting tokens into circulation is done with positive values in the $\Quantity$
fields of the tokens forged, and taking tokens out of circulation can be done
with negative quantities.

A transaction cannot simply forge arbitrary tokens. Restrictions on
Multi-Asset tokens are imposed, for each asset with ID $\var{pid}$, by the script
with the hash $\var{pid}$. Whether a given transaction adheres to the restrictions
prescribed by its script is verified as part of the processing of the transaction.
The forging mechanism is detailed in Section~\ref{sec:utxo}.

\subsection*{Transaction Body}

Besides the addition of the $\fun{forge}$ field to the transaction body,
note that the $\TxOut$ type in the body is not the same as
the $\TxOut$ in the system without multi-asset support. Instead of
$\Coin$, the transaction outputs now have type $\Value$.

The only change to the types related to transaction witnessing is the addition
of MPS scripts to the $\Script$ type, so we do not include the whole $\Tx$ type here.

\clearpage
