\section{Introduction}

This document describes the extensions to the multi-asset formal ledger specification~\ref{XX}
required for the support of non-native scripts, in particular Plutus Core. This underpins future Plutus development: future developments should require minimal or no changes to these ledger rules.
%
The two major extensions are:
\begin{inparaenum}
\item
the introduction
of non-native scripts, i.e. scripts that are not evaluated by the ledger; and
\item
  additions to the Shelley-era UTxO (unspent transaction output) model that are needed to support Plutus
  constructs (the ``extended UTxO'' model).
\end{inparaenum}
This document defines these extensions as changes to the multi-asset structured transition system,
using the same notation and conventions that were used for the multi-asset specification~\cite{XX}.
As with the multi-asset formal specification, these rules will be implemented in the form of an executable ledger specification that will then be
integrated with the Cardano node.

\subsection{Non-Native Scripts}

The Shelley formal specification introduced the concept of ``multi-signature'' scripts.
\emph{Native scripts}, such as these, are processed entirely by the ledger rules.
Their execution costs can easily be assessed before they are processed,
and any fees can be calculated directly by the ledger rules, based on e.g. the
size of the transaction.

In contrast, non-native scripts can perform arbitary
(and, in principle, Turing-complete) computations.
In order to bound execution costs to a pre-determined constant, we use a standard ``fuel'' approach~\cite{XX}.
We require transactions that use non-native scripts
to have a budget in terms of a number of abstract $\ExUnits$.
These provide a quantitative bound on resource usage in terms of metrics such as memory usage or abstract execution steps.
This abstract budget is then used as part of the concrete transaction fee calculation.

To change the $\ExUnits$ required to
run the same computation without a hard fork (for example because a
more efficient interpreter was introduced), every scripting language
converts the actual execution cost into $\ExUnits$ using a cost model,
$\CostMod$, which is a protocol parameter.

\subsection{Extended UTxO}

The specification of the extended UTxO model follows the description in~\cite{plutus_eutxo}.
All transaction outputs locked by non-native scripts must include the hash of an additional ``data'' item, which the transaction needs to supply. This data could also have been stored directly in the UTxO, but this scheme has been chosen for space efficiency reasons. The data is passed to a script that validates that the output is spent correctly, and can be used to encode state.
All transactions also need to supply a \emph{redeemer} for everything that is validated by a script, which is an additional piece of data passed to the script. The redeemer can be regarded as user-input for the scripts. Note that the same script can be used for multiple different purposes in the same transaction, so in general it does not suffice to include one redeemer per script.
