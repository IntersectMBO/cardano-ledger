\section{Introduction}

This document describes the extensions to the multi-asset formal ledger specification~\ref{XX},
that are needed for the Plutus Foundation.  This underpins future Plutus development: future developments should require minimal or no changes to these ledger rules.
%
The two major extensions are:
\begin{inparaenum}
\item
the introduction
of non-native scripts, i.e. scripts that are not evaluated by the ledger; and
\item
  additions to the Shelley-era UTxO (unspent transaction output) model that are needed to support Plutus
  constructs (the ``extended UTxO'' model).
\end{inparaenum}
This document defines these extensions as changes to the multi-asset structured transition system,
using the same notation and conventions that were used for the multi-asset specification~\ref{XX}.
As with the multi-asset formal specification, these rules will be implemented in the form of an executable ledger specification that will then be
integrated with the Cardano node.

\subsection{Non-Native Scripts}

The Shelley formal specification introduced the concept of ``multi-signature'' scripts.
\emph{Native scripts}, such as these, are processed entirely by the ledger rules.
Their execution costs can easily be assessed before they are processed,
and any fees can be calculated directly by the ledger rules, based on e.g. the
size of the transaction.
% and there are therefore no explicit cost rules in the ledger specification.
% In the current version of the Shelley ledger, there is no
% assessment of the cost of checking multisignature scripts.
% Instead,
% any additional fees incurred as a result of spending MSig-locked
% outputs are proportional to the change in transaction size due to
% including all the necessary signatures (rather than the cost of
% verifying them).

In contrast, non-native scripts can perform arbitary
(and, in principle, Turing-complete) computations.
In order to bound execution costs to a pre-determined constant, we use a standard ``fuel'' approach~\ref{XX},
%% Not completely true...  The cost of many interesting computations can be assessed, and others can be limited.
%% I've done a lot of work here :)
% which means it is impossible to assess
% their execution cost without executing them.
We require transactions that use non-native scripts
to have a budget in terms of a number of abstract $\ExUnits$.
These provide a quantitative bound on resource usage in terms of metrics such as memory usage or abstract execution steps.
This abstract budget is then used as part of the concrete transaction fee calculation, using an explicit
cost model $\CostMod$, that is provided as a protocol parameter.

% To change the $\ExUnits$ required to
% run the same computation without a hard fork (for example because a
% more efficient interpreter was introduced), every scripting language
% converts the actual execution cost into $\ExUnits$ using a cost model,
% $\CostMod$, which is a protocol parameter.

\subsection{Extended UTxO}

The specification of the extended UTxO model follows the description in~\cite{plutus_eutxo}.
All\todo{Some?} transaction outputs must include an additional data item that ...\todo{What is this - a hash, a token? And what does it represent?}.
This is passed to a script that validates that the output is spent correctly. All\todo{Confirm} transactions also need to supply a \emph{redeemer}, as an additional input\todo{Is this a transaction input?}.
The redeemer ...\todo{Explain what it does.}
For example, ...\todo{Give a high level example.}
% which is an additional piece of data, for everything that is validated
% by non-native scripts.
