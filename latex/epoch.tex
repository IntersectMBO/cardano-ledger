The pool reaping in rules are discribed in
Figure~\ref{fig:ts-types:pool-reap} and Figure~\ref{fig:rules:pool-reap}.

%%
%% Figure - Pool Reap
%%


\begin{figure}
  \emph{Pool Reap Transition}
  \begin{equation*}
    \_ \vdash \_ \trans{poolreap}{} \_ \in
    \powerset (\Slot \times \PState \times \PState)
  \end{equation*}
  %
  \caption{Pool Reap Transition}
  \label{fig:ts-types:pool-reap}
\end{figure}


%%
%% Figure - Pool Rules
%%
\begin{figure}

  \begin{equation}\label{eq:pool-reap}
    \inference[Pool-Reap]
    {
      \var{retired} = \var{retiring}^{-1}~\var{(\epoch{slot})}
      & \var{retired} \neq \emptyset
    }
    {
      \var{slot} \vdash
      \left(
      \begin{array}{r}
        \var{stpools} \\
        \var{pparams} \\
        \var{retiring}
      \end{array}
      \right)
      \trans{poolreap}{}
      \left(
      \begin{array}{rcl}
        \var{retired} & \subtractdom & \var{stpools} \\
        \var{retired} & \subtractdom & \var{pparams} \\
        \var{retired} & \subtractdom & \var{retiring} \\
      \end{array}
      \right)
    }
  \end{equation}
  \caption{Pool Reap Inference Rule}
  \label{fig:rules:pool-reap}

\end{figure}
