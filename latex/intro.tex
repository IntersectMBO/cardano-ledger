This document is a formal specification of the functionality of the ledger
on the blockchain. The blockchain layer of the
protocol and the interaction between the ledger and the blockchain
layer is presented in a separate document, see~\cite{}. The details of the
background and the larger context
for the design decisions formalized in this document are presented
in~\cite{delegation_design}

In this work,
we present important properties any implementation of the ledger must have.
Specifically, we model the following aspects
of the functionality of the ledger on the blockchain:

\begin{description}
\item[Preservation of value] relationship between the total value of input and
  outputs in a new transaction, and the unspent outputs.
\item[Witnesses] authentication of parts of the transaction data by means of
  cryptographic entities (such as signatures and private keys) contained in
  these transactions.
\item[Delegation] validity of delegation certificates, which delegate
  block-signing rights.
\item[Stake] staking rights associated to an address.
\end{description}

While the blockchain protocol is a reactive system driven by the arrival
of blocks causing updates to the ledger, the formal description is a collection
of rules which is a
static description of what a \textit{valid ledger} is. The specifics of the
semantics we use to define and apply
the rules we present in this document are explained in detail in
\cite{small_step_semantics}. A valid ledger state can only
reached by applying a sequence of inference rules, and any valid ledger state
is reachable by applying some sequence of these rules.

The structure of the rules we give here is such that their application is
deterministic. That is, given a specific initial state and relevant environmental
constants, there is no ambiguity
about which rule should be applied at any given time (i.e. which state
transition is allowed take place). This is an important property which reflects
the reality of the implementation - the blockchain evolves in a particular way
given some user activity and the passage of time, and its behaviour is
never unexpected.
