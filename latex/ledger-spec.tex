\documentclass[11pt,a4paper]{article}
\usepackage[margin=2.5cm]{geometry}
\usepackage{iohk}
\usepackage{microtype}
\usepackage{mathpazo} % nice fonts
\usepackage{amsmath}
\usepackage{amssymb}
\usepackage{amsthm}
\usepackage{latexsym}
\usepackage{mathtools}
\usepackage{stmaryrd}
\usepackage{extarrows}
\usepackage{slashed}
\usepackage[colon]{natbib}
\usepackage[unicode=true,pdftex,pdfa]{hyperref}
\usepackage{xcolor}
\usepackage[capitalise,noabbrev,nameinlink]{cleveref}
\usepackage{float}
\floatstyle{boxed}
\restylefloat{figure}

%%
%% Package `semantic` can be used for writing inference rules.
%%
\usepackage{semantic}
%% Setup for the semantic package
\setpremisesspace{20pt}

%%
%% Types
%%
\newcommand{\Tx}{\type{Tx}}
\newcommand{\Ix}{\type{Ix}}
\newcommand{\TxId}{\type{TxId}}
\newcommand{\Addr}{\type{Addr}}
\newcommand{\UTxO}{\type{UTxO}}
\newcommand{\Value}{\type{Value}}
\newcommand{\Coin}{\type{Coin}}
\newcommand{\PrtclConsts}{\type{PrtclConsts}}
\newcommand{\Slot}{\type{Slot}}
\newcommand{\Duration}{\type{Duration}}
\newcommand{\Allocs}{\type{Allocs}}

\newcommand{\DCert}{\type{DCert}}
\newcommand{\DCertRegKey}{\type{DCert_{regkey}}}
\newcommand{\DCertDeRegKey}{\type{DCert_{deregkey}}}
\newcommand{\DCertDeleg}{\type{DCert_{delegate}}}
\newcommand{\DCertRegPool}{\type{DCert_{regpool}}}
\newcommand{\DCertRetirePool}{\type{DCert_{retirepool}}}
\newcommand{\StakePool}{\type{StakePool}}
\newcommand{\ledgerState}{\ensuremath{\type{ledgerState}}}

\newcommand{\AddrRWD}{\type{Addr_{rwd}}}
\newcommand{\DState}{\type{DState}}
\newcommand{\DWState}{\type{DWState}}
\newcommand{\DWEnv}{\type{DWEnv}}
\newcommand{\PState}{\type{PState}}
\newcommand{\DCertBody}{\type{DCertBody}}
\newcommand{\DPoolReap}{\ensuremath{\type{poolreap}}}

%% Adding witnesses
\newcommand{\TxIn}{\type{TxIn}}
\newcommand{\TxOut}{\type{TxOut}}
\newcommand{\VKey}{\type{VKey}}
\newcommand{\SKey}{\type{SKey}}
\newcommand{\HashKey}{\type{HashKey}}
\newcommand{\KeyPair}{\type{KeyPair}}
\newcommand{\Sig}{\type{Sig}}
\newcommand{\Data}{\type{Data}}
%% Adding delegation
\newcommand{\Epoch}{\type{Epoch}}
\newcommand{\VKeyGen}{\type{VKeyGen}}
%% Blockchain
\newcommand{\Gkeys}{\var{G_{keys}}}
\newcommand{\Block}{\type{Block}}
\newcommand{\SlotId}{\type{SlotId}}
\newcommand{\UTxOEnv}{\type{UTxOEnv}}
\newcommand{\CEEnv}{\type{CEEnv}}
\newcommand{\CEState}{\type{CEState}}
\newcommand{\BDEnv}{\type{BDEnv}}
\newcommand{\BDState}{\type{BDState}}
\newcommand{\LEnv}{\type{LEnv}}
\newcommand{\LState}{\type{LState}}

%%
%% Functions
%%
\newcommand{\txins}[1]{\fun{txins}~ \var{#1}}
\newcommand{\txid}[1]{\fun{txid}~ \var{#1}}
\newcommand{\txouts}[1]{\fun{txouts}~ \var{#1}}
\newcommand{\values}[1]{\fun{values}~ #1}
\newcommand{\balance}[1]{\fun{balance}~ \var{#1}}
\newcommand{\ttl}[1]{\fun{ttl}~ \var{#1}}
\newcommand{\deposits}[2]{\fun{deposits}~ \var{#1} ~ \var{#2}}
\newcommand{\refund}[4]{\fun{refund}~ \var{#1}~ \var{#2}~ \var{#3}~ \var{#4}}
\newcommand{\refunds}[4]{\fun{refunds}~ \var{#1}~ \var{#2}~ \var{#3}~ \var{#4}}
\newcommand{\created}[5]{\fun{created}~ \var{#1}~ \var{#2}~ \var{#3}~ \var{#4}~ \var{#5}}
\newcommand{\destroyed}[2]{\fun{destroyed}~ \var{#1}~ \var{#2}}
\newcommand{\applyFun}[2]{\fun{#1}~\var{#2}}

\newcommand{\RegKey}[1]{\textsc{RegKey}(#1)}
\newcommand{\DeregKey}[1]{\textsc{DeregKey}(#1)}
\newcommand{\Delegate}[1]{\textsc{Delegate}(#1)}
\newcommand{\RegPool}[1]{\textsc{RegPool}(#1)}
\newcommand{\RetirePool}[1]{\textsc{RetirePool}(#1)}
\newcommand{\cauthor}[1]{\fun{author}~ \var{#1}}
\newcommand{\dpool}[1]{\fun{dpool}~ \var{#1}}
\newcommand{\stakepool}[1]{\fun{stakepool}~ \var{#1}}
\newcommand{\retire}[1]{\fun{retire}~ \var{#1}}
\newcommand{\addrRw}[1]{\fun{addr_{rwd}}~ \var{#1}}
\newcommand{\epoch}[1]{\fun{epoch}~ \var{#1}}
\newcommand{\dcerts}[1]{\fun{dcerts}~ \var{#1}}

%% UTxO witnesses
\newcommand{\inputs}[1]{\fun{inputs}~ \var{#1}}
\newcommand{\wits}[1]{\fun{wits}~ \var{#1}}
\newcommand{\verify}[3]{\fun{verify} ~ #1 ~ #2 ~ #3}
\newcommand{\sign}[2]{\fun{sign} ~ #1 ~ #2}
\newcommand{\serialised}[1]{\llbracket \var{#1} \rrbracket}
\newcommand{\addr}[1]{\fun{addr}~ \var{#1}}
\newcommand{\hashKey}[1]{\fun{hashKey}~ \var{#1}}
\newcommand{\txbody}[1]{\fun{txbody}~ \var{#1}}
\newcommand{\txfee}[1]{\fun{txfee}~ \var{#1}}
\newcommand{\minfee}[2]{\fun{minfee}~ \var{#1}~ \var{#2}}
\newcommand{\slotminus}[2]{\var{#1}~-_{s}~\var{#2}}
\DeclarePairedDelimiter\floor{\lfloor}{\rfloor}
% wildcard parameter
\newcommand{\wcard}[0]{\underline{\phantom{a}}}
%% Adding ledgers...
\newcommand{\utxo}[1]{\fun{utxo}~ #1}
%% Delegation
\newcommand{\delegatesName}{\fun{delegates}}
\newcommand{\delegates}[3]{\delegatesName~#1~#2~#3}
\newcommand{\dwho}[1]{\fun{dwho}~\var{#1}}
\newcommand{\depoch}[1]{\fun{depoch}~\var{#1}}
%% Delegation witnesses
\newcommand{\dbody}[1]{\fun{dbody}~\var{#1}}
\newcommand{\dwit}[1]{\fun{dwit}~\var{#1}}
%% Blockchain
\newcommand{\bwit}[1]{\fun{bwit}~\var{#1}}
\newcommand{\bslot}[1]{\fun{bslot}~\var{#1}}
\newcommand{\bbody}[1]{\fun{bbody}~\var{#1}}
\newcommand{\bdlgs}[1]{\fun{bdlgs}~\var{#1}}
%% ledgerstate constants
\newcommand{\genesisId}{\ensuremath{Genesis_{Id}}}
\newcommand{\genesisTxOut}{\ensuremath{Genesis_{Out}}}
\newcommand{\genesisUTxO}{\ensuremath{Genesis_{UTxO}}}
\newcommand{\emax}{\mathsf{E_{max}}}

\theoremstyle{definition}
\newtheorem{definition}{Definition}[section]

\theoremstyle{definition}
\newtheorem{property}{Property}[section]

\begin{document}

\hypersetup{
  pdftitle={Formal Specification of the Cardano Ledger with a Native
  Multicurrency Implementation},
  breaklinks=true,
  bookmarks=true,
  colorlinks=false,
  linkcolor={blue},
  citecolor={blue},
  urlcolor={blue},
  linkbordercolor={white},
  citebordercolor={white},
  urlbordercolor={white}
}

\title{Formal Specification of the Cardano Ledger with a Native
Multicurrency Implementation}

\author{
   Polina Vinogradova \\ {\small \texttt{polina.vinogradova@iohk.io}} \\
   }

\date{}

\maketitle

\begin{abstract}
This document presents the modifications of the Shelley ledger
specification
(see~\cite{shelley_spec}) which will enable it to support native
Multicurrency (see~\cite{multi_currency} and~\cite{formal_multicur})
using a small scripting language fully specified
by the ledger rules.
\end{abstract}

\section*{List of Contributors}
\label{acknowledgements}

Duncan Coutts,
Philipp Kant,
Michal Peyton Jones,
Jann Mueller,
Jared Corduan,
Matthias Gudemann,
Manuel Chakravarty,
Kevin Hammond


\tableofcontents
\listoffigures

\section{Introduction}
\label{sec:introduction}
\section{Introduction}
\label{sec:introduction}

This specification models the \textit{conditions} that the different parts of a
transaction have to fulfill so that they can extend a ledger, which is
represented here as a list of transactions. In particular, we model the
following aspects:

\begin{description}
\item[Preservation of value] relationship between the total value of input and
  outputs in a new transaction, and the unspent outputs.
\item[Witnesses] authentication of parts of the transaction data by means of
  cryptographic entities (such as signatures and private keys) contained in
  these transactions.
\item[Delegation] validity of delegation certificates, which delegate
  block-signing rights.
\item[Update validation] voting mechanism which captures the identification of
  the voters, and the participants that can post update proposals.
\end{description}

The following aspects will not be modeled (since they are not part of the Byron
release):
\begin{description}
\item[Stake] staking rights associated to an addresses.
\end{description}


\section{Notation}\label{sec:notation}

The transition system is explained in \cite{small_step_semantics}.

\begin{description}
\item[Powerset] Given a set $\type{X}$, $\powerset{\type{X}}$ is the set of all
  the subsets of $X$.
\item[Sequences] Given a set $\type{X}$, $\seqof{\type{X}}$ is the set of
  sequences having elements taken from $\type{X}$. The empty sequence is
  denoted by $\epsilon$, and given a sequence $\Lambda$, $\Lambda; \type{x}$ is
  the sequence that results from appending $\type{x} \in \type{X}$ to
  $\Lambda$.
\item[Functions] $A \to B$ denotes a \textbf{total function} from $A$ to $B$.
  Given a function $f$ we write $f~a$ for the application of $f$ to argument
  $a$.
\item[Fibre] Given a function $f: A \to B$ and $b\in B$, we write
  $f^{-1}~b$ for the \textbf{fibre} of $f$ at $b$, which is defined by
  $\{a \mid\ f a =  b\}$.
\item[Maps and partial functions] $A \mapsto B$ denotes a \textbf{partial
    function} from $A$ to $B$, which can be seen as a map (dictionary) with
  keys in $A$ and values in $B$. Given a map $m \in A \mapsto B$, notation
  $a \mapsto b \in m$ is equivalent to $m~ a = b$.
\item[Map Operations] Figure \ref{fig:notation:nonstandard}
  describes some non-standard map operations.

\end{description}

In Figure~\ref{fig:notation:nonstandard}, we specify the notation we use in
the rest of the document.

\begin{figure}
  \begin{align*}
    \var{set} \restrictdom \var{map}
    & = \{ k \mapsto v \mid k \mapsto v \in \var{map}, ~ k \in \var{set} \}
    & \text{domain foo restriction}
    \\
    \var{set} \subtractdom \var{map}
    & = \{ k \mapsto v \mid k \mapsto v \in \var{map}, ~ k \notin \var{set} \}
    & \text{domain exclusion}
    \\
    \var{map} \restrictrange \var{set}
    & = \{ k \mapsto v \mid k \mapsto v \in \var{map}, ~ v \in \var{set} \}
    & \text{range restriction}
    \\
    \var{map} \subtractrange \var{set}
    & = \{ k \mapsto v \mid k \mapsto v \in \var{map}, ~ v \notin \var{set} \}
    & \text{range exclusion}
    \\
    A \unionoverrideRight B
    & = (\dom B \subtractdom A)\cup B
    & \text{union override right}
    \\
    A \unionoverrideLeft B
    & = A \cup (\dom A \subtractdom B)
    & \text{union override left}
  \end{align*}
  \caption{Non-standard map operators}
  \label{fig:notation:nonstandard}
\end{figure}

\section{Cryptographic primitives}
\label{sec:crypto-primitives}


Figure~\ref{fig:crypto-defs} introduces the cryptographic abstractions used in
this document. We begin by listing the abstract types, which are meant to
represent the corresponding concepts in cryptography. Only the funcitonality
explicitly stated in the figures below is assumed within the scope of this paper.
That is, their exact
implementation remains open to interpretation, and we do not rely on
any additional properties derived from the study or implementation of public key
cryptography outside this work. The types and rules we give here are needed in
order to guarantee certain security properties of the delegation process, which
we discuss later.

The cryptographic concepts required for the formal definition
of delegation with witnesses include public-private key pairs, one-way functions
and signatures. The constraint we introduce states that a signature of
some data signed with a (private) key is only correct whenever we can verify
it using the corresponding public key.

\begin{figure}
  \emph{Abstract types}
  %
  \begin{equation*}
    \begin{array}{r@{~\in~}lr}
      \var{sk} & \SKey & \text{private signing key}\\
      \var{vk} & \VKey & \text{public verifying key}\\
      \var{hk} & \HashKey & \text{hashKey of a key}\\
      \sigma & \Sig  & \text{signature}\\
      \var{d} & \Data  & \text{data}\\
    \end{array}
  \end{equation*}
  \emph{Derived types}
  \begin{equation*}
    \begin{array}{r@{~\in~}lr}
      (sk, vk) & \KeyPair & \text{signing-verifying key pairs}
    \end{array}
  \end{equation*}
  \emph{Abstract functions}
  %
  \begin{equation*}
    \begin{array}{r@{~\in~}lr}
      \hashKey{} & \VKey \to \HashKey
      & \text{hashKey function} \\
      %
      \fun{verify} & \powerset{\left(\VKey \times \Data \times \Sig\right)}
      & \text{verification relation}\\
    \end{array}
  \end{equation*}
  \emph{Constraints}
  \begin{align*}
    & \forall (sk, vk) \in \KeyPair,~ m \in \Data,~ \sigma \in \Sig \cdot
      \verify{vk}{m}{\sigma} \iff \sign{sk}{m} = \sigma
  \end{align*}
  \emph{Notation for serialized and verified data}
  \begin{align*}
    & \serialised{x} & \text{serialised representation of } x\\
    & \mathcal{V}_{\var{vk}}{\serialised{m}}_{\sigma} = \verify{vk}{m}{\sigma}
      & \text{shorthand notation for } \fun{verify}
  \end{align*}
  \caption{Cryptographic definitions}
  \label{fig:crypto-defs}
\end{figure}

\section{Serialization}
\label{sec:serialization}


Besides basic cryptographic abstractions, we also make use of some abstract
data storage properties in this document in order to build necessary definitions
and make judgement calls about them.

Abstract data types in this paper are essentially placeholders with names
indicating the data types they are meant to represent in an implementation.
Derived types are made up of data structures (i.e. products, lists, finite
maps, etc.) built from abstract types. The underlying structure of a data type
is implementation-dependent, and furthermore, the way the data is stored on
physical storage can vary as well.

Serialization is a physical manipulation of data on a given storage device.
In this document, the properties and rules we state involving serialization are
assumed to hold true independently of the storage medium and style of data
organization chosen for an implementation.


\begin{todo}
  Discuss here serialization and
  \href{https://iohk.myjetbrains.com/youtrack/issue/CDEC-628}{composable
    serialization}
\end{todo}

\section{Delegation}
\label{sec:delegation}
\section{Delegation}
\label{sec:delegation}

An agent owning a key that can sign new blocks can delegate its signing rights
to another key by means of \textit{delegation certificates}. These certificates
are included in the ledger, and therefore also included in the body of the
blocks in the blockchain.

There are several restrictions on a certificate posted on the blockchain:
\begin{enumerate}
\item Only genesis keys can delegate.
\item Certificates must be properly signed by the delegator.
\item Any given key can delegate at most once per-epoch.
\item Any given key can issue at most one certificate in a given slot.
\item The epochs in the certificates must refer to the current or to the next
  epoch. We do not want to allow certificates from past epochs so that a
  delegation certificate cannot be replayed. On the other hand if we allow
  certificates with arbitrary future epochs, then a malicious key can issue a
  delegation certificate per-slot, setting the epoch to a sufficiently large
  value. This will cause a blow up in the size of the ledger state since we
  will not be able to clean $\var{eks}$ (we only clean past epochs). Also note
  that we do not check the relation between the certificate epoch and the slot
  in which the certificate becomes active. This would bring additional
  complexity without any obvious benefit.
\item Certificates do not become active immediately, but they require a certain
  number of slots till they become stable in all the nodes.
\end{enumerate}
These conditions are formalized in \cref{fig:rules:delegation-scheduling}.
Rule~\ref{eq:rule:delegation-scheduling} determines when a certificate can
become ``scheduled''. The definitions used in this rules are presented in
\cref{fig:defs:delegation-scheduling}, and the types of the system induced by
$\trans{sdeleg}{\wcard}$ are presented in
\cref{fig:ts-types:delegation-scheduling}.

\begin{figure}[htb]
  \emph{Abstract types}
  \begin{equation*}
    \begin{array}{r@{~\in~}lr}
      c & \DCert & \text{delegation certificate}\\
      \var{vk_g} & \VKeyGen & \text{genesis verification key}\\
    \end{array}
  \end{equation*}

  \emph{Derived types}
  \begin{equation*}
    \begin{array}{r@{~\in~}l@{\qquad=\qquad}r@{~\in~}lr}
      \var{e} & \Epoch & n & \mathbb{N} & \text{epoch}\\
      \var{s} & \Slot & s & \mathbb{N} & \text{slot}\\
      \var{d} & \SlotCount & s & \mathbb{N} & \text{slot}
    \end{array}
  \end{equation*}

  \emph{Constraints}
  \begin{align*}
    \VKeyGen \subseteq \VKey
  \end{align*}

  \emph{Abstract functions}
  \begin{equation*}
    \begin{array}{r@{~\in~}lr}
      \fun{dbody} & \DCert \to (\VKey \times \Epoch)
      & \text{body of the delegation certificate}\\
      \fun{dwit} & \DCert \to (\VKeyGen \times \Sig)
      & \text{witness for the delegation certificate}\\
      \fun{dwho} & \DCert \mapsto (\VKeyGen \times \VKey)
      & \text{who delegates to whom in the certificate}\\
      \fun{depoch} & \DCert \mapsto \Epoch
      & \text{certificate epoch}
    \end{array}
  \end{equation*}
  \caption{Delegation scheduling definitions}
  \label{fig:defs:delegation-scheduling}
\end{figure}

\begin{figure}[htb]
  \emph{Delegation scheduling environments}
  \begin{equation*}
    \DSEnv =
    \left(
      \begin{array}{r@{~\in~}lr}
        \mathcal{K} & \powerset{\VKeyGen} & \text{allowed delegators}\\
        \var{e} & \Epoch & \text{epoch}\\
        \var{s} & \Slot & \text{slot}\\
        \var{k} & \SlotCount & \text{chain stability parameter}
      \end{array}
    \right)
  \end{equation*}

  \emph{Delegation scheduling states}
  \begin{equation*}
    \DSState
    = \left(
      \begin{array}{r@{~\in~}lr}
        \var{sds} & \seqof{(\Slot \times (\VKeyGen \times \VKey))} & \text{scheduled delegations}\\
        \var{eks} & \powerset{(\Epoch \times \VKeyGen)} & \text{key-epoch delegations}
      \end{array}
    \right)
  \end{equation*}

  \emph{Delegation scheduling transitions}
  \begin{equation*}
    \var{\_} \vdash
    \var{\_} \trans{sdeleg}{\_} \var{\_}
    \subseteq \powerset (\DSEnv \times \DSState \times \DCert \times \DSState)
  \end{equation*}
  \caption{Delegation scheduling transition-system types}
  \label{fig:ts-types:delegation-scheduling}
\end{figure}

\begin{figure}[htb]
  \begin{equation}
    \inference
    {
    }
    {
      {\begin{array}{l}
       \mathcal{K}\\
        e\\
        s\\
        d
      \end{array}}
      \vdash
      \left(
        \begin{array}{l}
          \epsilon\\
          \emptyset
        \end{array}
      \right)
    }
  \end{equation}
  \nextdef
  \begin{equation}
    \label{eq:rule:delegation-scheduling}
    \inference
    {
      (\var{vk_s},~ \sigma) \leteq \dwit{c}
      & \verify{vk_s}{\serialised{\dbody{c}}}{\sigma} & vk_s \in \mathcal{K}\\ ~ \\
      (\var{vk_s},~ \var{vk_d}) \leteq \dwho{c} & e_d \leteq \depoch{c}
      & (e_d,~ \var{vk_s}) \notin \var{eks} & 0 \leq e_d - e \leq 1 \\ ~ \\
      d \leteq 2 \cdot k & (s + d,~ (\var{vk_s},~ \wcard)) \notin \var{sds}\\
    }
    {
      {\begin{array}{l}
       \mathcal{K}\\
        e\\
        s\\
        k
      \end{array}}
      \vdash
      {
        \left(
          \begin{array}{l}
            \var{sds}\\
            \var{eks}
          \end{array}
        \right)
      }
      \trans{sdeleg}{c}
      {
        \left(
          \begin{array}{l}
            \var{sds}; (s + d,~ (\var{vk_s},~ \var{vk_d}))\\
            \var{eks} \cup \{(e_d,~ \var{vk_s})\}
          \end{array}
        \right)
      }
    }
  \end{equation}
  \caption{Delegation scheduling rules}
  \label{fig:rules:delegation-scheduling}
\end{figure}

\clearpage

The rules in Figure~\ref{fig:rules:delegation} model the activation of
delegation certificates. Once a scheduled certificate becomes active
(see~\cref{sec:delegation-interface-rules}), the delegation map is changed by
it only if:
\begin{itemize}
\item The delegating key ($\var{vk_s}$) did not activate a delegation
  certificate in a slot greater or equal than the certificate slot ($s$). This
  check is performed to avoid having the constraint that the delegation
  certificates have to be activated in slot order.
\item The key being delegated to ($\var{vk_d}$) has not been delegated by
  another key (injectivity constraint).
\end{itemize}
The reason why we check that the delegation map is injective is to avoid a
potential risk (during the OBFT era) in which a malicious node gets control of
a genesis key $\var{vk_m}$ that issued the maximum number of blocks in a given
window. By delegating to another key $\var{vk_d}$, which was already delegated to
by some other key $\var{vk_g}$, the malicious node could prevent $\var{vk_g}$
from issuing blocks. Even though the delegation certificates take several slots
to become effective, the malicious node could calculate when the certificate
would become active, and issue a delegation certificate at the right time.

As an additional advantage, by having an injective delegation map, we are able
to simplify our specification when it comes to counting the blocks issued by
(delegates of) genesis keys.

Note also, that we could not impose the injectivity constraint in
Rule~\ref{eq:rule:delegation-scheduling} since we do not have information about
the delegations that will become effective. We could of course detect a
violation in the injectivity constraint when scheduling a delegation
certificate, but this will lead to a complex computation and larger state in
said rule.

Finally, note that we do not want to reject a scheduled delegation that would
violate the injectivity constraint (since delegation might not have been
scheduled by the node issuing the block). Instead, we simply ignore the
delegation certificate (Rule~\ref{eq:rule:delegation-nop}).

\begin{figure}[htb]
  \begin{align*}
    & \unionoverrideRight \in (A \mapsto B) \to (A \mapsto B) \to (A \mapsto B)
    & \text{union override}\\
    & d_0 \unionoverrideRight d_1 = d_1 \cup (\dom d_1 \subtractdom d_0)
  \end{align*}
  \caption{Functions used in delegation rules}
  \label{fig:funcs:delegation}
\end{figure}

\begin{figure}[htb]
  \emph{Delegation environments}
  \begin{equation*}
    \DEnv =
    \left(
      \begin{array}{r@{~\in~}lr}
        \mathcal{K} & \powerset{\VKeyGen} & \text{allowed delegators}
      \end{array}
    \right)
  \end{equation*}

  \emph{Delegation states}
  \begin{align*}
    & \DState
      = \left(
        \begin{array}{r@{~\in~}lr}
          \var{dms} & \VKeyGen \mapsto \VKey & \text{delegation map}\\
          \var{dws} & \VKeyGen \mapsto \Slot & \text{when last delegation occurred}\\
        \end{array}\right)
  \end{align*}
  \emph{Delegation transitions}
  \begin{equation*}
    \_ \vdash \_ \trans{adeleg}{\_} \_ \in
    \powerset (\DEnv \times \DState \times (\Slot \times (\VKeyGen \times \VKey)) \times \DState)
    \end{equation*}
  \caption{Delegation transition-system types}
  \label{fig:ts-types:delegation}
\end{figure}

\begin{figure}[htb]
  \begin{equation}
    \inference
    {
      \var{dms_0} \leteq \Set{k \mapsto k}{k \in \mathcal{K}} &
      \var{dws_0} \leteq \Set{k \mapsto 0}{k \in \mathcal{K}}
    }
    {
      \mathcal{K}
      \vdash
      \left(
        \begin{array}{l}
          \var{dms_0}\\
          \var{dws_0}
        \end{array}
      \right)
    }
  \end{equation}
  \nextdef
  \begin{equation}\label{eq:rule:delegation-change}
    \inference
    {
      \var{vk_d} \notin \range~\var{dms} & (\var{vk_s} \mapsto s_p \in \var{dws} \Rightarrow s_p < s)
    }
    {
      \mathcal{K}
      \vdash
      \left(
      \begin{array}{r}
        \var{dms}\\
        \var{dws}
      \end{array}
      \right)
      \trans{adeleg}{(s,~ (vk_s,~ vk_d))}
      \left(
      \begin{array}{lcl}
        \var{dms} & \unionoverrideRight & \{\var{vk_s} \mapsto \var{vk_d}\}\\
        \var{dws} & \unionoverrideRight & \{\var{vk_s} \mapsto s \}
      \end{array}
      \right)
    }
  \end{equation}
  \nextdef
  \begin{equation}\label{eq:rule:delegation-nop}
    \inference
    {\var{vk_d} \in \range~\var{dms} \vee (\var{vk_s} \mapsto s_p  \in \var{dws}  \wedge s \leq s_p)
    }
    {
      \mathcal{K}
      \vdash
      \left(
      \begin{array}{r}
        \var{dms}\\
        \var{dws}
      \end{array}
      \right)
      \trans{adeleg}{(s,~ (\var{vk_s},~ \var{vk_d}))}
      \left(
      \begin{array}{lcl}
        \var{dms}\\
        \var{dws}
      \end{array}
      \right)
    }
  \end{equation}
  \caption{Delegation inference rules}
  \label{fig:rules:delegation}
\end{figure}

\clearpage

\subsection{Delegation sequences}
\label{sec:delegation-sequences}

This section presents the rules that model the effect that sequences of
delegations have on the ledger.

\begin{figure}[htb]
  \begin{equation}
    \inference[Initial-SDELEGS]
    {
    }
    {
      {\begin{array}{l}
       \mathcal{K}\\
        e\\
        s\\
        d
      \end{array}}
      \vdash
      \left(
        \begin{array}{l}
          \epsilon\\
          \emptyset
        \end{array}
      \right)
    }
  \end{equation}
  \nextdef
  \begin{equation}
    \label{eq:rule:delegation-scheduling-seq-base}
    \inference
    {
    }
    {
      {\begin{array}{l}
         \mathcal{K} \\
         e\\
         s\\
         d
       \end{array}}
      \vdash
      {
        \left(
          \begin{array}{l}
            \var{sds}\\
            \var{eks}
          \end{array}
        \right)
      }
      \trans{sdelegs}{\epsilon}
      {
        \left(
          \begin{array}{l}
            \var{sds}\\
            \var{eks}
          \end{array}
        \right)
      }
    }
  \end{equation}
  \nextdef
  \begin{equation}
    \label{eq:rule:delegation-scheduling-seq-ind}
    \inference
    {
      {\begin{array}{l}
         \mathcal{K} \\
         e\\
         s\\
         d
       \end{array}}
      \vdash
      {
        \left(
          \begin{array}{l}
            \var{sds}\\
            \var{eks}
          \end{array}
        \right)
      }
      \trans{sdelegs}{\Gamma}
      {
        \left(
          \begin{array}{l}
            \var{sds'}\\
            \var{eks'}
          \end{array}
        \right)
      }
      &
      {\begin{array}{l}
         \mathcal{K} \\
         e\\
         s\\
         d
       \end{array}}
      \vdash
      {
        \left(
          \begin{array}{l}
            \var{sds'}\\
            \var{eks'}
          \end{array}
        \right)
      }
      \trans{sdeleg}{c}
      {
        \left(
          \begin{array}{l}
            \var{sds''}\\
            \var{eks''}
          \end{array}
        \right)
      }
    }
    {
      {\begin{array}{l}
         \mathcal{K} \\
         e\\
         s\\
         d
       \end{array}}
      \vdash
      {
        \left(
          \begin{array}{l}
            \var{sds}\\
            \var{eks}
          \end{array}
        \right)
      }
      \trans{sdelegs}{\Gamma; c}
      {
        \left(
          \begin{array}{l}
            \var{sds''}\\
            \var{eks''}
          \end{array}
        \right)
      }
    }
  \end{equation}
  \caption{Delegation scheduling sequence rules}
  \label{fig:rules:delegation-scheduling-seq}
\end{figure}

\begin{figure}
  \begin{equation}
    \inference[Initial-ADELEGS]
    {
      \var{dms_0} \leteq \Set{k \mapsto k}{k \in \mathcal{K}} &
      \var{dws_0} \leteq \Set{k \mapsto 0}{k \in \mathcal{K}}
    }
    {
      \mathcal{K}
      \vdash
      \left(
        \begin{array}{l}
          \var{dms_0}\\
          \var{dws_0}
        \end{array}
      \right)
    }
  \end{equation}
  \nextdef
  \begin{equation}
    \label{eq:rule:delegation-seq-base}
    \inference
    {
    }
    {
      \mathcal{K}
      \vdash
      {
        \left(
          \begin{array}{l}
            \var{dms}\\
            \var{dws}
          \end{array}
        \right)
      }
      \trans{adelegs}{\epsilon}
      {
        \left(
          \begin{array}{l}
            \var{dms}\\
            \var{dws}
          \end{array}
        \right)
      }
    }
  \end{equation}
  \nextdef
  \begin{equation}
    \label{eq:rule:delegation-seq-ind}
    \inference
    {
      {
        \left(
          \begin{array}{l}
            \var{dms}\\
            \var{dws}
          \end{array}
        \right)
      }
      \trans{adelegs}{\Gamma}
      {
        \left(
          \begin{array}{l}
            \var{dms'}\\
            \var{dws'}
          \end{array}
        \right)
      }
      &
      {
        \left(
          \begin{array}{l}
            \var{dms'}\\
            \var{dws'}
          \end{array}
        \right)
      }
      \trans{adeleg}{c}
      {
        \left(
          \begin{array}{l}
            \var{dms''}\\
            \var{dws''}
          \end{array}
        \right)
      }
    }
    {
      \mathcal{K}
      \vdash
      {
        \left(
          \begin{array}{l}
            \var{dms}\\
            \var{dws}
          \end{array}
        \right)
      }
      \trans{adelegs}{\Gamma; c}
      {
        \left(
          \begin{array}{l}
            \var{dms''}\\
            \var{dws''}
          \end{array}
        \right)
      }
    }
  \end{equation}
  \caption{Delegations sequence rules }
  \label{fig:rules:delegation-seq}
\end{figure}



\section{UTxO}
\label{sec:utxo}

\subsection{Deposits and Refunds}
\label{sec:deps-refunds}

A key constraint that must be satisfied by any software that is a part of
the implementation of the
Cardano cryptocurrency is called the \textit{general accounting
property}, or the \textit{preservation of value} condition. If
condition is not satisfied, this is an indicator of
incorrect accounting, potentially due to
malicious disruption.

The preservation of value is expressed as an equality that uses values in
the ledger state and the environment, as well as the values in the body of
the signal transaction.
We have defined the rules of the delegation protocol in a way that should
consistently satisfy the presevation of value. In the future, we hope to
give a formally-verified proof that every \textit{valid} ledger state satisfies
this property.

We present the types and maps needed for deposit and
refund calculations in Figure~\ref{fig:defs:deposits}. The $\PrtclConsts$ is
an abstract type that will represent an environment variable that contains
values on which the functionality of the blockchain protocol depends, such
as the fees transactions are obligated to pay to be processed. In particular,
we define two maps here that return values stored in the set of protocol
constants. The map $\fun{dvalue}$ returns the amount a certificate must
deposit to claim a specific resource (i.e. for the declaration of a new staking
key or the registration of a stake pool).

The map $\fun{decay}$
represents the rate of decrease of the value of a unit of $\Coin$.
The constant returned by $\fun{decay}$ consists of two values. The first is
a natural number which determines the minimal proportion of a deposit that will
be refunded on resource release. The second is a positive rational number
used to determine the rate of (exponential) decrease of the value.
Recall that the $\Allocs$ type pairs a hash key with a slot number and is a generic datatype
which we use below to represent both individual and pool allocation
parameters passed to refund calculations.

For a given transaction and protocol constants, the map $\fun{dresource}$
returns all the certificates of that transaction which allocate resources
(i.e. the stake key and stake pool registration certificates). The map
$\fun{dderegister}$ returns the resource release certificates for the
deregistration of a stake key. The map which returns only
the retirement type certificates is $\fun{dretire}$. The function $\fun{ttl}$ gives the
time slot in which the validity of a given transaction will expire.

The functions used to perform the calculations needed to describe the
preservation of value
are presented in Figure~\ref{fig:functions:deposits}.
The function
$\fun{deposits}$ returns the total deposits made by a transaction. Specifically,
it sums up the values of all the deposits made by the resource-allocating
certificates of a given transaction.

The map $\fun{refund}$, given the protocol constants, current resource allocations, and
the current slot number, calculates the refund that will be issued
for a given certificate.

Whenever the certificate is for the deregistration of
an individual key or a stake pool retirement, and there are resources allocated
to the author's hash key in the $\var{allocs}$ set, a refund is issued for a
fraction of the amount deposited by the certificate. This fraction is calculated
based on the difference between the current slot number and the slot number in
which a given resource was allocated, as well as the
decay and minimum refund values indicated in the protocol constants.
Otherwise, no refund is issued.

\begin{todo}
Should we do a a check that $\delta \geq 0$ here? Or some other check?
\end{todo}

\begin{todo}
Explain delta and why we need the resource allocations at all as inputs
\end{todo}

Unlike transaction fees, which can be greater than the minimum fee, and are
indicated in the transactions themselves, the deposit amount for the registration
of an individual or stake pool is stored directly in the protocol constants,
and must be paid as an amount exactly matching the protocol's requirement.
For this reason, the refund calculation does not need to look up the deposit
amout in the data of the transaction
that paid the original registration deposit, but instead looks in the protocol
constants for this information.

The total refund for a transaction is calculated by the function $\fun{refunds}$
given a set of protocol constants and the current slot number, as well as
two sets of resource allocation parameters,
$\var{dallocs}$ and $\var{pallocs}$ of type $\Allocs$, representing
individual and pool allocations, respectively. For a signal transaction $\var{tx}$,
the $\fun{refunds}$ map
sums up the total refunds for all the stake key deregistration certificates
$\var{tx}$ is carrying, as well as the
total refunds for all the pool retirement certificates in
$\var{tx}$, and adds these two values together.

The resource allocations which the $\fun{refunds}$ calculation takes
as a parameter are sourced from the current ledger state. That is,
when we define any ledger state transition rule making use of this calculation,
$\var{dallocs}$ (the individual key allocations) will be exactly the
$\var{stkeys}$ state variable in $\DState$,
and $\var{pallocs}$ (the stake pool allocations) will be the stake pool
allocations in $\var{stpools}$ with
the certificate data removed, i.e. $\fun{poolAllocs}~\var{stpools}$.
We use here the $\fun{poolAllocs}$ map to remove them for
datatype consistency (using $\Allocs$ for all refund calculations).

For the specifics of the book-keeping of this resource allocation and
de-allocation, see the
discussion in Sections~\ref{sec:deleg-rules} and~\ref{sec:pool-rules}.

Note that the the stake pool retirement refunds are issued not when a certificate
scheduling the retirement is processed, but at the epoch boundary for which
the retirement is scheduled. If the retirement is cancelled or rescheduled
(a new registration or retirement
certificate is issued for this stake pool), a deposit adjustment
be made according rules presented in Figure~\cref{fig:pool-rules}.

Note also that that a refund for a resource-releasing certificate in a
transaction can only be issued if the refund is bigger than the minimum
transaction fee. In order to receive a refund in this case, the transaction
body must also contain inputs onto which this refund can be added.


\begin{note}
  We define $\fun{refund}$ by cases on whether or not
  the refunding certificate has a corresponding
  resource creating certificate.
  If our rules are correct, then $\fun{refund}$
  is only called in the case where such a matching
  certificate exists.
\end{note}


\begin{figure*}
  \emph{Abstract types}
  \begin{equation*}
    \begin{array}{r@{~\in~}lr}
      pc & \PrtclConsts & \text{protocol constants}
    \end{array}
  \end{equation*}
  %
  \emph{Abstract Functions}
  \begin{equation*}
    \begin{array}{r@{~\in~}lr}
      \fun{dvalue} & \PrtclConsts \to \DCert \to \Coin
        & \text{deposit amount of a certificate}\\

      \fun{decay} & \PrtclConsts \to \mathbb{N}\times\mathbb{Q}^{+}
        & \text{decay constants}\\

      \fun{dresource} & \Tx \to \powerset{(\DCertRegKey \uniondistinct \DCertRegPool)}
        & \text{resource allocating certificates}\\

      \fun{dderegister} & \Tx \to \powerset{\DCertDeRegKey}
        & \text{resource releasing certificates}\\

      \fun{dretire} & \Tx \to \powerset{\DCertRetirePool}
        & \text{resource releasing certificates}\\

      \fun{ttl} & \Tx \to \Slot
        & \text{time to live}\\
    \end{array}
  \end{equation*}
  \caption{Definitions used in Deposits}
  \label{fig:defs:deposits}
\end{figure*}

\begin{figure}
  \begin{align*}
      & \fun{deposits} \in \PrtclConsts \to \Tx \to \Coin
      & \text{total deposits for transaction} \\
      & \fun{deposits}~{pc}~{tx} = \sum\limits_{c \in \fun{dresource}~tx} (\fun{dvalue}~pc~c)
      \nextdef
      & \fun{refund} \in \PrtclConsts \to \Allocs \to \Slot \to \DCert \to \Coin
      & \text{total refund for a certificate} \\
      & \refund{allocs}{pc}{slot}{c} =\\
      & \begin{cases}
        0 & \text{if not}~(\fun{releasing}~c)\\
            0 & \text{if}~\cauthor c \notin allocs\\
            \floor*{
              \left(\fun{dvalue}~pc~c\right) \cdot
            \left(d_{\min}+(1-d_{\min})\cdot e^{-\lambda\cdot\delta}\right)}
            & \text{otherwise}
        \end{cases}\\
      &
      \begin{array}{lr@{~=~}l}
        \where &\fun{releasing}~\var{c} & \DeregKey{c} \lor \RetirePool{c}\\
        & d_{\min},~\lambda & \fun{decay}~pc\\
        &\delta & \slotminus{slot}{(allocs~(\cauthor c))}\\
      \end{array}\\
      \nextdef
      & \fun{refunds} \in \PrtclConsts \to \Allocs \to \Allocs \to \Tx \to \Coin
      & \text{total refunds for transaction} \\
      & \refunds{pc}{stkeys}{stpools}{tx} =\\
      &   \sum\limits_{c \in \fun{dderegister}~tx} \refund{pc}{stkeys}{(\ttl{tx})}{c}\\
      &   ~~~+ \sum\limits_{c \in \fun{dretire}~tx} \refund{pc}{stpools}{(\retire{c})}{c}
  \end{align*}
  \caption{Functions used in Deposits}
  \label{fig:functions:deposits}
\end{figure}



\subsection{UTxO Transitions}
\label{sec:state-trans-utxo-1}


The types involved in defining a UTxO and its transitions are defined in
Figure~\ref{fig:defs:utxo}. A transaction $\var{tx}$ (a term of type $\Tx$)
contains a list of inputs (a list of terms of type $\TxIn$), and outputs
(a finite map $\Ix \mapsto \TxOut$). Each input in the list
corresponds to those $\TxOut$ terms in the outputs list indexed by the same
$\Ix$ as in the second coordinate of the input. This way, a transaction can have
inputs from multiple prior transactions and outputs to multiple addresses
from each of the input transactions.

The map $\var{txid}$ is a one-way function which computes the $\var{txid}$
of a given transaction. A transaction can be uniquely identified
by its $\var{txid}$ in a given context.

Now, the actual type $\Tx$ of a transaction is likely to necessitate additional data
in an actual implementation. For example, transaction fees (discussed below)
or certificates to be applied
may need to be part of the transaction data. Much of the time, however, when defining
properties in this document, we want to refer to only the core of the transaction.
For this reason, we make use here
of the $\fun{txbody}$ map. This map returns only the data in the given
transaction pertinent to the transfer of coins (i.e. the lists of inputs and outputs)

The transaction fee $\fun{txfee}$ is the
fee a transaction contributes to the system (this value depends only on
the transaction itself). The fee does not need to have an explicit output
specified, and are currently implicitly included in the total input value of
a transaction.

The $\fun{minfee}$ is the minimum
fee that must be paid by a transaction to be applied to the current UTxO.
This minimum fee
is calculated based on the context of the transaction (i.e. a collection of
values the blockchain protocol keeps track of, $\PrtclConsts$), as well as
the contents of the transaction itself.


\begin{figure*}
  \emph{Abstract types}
  %
  \begin{equation*}
    \begin{array}{r@{~\in~}lr}
      \var{txid} & \TxId & \text{transaction id}\\
      %
      ix & \Ix & \text{index}\\
      %
      \var{addr} & \Addr & \text{address}\\
      %
      c & \Coin & \text{currency value}\\
      %
      slot & \Slot & \text{slot}
    \end{array}
  \end{equation*}
  \emph{Derived types}
  %
  \begin{equation*}
    \begin{array}{r@{~\in~}l@{\qquad=\qquad}r@{~\in~}lr}
      \var{txin}
      & \TxIn
      & (\var{txid}, \var{ix})
      & \TxId \times \Ix
      & \text{transaction input}
      \\
      \var{txout}
      & \type{TxOut}
      & (\var{addr}, c)
      & \Addr \times \Coin
      & \text{transaction output}
      \\
      \var{utxo}
      & \UTxO
      & \var{txin} \mapsto \var{txout}
      & \TxIn \mapsto \TxOut
      & \text{unspent tx outputs}
    \end{array}
  \end{equation*}
  %
  \emph{Abstract Functions}
  \begin{equation*}
    \begin{array}{r@{~\in~}lr}
      \txid{} & \Tx \to \TxId & \text{compute transaction id}\\
      %
      \fun{txbody} & \Tx \to \powerset{\TxIn} \times (\Ix \mapsto \TxOut)
                                  & \text{transaction body}\\
      %
      \fun{txfee} & \Tx \to \Coin & \text{transaction fee}\\
      %
      \fun{minfee} & \PrtclConsts \to \Tx \to \Coin & \text{minimum fee}
    \end{array}
  \end{equation*}
  \caption{Definitions used in the UTxO transition system}
  \label{fig:defs:utxo}
\end{figure*}


A set of functions on UTxOs and transactions appearing in this document,
along with their types are defined in Figure~\ref{fig:derived-defs:utxo}.
The map $\fun{txins}$ returns the list of inputs of a given transactions, i.e.
the pairs type $\TxIn$ within that transaction.

The map $\fun{txouts}~ \var{tx}$ builds a UTxO (finite map) by taking the
list of inputs of $\var{tx}$, replacing the original $\var{txid}$ in the pair
with the $\var{txid}$ of $\var{tx}$, and mapping it to the output with the
corresponding index in the list of outputs of $\var{tx}$ (thereby making a list
of the unspent outputs of \textit{this} transaction).

The $\fun{balance}$ map, as expected, gives the sum total of all the coin in
a given UTxO. The calculation $\fun{created}~ \var{tx}$ gives the value created
by the transaction $\var{tx}$ in the context of the relevant protocol
constants, the current UTxO on the ledger, and the resource allocation for
$\DState$ and $\PState$. The calculation is a sum of all coin in the inputs of
$\var{tx}$ plus the refunds of the $\var{tx}$ in this context.

The $\fun{destroyed}~ \var{pc}~ \var{tx}$ calculation, on the other hand, sums up the
coin in the outputs of $\var{tx}$, then adds it to the fee that the transaction
has paid and the deposits made by the transaction for the certificates
registering stake pools. This calculation gives the total coin value that
the transaction removes from the current UTxO (given the necessary protocol
constants).


\begin{figure}
  \begin{align*}
    & \fun{txins} \in \Tx \to \powerset{\TxIn}
    & \text{transaction inputs} \\
    & \txins{tx} = \var{inputs} \where \txbody{tx} = (\var{inputs}, ~\wcard)
    \nextdef
    & \fun{txouts} \in \Tx \to \UTxO
    & \text{transaction outputs as UTxO} \\
    & \fun{txouts} ~ \var{tx} =
      \left\{ (\fun{txid} ~ \var{tx}, \var{ix}) \mapsto \var{txout} ~
      \middle| \begin{array}{l@{~}c@{~}l}
                 (\_, \var{outputs}) & = & \txbody{tx} \\
                 \var{ix} \mapsto \var{txout} & \in & \var{outputs}
               \end{array}
      \right\}
    \nextdef
    & \fun{balance} \in \UTxO \to \Coin
    & \text{UTxO balance} \\
    & \fun{balance} ~ utxo = \sum_{(~\wcard ~ \mapsto (\wcard, ~c)) \in \var{utxo}} c
    \nextdef
    & \fun{created} \in \PrtclConsts \to \UTxO \to \Allocs \to \Allocs \to \Tx \to \Coin
    & \text{value created} \\
    & \created{pc}{utxo}{stkeys}{stpools}{tx} = \\
    & ~~\balance{(\txins{tx} \restrictdom \var{utxo})} + \refunds{pc}{stkeys}{stpools}{tx}
    \nextdef
    & \fun{destroyed} \in \PrtclConsts \to \Tx \to \Coin
    & \text{value destroyed} \\
    & \fun{destroyed} ~ pc ~ tx =
      \balance{(\txouts{tx})}  + \txfee{tx} + \deposits{pc}{tx}\\
  \end{align*}

  \begin{align*}
    \var{ins} \restrictdom \var{utxo}
    & = \{ i \mapsto o \mid i \mapsto o \in \var{utxo}, ~ i \in \var{ins} \}
    & \text{domain restriction}
    \\
    \var{ins} \subtractdom \var{utxo}
    & = \{ i \mapsto o \mid i \mapsto o \in \var{utxo}, ~ i \notin \var{ins} \}
    & \text{domain exclusion}
    \\
    \var{utxo} \restrictrange \var{outs}
    & = \{ i \mapsto o \mid i \mapsto o \in \var{utxo}, ~ o \in \var{outs} \}
    & \text{range restriction}
  \end{align*}
  \caption{Functions used in UTxO rules}
  \label{fig:derived-defs:utxo}
\end{figure}


The type of the transition of a UTxO is presented in Figure
~\ref{fig:ts-types:utxo}. A
transition can valid in the context of a given environment, $\UTxOEnv$.
This environment is made up of several variables.
The $\PrtclConsts$ is needed as part of the environment here in order to
keep track of the $\fun{minfee}$ value (as well as
to perform other calculations). The stake key and stake pool resource allocations
are needed in order to compute the $\fun{created}$ and
$\fun{destroyed}$ values of a given $\var{tx}$. The current slot number
is used in the inference rules defining valid transitions.
The $\var{tx}$ here is the
signal that triggers the transition.


\begin{figure}
  \emph{UTxO environment}
  \begin{equation*}
    \UTxOEnv =
    \left(
      \begin{array}{r@{~\in~}lr}
        \var{slot} & \Slot & \text{current slot}\\
        \var{pc} & \PrtclConsts & \text{protocol constants}\\
        \var{stkeys} & \Allocs & \text{stake key allocations}\\
        \var{stpools} & \Allocs & \text{stake pool allocations}\\
      \end{array}
    \right)
  \end{equation*}
  %
  \emph{UTxO transitions}
  \begin{equation*}
    \_ \vdash
    \var{\_} \trans{utxo}{\_} \var{\_}
    \subseteq \powerset (\UTxOEnv \times \UTxO \times \Tx \times \UTxO)
  \end{equation*}
  \caption{UTxO transition-system types}
  \label{fig:ts-types:utxo}
\end{figure}


The inference rule for unspent outputs are presented in
Figure~\ref{fig:rules:utxo}.
Rule~\ref{eq:utxo-inductive} specifies the conditions under which a transaction can
be applied to a set of unspent outputs in a given environment, and how the set
of unspent outputs changes as a resul:

\begin{itemize}
\item The trasaction is live (its time to live is less than the current slot)
\item The transaction has at least one input
\item Each input spent in the transaction must be in the set of unspent
  outputs.
\item The fee paid by the transaction has to be greater than or equal to the
minimum fee.
\item The amount of coin created by the transaction must be the same as
the amount destroyed (i.e. the \textit{preservation of value} must hold)
\item If the above conditions hold, then the new state will not have the inputs
  spent in transaction $\var{tx}$ and it will have the new outputs in
  $\var{tx}$.
\end{itemize}

Here we again emphasize the importance of the \textit{preservation of value}
condition. No valid UTxO state can be reached without this condition
being satisfied at every step of the transitions leading to a given UTxO state.
Since the total amout of Ada at genesis is known, this property can be
checked for arbitrarily generated valid ledgers.

Note that the fee paid by a transaction is required to be greater than or
equal to the minim fee. The reason for allowing it to be greater is to have
the option for the implementation to make processing decisions based on the
size of the fee. Such as, a larger fee would result in faster processing of
a transaction.

The transaction is required to have at least one input to avoid a situation
where the witnessing replay protection may fail. Specifically, when a transaction is
spending only refunds from deregistration or retirement certificates it is
carrying.

\begin{todo}
  Explain exactly how refunds get spent (how the ledger is updated to include
  them as unspent outputs)
\end{todo}

According to this rule, when a UTxO state update is triggered by a
transaction (signal) $\var{tx}$, and the above conditions are met, the UTxO
changes as follows:

\begin{itemize}
\item remove from the UTxO all the $(\var{txin}, \var{txout})$ pairs
associated with the $\var{txins}$'s in the $\var{inputs}$ list of $\var{tx}$.
\item add all the $\var{outputs}$ of $\var{tx}$ to the
UTxO, now associated with the $\var{txid}(\var{tx})$ instead.
\end{itemize}


\begin{figure}
  \begin{equation}\label{eq:utxo-inductive}
    \inference[UTxO-inductive]
    { \ttl tx \leq \var{slot}
      & \txins{tx} \neq \emptyset
      \\
      \txins{tx} \subseteq \dom \var{utxo}
      & \minfee{pc}{tx} \leq \txfee{tx}
      \\
      \created{pc}{utxo}{stkeys}{stpools}{tx} = \destroyed{pc}{tx}
    }
    {
      \begin{array}{l}
        \var{slot}\\
        \var{pc}\\
        \var{stkeys}\\
        \var{stpools}\\
      \end{array}
      \vdash \var{utxo} \trans{utxo}{tx}
      (\txins{tx} \subtractdom \var{utxo}) \cup \txouts{tx}
    }
  \end{equation}
  \caption{UTxO inference rules}
  \label{fig:rules:utxo}
\end{figure}


\begin{note}
  $\Coin$ is defined as a primitive type, but there is a difference
  between implementing it with $\mathbb{N}$ versus $\mathbb{Z}$.
  Since this is a pure UTxO ledger, $\mathbb{N}$ suffices.
  If, however, $\mathbb{Z}$ is used, then extra validation is required
  to ensure that all $\TxOut$ are non-negative.
  This extra condition would be added to \cref{eq:utxo-inductive}.
\end{note}

\subsection{Properties}
\label{sec:utxo-properties}

\begin{todo}
  Can we prove properties of the transition system of this section? For
  instance we might like to formalize ``double spending'' and prove that these
  rules prevent it. Do we want it?
\end{todo}


\subsection{Witnesses}
\label{sec:witnesses}

The definitions needed to add witnessing to the UTxO transitions described above
are presented in Figure~\ref{fig:defs:utxow}. A transaction is witnessed by
a signature and a verification key corresponding to this signature. The
$\fun{hashKey_{spend}}$ map associates to an address the hash of the spending key
at this address.

\begin{figure}
  \emph{Abstract functions}
  %
  \begin{equation*}
    \begin{array}{r@{~\in~}lr}
      \fun{wits} & \Tx \to \powerset{(\VKey \times \Sig)}
      & \text{witnesses of a transaction}\\
      \fun{hashKey_{spend}} & \Addr \mapsto \HashKey
      & \text{hashKey of a spending key in an address}\\
    \end{array}
  \end{equation*}
  \caption{Definitions used in the UTxO transition system with witnesses}
  \label{fig:defs:utxow}
\end{figure}

In Figure~\ref{fig:derived-defs:utxow}, we give the definitions of two maps
associated with a UTxO. The first, $\fun{addr}$, gives the finite map
which associates to a given $\var{txin}$ in the UTxO the address of the
corresponding $\var{txout}$. The map $\fun{addr_h}$, given a UTxO, returns the
finite map which associates $\var{txin}$ with the hash of a spending key in
the address in the corresponding $\var{txout}$ in the UTxO.

\begin{figure}
  \begin{align*}
    & \addr{}{} \in \UTxO \to \TxIn \mapsto \Addr & \text{address of an input}\\
    & \addr{utxo} = \{ i \mapsto a \mid i \mapsto (a, \wcard) \in \var{utxo} \} \\
    \nextdef
    & \fun{addr_h} \in \UTxO \to \TxIn \mapsto \HashKey & \text{hashKey of an input address}\\
    & \fun{addr_h}~utxo = \{ i \mapsto h \mid i \mapsto (a, \wcard) \in \var{utxo}
      \wedge a \mapsto h \in \fun{hashKey_{spend}} \}
  \end{align*}
  \caption{Functions used in rules witnesses}
  \label{fig:derived-defs:utxow}
\end{figure}


Note that the UTxO transitions with and without witnesses have the same type
(see Figure~\ref{fig:ts-types:utxo} and Figure~\ref{fig:ts-types:utxow}).
The witnessed transition rule, in fact, defines the same UTxO update as the
non-witnessed rule. Essentially, the witnessed rule says we may only apply the
the non-witnessed rule whenever the following
additional precondition is satisfied
(stated in Rule~\ref{eq:utxo-witness-inductive} in
Figure~\ref{fig:rules:utxow}):

\begin{itemize}
 \item For each input $\var{i}$ of the signal transaction $\var{tx}$,
 there exists a verifiable witness (there could be more than one valid
 key-signature pair serving as a witness) such the hash of the spending key
 in the address corresponding to the output of $\var{i}$ in the current UTxO
 is the same as the hash of the verification key of the witness.
\end{itemize}

The purpose of witnessing is make sure that the intended action is authorized by
the holder of the signing key, providing replay protection as a
consequence. 
Due to the nature
of UTxO stype accounting, this replay protection is needed only to prevent
the re-application of old certificates. The way this is
achieved by the above precondition is by requiring a valid signature on
the coin transfers made by the transaction (the body of the transaction).

We enforce the condition here that the list of inputs of a transaction can never be empty.
So, since all the inputs of an applied transaction have already been included
in the current UTxO, it may not be replayed with the same inputs. Thus,
it must also have a different signature. This would prevent it from
being applied to the UTxO.

\begin{figure}
  \emph{UTxO with witness transitions}
  \begin{equation*}
    \var{\_} \vdash
    \var{\_} \trans{utxow}{\_} \var{\_}
    \subseteq \powerset (\UTxOEnv \times \UTxO \times \Tx \times \UTxO)
  \end{equation*}
  \caption{UTxO with witness transition-system types}
  \label{fig:ts-types:utxow}
\end{figure}


\begin{figure}
  \begin{equation}
    \label{eq:utxo-witness-inductive}
    \inference[UTxO-wit]
    {
      {
        \begin{array}{l}
        \var{slot}\\
        \var{pc}
        \end{array}
      }
      \vdash \var{utxo} \trans{utxo}{tx} \var{utxo'}\\ ~ \\
      & \forall i \in \txins{tx} \cdot \exists (\var{vk}, \sigma) \in \wits{\var{tx}}
      \cdot
      \mathcal{V}_{\var{vk}}{\serialised{\txbody{tx}}}_{\sigma}
      \wedge  \fun{addr_h}~{utxo}~i = \hashKey{vk}\\
    }
    {
      \begin{array}{l}
        \var{slot}\\
        \var{pc}\\
      \end{array}
      \vdash \var{utxo} \trans{utxow}{tx} \var{utxo'}
    }
  \end{equation}
  \caption{UTxO with witnesses inference rules}
  \label{fig:rules:utxow}
\end{figure}

\subsection{Ledger State Transition}
\label{sec:ledger}

Having formalized transition types and rules of the state of the UTxO
on the ledger, the stake pools, and the delegations on the ledger, we can
now combine these into a single a ledger transition type (and inference rule).

The full ledger state consists of the $\UTxO$, $\DState$, and $\PState$ (see Figure~\ref{fig:ts-types:ledger}). In order to describe every possible valid
ledger state, we must also include necessary environment variables in our
inference rules. Note that different environments (e.g. different transaction
fees) result in distinct sets of valid states. The environment variables
we must track are the current slot number and the protocol constants.

\begin{figure}
  \emph{Ledger environment}
  \begin{equation*}
    \LEnv =
    \left(
      \begin{array}{r@{~\in~}lr}
        \var{slot} & \Slot & \text{current slot}\\
        \var{pc} & \PrtclConsts & \text{protocol constants}\\
      \end{array}
    \right)
  \end{equation*}
  %
  \emph{Ledger state}
  \begin{equation*}
    \LState =
    \left(
      \begin{array}{r@{~\in~}lr}
        \var{utxo} & \UTxO & \text{UTxO}\\
        \var{dwstate} & \DWState & \text{delegation witnesess state}\\
      \end{array}
    \right)
  \end{equation*}
  %
  \emph{Ledger transitions}
  \begin{equation*}
    \_ \vdash
    \var{\_} \trans{ledger}{\_} \var{\_}
    \subseteq \powerset (\LEnv \times \LState \times \Tx \times \LState)
  \end{equation*}
  \caption{Ledger transition-system types}
  \label{fig:ts-types:ledger}
\end{figure}


The inference rule describing the complete ledger state transition is given in
Figure~\ref{fig:rules:ledger}. It states that we may combine the transitions
of $\UTxO$ and $\DWState$ triggered by the a signal transaction
in a particular context into a single transition that follows the transition
rules of all three components.

Note that in our complete ledger state transition, $\var{stkeys}$ and
$\var{stpools}$ are part of the environment of the antecedent UTxO transition,
but are in the set of state variables of the total ledger transition (as well
as the DELEGS transition). This is because this information is needed in order
to update the UTxO, but is not part of the UTxO state and is not affected by the
update.

Here, again, we have a case with two non-identical signals in antecedent
transitions to define the consequent transition. However, in this case,
this is presented in such a way for clarity and ease of reading. The $\DWState$
transition could have been expressed as a transition signaled by a transaction
$\var{tx}$, just like the UTxO transition. Instead, we stripped away the
irrelevant content in the transaction for a delegation state transition,
considering only the list of certificates in $\var{tx}$, which affect which the delegations
and stake pools.



\begin{figure}
  \begin{equation}
    \label{eq:ledger}
    \inference[ledger]
    {
      \Gamma = \dcerts{tx} \\ ~ \\
      {
        \begin{array}{r}
        slot\\
        pc\\
        stkeys\\
        stpools\\
        \end{array}
      }
      \vdash \var{utxo} \trans{utxow}{tx} \var{utxo'}\\~\\~\\
      %
      {
        \begin{array}{l}
          tx \\
          slot \\
        \end{array}
      }
      \vdash
      dwstate \trans{delegs}{\Gamma} dwstate'
    }
    {
      \begin{array}{l}
        \var{slot}\\
        \var{pc}\\
      \end{array}
      \vdash
      \left(
        \begin{array}{ll}
          utxo \\
          dwstate \\
        \end{array}
      \right)
      \trans{ledger}{tx}
      \left(
        \begin{array}{ll}
          utxo' \\
          dwstate' \\
        \end{array}
      \right)
    }
  \end{equation}
  \caption{Ledger inference rule}
  \label{fig:rules:ledger}
\end{figure}

\section{Properties}
\label{sec:properties}

\begin{todo}
  discuss properties
\end{todo}

\section{Formal Properties}
\label{sec:properties}

This appendix collects the main formal properties that the new ledger rules are expected to satisfy.

\begin{enumerate}[label=P{\arabic*}:\ ]
\item
  \emph{Consistency with Shelley.}
  All formal properties that were defined for Shelley in \ref{XX} remain valid.\todo{Confirm this, or list any exceptions.}
\item
  \emph{Consistency with Multi-Asset.}
  All formal properties that were defined for multi-asset tokens in \ref{XX} also remain valid.\todo{Confirm this, or list any exceptions.}
\item
  \emph{Ada Consistency.}
  Any token with the $\PolicyID$ of Ada is an Ada token, i.e. it
  also has the $\AssetID$ of Ada.
\item
  \emph{General Accounting.}
  The \emph{general accounting} property\todo{Reference or explain this} holds for any transaction,
  whether it is fully processed or just paying fees. In particular,
  this implies that the total amount of Ada in the system is constant.
\item
  \emph{Fee Movement.}
  If a transaction is accepted and marked as paying fees only
  (i.e. $\fun{txvaltag}\, tx = \True$), then the only change to the ledger
  when processing the transaction is that the inputs marked for paying
  fees are moved to the fee pot.
\item
  \emph{Transaction Validation.}
  If a Shelley transaction is accepted, it is fully processed.\todo{What does this mean? Is this: Transaction consistency - any transaction that is successfully validated can be executed successfully by the interpreter.}
\item
  \emph{Extended UTxO Validation.}
  If a transaction extends the UTxO, all its scripts validate, and
  if it has a script that does not validate, it cannot extend the
  UTxO.
\item
  \emph{Non-Token Forging.}
  A valid transaction that does not forge tokens satisfies the
  accounting property of the Shelley ledger where the type $\Coin$ is
  replaced by $\Value$.
\item
  \todo{\emph{Script consistency.} Scripts are only processed by valid version of the interpreter.}
\item
  \todo{\emph{Backwards compatibility.} All scripts that could be processed by any previous version of the interpreter
    remain valid indefinitely.}
\item
  \todo{\emph{Cost consistency.} - no transaction exceeds the specified resource bounds.}
\item
  \todo{\emph{Backwards Compatibility.} Any transaction that was accepted in a previous version of the ledger rules
    has exactly the same cost and effect, except that the transaction output is extended.}
\item
  ... \todo{Anything else?}
\end{enumerate}


\addcontentsline{toc}{section}{References}
\bibliographystyle{plainnat}
\bibliography{references}

\end{document}
