\documentclass[11pt,a4paper]{article}
\usepackage[margin=2.5cm]{geometry}
\usepackage{iohk}
\usepackage{microtype}
\usepackage{mathpazo} % nice fonts
\usepackage{amsmath}
\usepackage{amssymb}
\usepackage{latexsym}
\usepackage{mathtools}
\usepackage{stmaryrd}
\usepackage{extarrows}
\usepackage{slashed}
\usepackage[colon]{natbib}
\usepackage[unicode=true,pdftex,pdfa]{hyperref}
\usepackage{xcolor}
\usepackage[capitalise,noabbrev,nameinlink]{cleveref}
\usepackage{float}
\floatstyle{boxed}
\restylefloat{figure}

%%
%% Package `semantic` can be used for writing inference rules.
%%
\usepackage{semantic}
%% Setup for the semantic package
\setpremisesspace{20pt}

%%
%% Types
%%
\newcommand{\Tx}{\type{Tx}}
\newcommand{\Ix}{\type{Ix}}
\newcommand{\TxId}{\type{TxId}}
\newcommand{\Addr}{\type{Addr}}
\newcommand{\UTxO}{\type{UTxO}}
\newcommand{\Value}{\type{Value}}
\newcommand{\Coin}{\type{Coin}}
\newcommand{\PrtclConsts}{\type{PrtclConsts}}
%% Adding witnesses
\newcommand{\TxIn}{\type{TxIn}}
\newcommand{\TxOut}{\type{TxOut}}
\newcommand{\VKey}{\type{VKey}}
\newcommand{\SKey}{\type{SKey}}
\newcommand{\Hash}{\type{Hash}}
\newcommand{\SkVk}{\type{SkVk}}
\newcommand{\Sig}{\type{Sig}}
\newcommand{\Data}{\type{Data}}
%% Adding delegation
\newcommand{\Epoch}{\type{Epoch}}
\newcommand{\VKeyGen}{\type{VKeyGen}}
%% Blockchain
\newcommand{\Gkeys}{\var{G_{keys}}}
\newcommand{\Block}{\type{Block}}
\newcommand{\SlotId}{\type{SlotId}}
\newcommand{\CEEnv}{\type{CEEnv}}
\newcommand{\CEState}{\type{CEState}}
\newcommand{\BDEnv}{\type{BDEnv}}
\newcommand{\BDState}{\type{BDState}}

%%
%% Functions
%%
\newcommand{\txins}[1]{\fun{txins}~ \var{#1}}
\newcommand{\txid}[1]{\fun{txid}~ \var{#1}}
\newcommand{\txouts}[1]{\fun{txouts}~ \var{#1}}
\newcommand{\values}[1]{\fun{values}~ #1}
\newcommand{\balance}[1]{\fun{balance}~ \var{#1}}
%% UTxO witnesses
\newcommand{\inputs}[1]{\fun{inputs}~ \var{#1}}
\newcommand{\wits}[1]{\fun{wits}~ \var{#1}}
\newcommand{\verify}[3]{\fun{verify} ~ #1 ~ #2 ~ #3}
\newcommand{\sign}[2]{\fun{sign} ~ #1 ~ #2}
\newcommand{\serialised}[1]{\llbracket \var{#1} \rrbracket}
\newcommand{\addr}[1]{\fun{addr}~ \var{#1}}
\newcommand{\hash}[1]{\fun{hash}~ \var{#1}}
\newcommand{\txbody}[1]{\fun{txbody}~ \var{#1}}
\newcommand{\txfee}[1]{\fun{txfee}~ \var{#1}}
\newcommand{\minfee}[2]{\fun{minfee}~ \var{#1}~ \var{#2}}
% wildcard parameter
\newcommand{\wcard}[0]{\underline{\phantom{a}}}
%% Adding ledgers...
\newcommand{\utxo}[1]{\fun{utxo}~ #1}
%% Delegation
\newcommand{\delegatesName}{\fun{delegates}}
\newcommand{\delegates}[3]{\delegatesName~#1~#2~#3}
\newcommand{\dwho}[1]{\fun{dwho}~\var{#1}}
\newcommand{\depoch}[1]{\fun{depoch}~\var{#1}}
%% Delegation witnesses
\newcommand{\dbody}[1]{\fun{dbody}~\var{#1}}
\newcommand{\dwit}[1]{\fun{dwit}~\var{#1}}
%% Blockchain
\newcommand{\bwit}[1]{\fun{bwit}~\var{#1}}
\newcommand{\bslot}[1]{\fun{bslot}~\var{#1}}
\newcommand{\bbody}[1]{\fun{bbody}~\var{#1}}
\newcommand{\bdlgs}[1]{\fun{bdlgs}~\var{#1}}

\begin{document}

\hypersetup{
  pdftitle={Formal Specification of the Cardano Ledger with a Native
  Multicurrency Implementation},
  breaklinks=true,
  bookmarks=true,
  colorlinks=false,
  linkcolor={blue},
  citecolor={blue},
  urlcolor={blue},
  linkbordercolor={white},
  citebordercolor={white},
  urlbordercolor={white}
}

\title{Formal Specification of the Cardano Ledger with a Native
Multicurrency Implementation}

\author{
   Polina Vinogradova \\ {\small \texttt{polina.vinogradova@iohk.io}} \\
   }

\date{}

\maketitle

\begin{abstract}
This document presents the modifications of the Shelley ledger
specification
(see~\cite{shelley_spec}) which will enable it to support native
Multicurrency (see~\cite{multi_currency} and~\cite{formal_multicur})
using a small scripting language fully specified
by the ledger rules.
\end{abstract}

\section*{List of Contributors}
\label{acknowledgements}

Duncan Coutts,
Philipp Kant,
Michal Peyton Jones,
Jann Mueller,
Jared Corduan,
Matthias Gudemann,
Manuel Chakravarty,
Kevin Hammond


\tableofcontents
\listoffigures

\section{Introduction}
\label{sec:introduction}
\section{Introduction}
\label{sec:introduction}

This specification models the \textit{conditions} that the different parts of a
transaction have to fulfill so that they can extend a ledger, which is
represented here as a list of transactions. In particular, we model the
following aspects:

\begin{description}
\item[Preservation of value] relationship between the total value of input and
  outputs in a new transaction, and the unspent outputs.
\item[Witnesses] authentication of parts of the transaction data by means of
  cryptographic entities (such as signatures and private keys) contained in
  these transactions.
\item[Delegation] validity of delegation certificates, which delegate
  block-signing rights.
\item[Update validation] voting mechanism which captures the identification of
  the voters, and the participants that can post update proposals.
\end{description}

The following aspects will not be modeled (since they are not part of the Byron
release):
\begin{description}
\item[Stake] staking rights associated to an addresses.
\end{description}


\section{Notation}\label{sec:notation}

\begin{description}
\item[Powerset] Given a set $\type{X}$, $\powerset{\type{X}}$ is the set of all
  the subsets of $X$.
\item[Sequences] Given a set $\type{X}$, $\seqof{\type{X}}$ is the set of
  sequences having elements taken from $\type{X}$. The empty sequence is
  denoted by $\epsilon$, and given a sequence $\Lambda$, $\Lambda; \type{x}$ is
  the sequence that results from appending $\type{x} \in \type{X}$ to
  $\Lambda$.
\item[Functions] $A \to B$ denotes a \textbf{total function} from $A$ to $B$.
  Given a function $f$ we write $f~a$ for the application of $f$ to argument
  $a$.
\item[Fibre] Given a function $f: A \to B$ and $b\in B$, we write
  $f^{-1}~b$ for the \textbf{fibre} of $f$ at $b$, which is defined by
  $\{a \mid\ f a =  b\}$.
\item[Maps and partial functions] $A \mapsto B$ denotes a \textbf{partial
    function} from $A$ to $B$, which can be seen as a map (dictionary) with
  keys in $A$ and values in $B$. Given a map $m \in A \mapsto B$, notation
  $a \mapsto b \in m$ is equivalent to $m~ a = b$.
\end{description}

\section{Cryptographic primitives}
\label{sec:crypto-primitives}

Figure~\ref{fig:crypto-defs} introduces the cryptographic abstractions used in
this document. The cryptographic concepts required for the formal definition
of delegation with with witnesses include public-private key pairs, hash
maps and signatures. The constraint we introduce states that a signature of
some data signed with a (private) key is only correct whenever we can verify
it using the corresponding public key.

\begin{figure}
  \emph{Abstract types}
  %
  \begin{equation*}
    \begin{array}{r@{~\in~}lr}
      \var{sk} & \SKey & \text{private signing key}\\
      \var{vk} & \VKey & \text{public verifying key}\\
      \var{hk} & \Hash & \text{hash of a key}\\
      \sigma & \Sig  & \text{signature}\\
      \var{d} & \Data  & \text{data}\\
    \end{array}
  \end{equation*}
  \emph{Derived types}
  \begin{equation*}
    \begin{array}{r@{~\in~}lr}
      (sk, vk) & \SkVk & \text{signing-verifying key pairs}
    \end{array}
  \end{equation*}
  \emph{Abstract functions}
  %
  \begin{equation*}
    \begin{array}{r@{~\in~}lr}
      \hash{} & \VKey \to \Hash
      & \text{hash function} \\
      %
      \fun{verify} & \powerset{\VKey \times \Data \times \Sig}
      & \text{verification relation}\\
    \end{array}
  \end{equation*}
  \emph{Constraints}
  \begin{align*}
    & \forall (sk, vk) \in \SkVk,~ m \in \Data,~ \sigma \in \Sig,
      \verify{vk}{m}{\sigma} \iff \sign{sk}{m} = \sigma
  \end{align*}
  \emph{Notation for serialized and verified data}
  \begin{align*}
    & \serialised{x} & \text{serialised representation of } x\\
    & \mathcal{V}_{\var{vk}}{\serialised{m}}_{\sigma} = \verify{vk}{m}{\sigma}
      & \text{shorthand notation for } \fun{verify}
  \end{align*}
  \caption{Cryptographic definitions}
  \label{fig:crypto-defs}
\end{figure}

\section{Serialization}
\label{sec:serialization}

\begin{todo}
  Discuss here serialization and
  \href{https://iohk.myjetbrains.com/youtrack/issue/CDEC-628}{composable
    serialization}
\end{todo}

\section{UTxO}
\label{sec:state-trans-utxo-1}

The types involved in defining a UTxO and transitions thereof are defined in
Figure~\ref{fig:defs:utxo}. The map $\var{txid}$, computes a hash of a given
transaction to give the ID.
The transaction fee $\fun{txfee}$ is the actual
fee a transaction contributes to the system (this value depends only on
the transaction itself), whereas $\fun{minfee}$ is the minimum
fee required to apply a given transaction to the current UTxO. This minimum
depends additionally on the context of this transaction (i.e. a collection of
values the blockchain protocol keeps track of, $\PrtclConsts$).

A set of functions on UTxOs and transactions which we will require here,
along with their types are defined in Figure~\ref{fig:derived-defs:utxo}.
The type of the transition on UTxO is presented in~\ref{fig:ts-types:utxo}.
The $\PrtclConsts$ is needed as part of the context here in order to be
able to determine and make use of the $\fun{minfee}$ value.

The transition rules for unspent outputs are presented in
Figure~\ref{fig:rules:utxo}.

\begin{figure*}
  \emph{Primitive types}
  %
  \begin{equation*}
    \begin{array}{r@{~\in~}lr}
      \var{txid} & \TxId & \text{transaction id}\\
      %
      ix & \Ix & \text{index}\\
      %
      \var{addr} & \Addr & \text{address}\\
      %
      c & \Coin & \text{currency value}\\
      %
      pc & \PrtclConsts & \text{protocol constants}
    \end{array}
  \end{equation*}
  \emph{Derived types}
  %
  \begin{equation*}
    \begin{array}{r@{~\in~}l@{\qquad=\qquad}r@{~\in~}lr}
      \var{txin}
      & \TxIn
      & (\var{txid}, \var{ix})
      & \TxId \times \Ix
      & \text{transaction input}
      \\
      \var{txout}
      & \type{TxOut}
      & (\var{addr}, c)
      & \Addr \times \Coin
      & \text{transaction output}
      \\
      \var{utxo}
      & \UTxO
      & \var{txin} \mapsto \var{txout}
      & \TxIn \mapsto \TxOut
      & \text{unspent tx outputs}
      \\
      \var{tx}
      & \Tx
      & (\var{inputs}, \var{outputs})
      & \powerset{(\TxIn)} \times (\Ix \mapsto \TxOut)
      & \text{transaction}

    \end{array}
  \end{equation*}
  %
  %\emph{Abstract types}
  %\begin{equation*}
%    \begin{array}{r@{~\in~}lr}
%      \var{tx} & \Tx & \text{transaction}\\
%    \end{array}
%  \end{equation*}
  %
  \emph{Abstract Functions}
  \begin{equation*}
    \begin{array}{r@{~\in~}lr}
      \txid{} & \Tx \to \TxId & \text{compute transaction id}\\
      %
      %\fun{txbody} & \Tx \to \powerset{\TxIn} \times (\Ix \mapsto \TxOut)
      %                            & \text{transaction body}\\
      %
      \fun{txfee} & \Tx \to \Coin & \text{transaction fee}\\
      %
      \fun{minfee} & \PrtclConsts \to \Tx \to \Coin & \text{minimum fee}
    \end{array}
  \end{equation*}
  \caption{Definitions used in the UTxO transition system}
  \label{fig:defs:utxo}
\end{figure*}

\begin{figure}
  \begin{align*}
    & \fun{txins} \in \Tx \to \powerset{(\TxIn)}
    & \text{transaction inputs} \\
    & \txins{(inputs,~\wcard)} = \var{inputs}
    \nextdef
    & \fun{txouts} \in \Tx \to \UTxO
    & \text{transaction outputs as UTxO} \\
    & \fun{txouts} ~ \var{tx} =
      \left\{ (\fun{txid} ~ \var{tx}, \var{ix}) \mapsto \var{txout} ~
      \middle| \begin{array}{l@{~}c@{~}l}
                 (\_, \var{outputs}) & = & \txbody{tx} \\
                 \var{ix} \mapsto \var{txout} & \in & \var{outputs}
               \end{array}
      \right\}
    \nextdef
    & \fun{balance} \in \UTxO \to \Coin
    & \text{UTxO balance} \\
    & \fun{balance} ~ utxo = \sum_{(~\wcard ~ \mapsto (\wcard, ~c)) \in \var{utxo}} c
  \end{align*}

  \begin{align*}
    \var{ins} \restrictdom \var{utxo}
    & = \{ i \mapsto o \mid i \mapsto o \in \var{utxo}, ~ i \in \var{ins} \}
    & \text{domain restriction}
    \\
    \var{ins} \subtractdom \var{utxo}
    & = \{ i \mapsto o \mid i \mapsto o \in \var{utxo}, ~ i \notin \var{ins} \}
    & \text{domain exclusion}
    \\
    \var{utxo} \restrictrange \var{outs}
    & = \{ i \mapsto o \mid i \mapsto o \in \var{utxo}, ~ o \in \var{outs} \}
    & \text{range restriction}
  \end{align*}
  \caption{Functions used in UTxO rules}
  \label{fig:derived-defs:utxo}
\end{figure}

\begin{figure}
  \emph{UTxO transitions}
  \begin{equation*}
    \_ \vdash
    \var{\_} \trans{utxo}{\_} \var{\_}
    \subseteq \powerset (\PrtclConsts \times \UTxO \times \Tx \times \UTxO)
  \end{equation*}
  \caption{UTxO transition-system types}
  \label{fig:ts-types:utxo}
\end{figure}

\begin{figure}
  \begin{equation}\label{eq:utxo-inductive}
    \inference[UTxO-inductive]
    { \txins{tx} \subseteq \dom \var{utxo} & \minfee{pc}{tx} \leq \txfee{tx}\\
      \balance{(\txouts{tx})}  + \txfee{tx} =
        \balance{(\txins{tx} \restrictdom \var{utxo})}
    }
    {\var{pc} \vdash \var{utxo} \trans{utxo}{tx}
      (\txins{tx} \subtractdom \var{utxo}) \cup \txouts{tx}
    }
  \end{equation}
  \caption{UTxO inference rules}
  \label{fig:rules:utxo}
\end{figure}

Rule~\ref{eq:utxo-inductive} specifies under which conditions a transaction can
be applied to a set of unspent outputs, and how the set of unspent output
changes with a transaction:
\begin{itemize}
\item Each input spent in the transaction must be in the set of unspent
  outputs.
\item The fee paid by the transaction has to be less than or equal to the
minimum fee.
\item The balance of the unspent outputs in a transaction (i.e. the total
  amount paid in a transaction) must be equal or less than the amount of spent
  inputs.
\item If the above conditions hold, then the new state will not have the inputs
  spent in transaction $\var{tx}$ and it will have the new outputs in
  $\var{tx}$.
\end{itemize}

According to this rule, when the state of a UTxO is updated with a transaction
$\var{tx}$, and the above conditions are met, the UTxO changes as follows:

\begin{itemize}
\item remove from the UTxO all the $(\var{txin}, \var{txout})$ pairs
associated with the $\var{txins}$'s in the $\var{inputs}$ list of $\var{tx}$.
\item add all the $\var{outputs}$ of $\var{tx}$ to the
UTxO, now associated with the $\var{txid}(\var{tx})$ instead.
\end{itemize}

\begin{note}
  $\Coin$ is defined as a primitive type, but there is a difference
  between implementing it with $\mathbb{N}$ versus $\mathbb{Z}$.
  Since this is a pure UTxO ledger, $\mathbb{N}$ suffices.
  If, however, $\mathbb{Z}$ is used, then extra validation is required
  to ensure that all $\TxOut$ are non-negative.
  This extra condition would be added to \cref{eq:utxo-inductive}.
\end{note}

\subsection{Properties}
\label{sec:utxo-properties}

\begin{todo}
  Can we prove properties of the transition system of this section? For
  instance we might like to formalize ``double spending'' and prove that these
  rules prevent it. Do we want it?
\end{todo}

\subsection{Witnesses}
\label{sec:witnesses}

The definitions needed to add witnesses to the UTxO transitions described above
are presented in Figure~\ref{fig:defs:utxow}. A transaction is witnessed by
a signature and a verification key corresponding to this signature. The
$\fun{hash_{spend}}$ map associates to an address the hash of the spending key
at this address.

%\begin{note}
%  More details
%\end{note}

In Figure~\ref{fig:derived-defs:utxow}, we give the definitions of two maps
associated with a UTxO. The first, $\fun{addr}$, gives the finite map
which associates to a given $\var{txin}$ in the UTxO the address of the
corresponding $\var{txout}$. The map $\fun{addr_h}$, given a UTxO, returns the
finite map which associates $\var{txin}$ with the hash of a spending key in
the address in the corresponding $\var{txout}$ in the UTxO.

Note that the UTxO transitions with and without witnesses have the same type
(see Figure~\ref{fig:ts-types:utxo} and Figure~\ref{fig:ts-types:utxow}).
The witnessed transition rule, in fact, defines the same UTxO update as the
non-witnessed rule. It has, however, an additional precondition
(stated in Rule~\ref{eq:utxo-witness-inductive} in
Figure~\ref{fig:rules:utxow}):

\begin{itemize}
 \item For each of the inputs of the transaction $\var{tx}$ to be applied,
 there exists a verifiable witness (there could be more than one valid
 key-signature pair serving as a witness) such that the address of
\end{itemize}

%Note that
%Rule~\ref{eq:utxo-witness-inductive} uses the transition relation defined in
%Figure~\ref{fig:rules:utxo}. The main reason for doing this is to define
%the rules incrementally, modeling different aspects in isolation to keep the
%rules as simple as possible.

%Also note that the $\trans{utxo}{}$ relation could
%have been defined in terms of $\trans{utxow}{}$ (thus composing the rules in a
%different order). The choice here is arbitrary.

\begin{figure}
  \emph{Abstract functions}
  %
  \begin{equation*}
    \begin{array}{r@{~\in~}lr}
      \fun{wits} & \Tx \to \powerset{(\VKey \times \Sig)}
      & \text{witnesses of a transaction}\\
      \fun{hash_{spend}} & \Addr \mapsto \Hash
      & \text{hash of a spending key in an address}\\
    \end{array}
  \end{equation*}
  \caption{Definitions used in the UTxO transition system with witnesses}
  \label{fig:defs:utxow}
\end{figure}

\begin{figure}
  \begin{align*}
    & \addr{}{} \in \UTxO \to \TxIn \mapsto \Addr & \text{address of an input}\\
    & \addr{utxo} = \{ i \mapsto a \mid i \mapsto (a, \wcard) \in \var{utxo} \} \\
    \nextdef
    & \fun{addr_h} \in \UTxO \to \TxIn \mapsto \Hash & \text{hash of an input address}\\
    & \fun{addr_h}~utxo = \{ i \mapsto h \mid i \mapsto (a, \wcard) \in \var{utxo}
      \wedge a \mapsto h \in \fun{hash_{spend}} \}
  \end{align*}
  \caption{Functions used in rules witnesses}
  \label{fig:derived-defs:utxow}
\end{figure}

\begin{figure}
  \emph{UTxO with witness transitions}
  \begin{equation*}
    \var{\_} \vdash
    \var{\_} \trans{utxow}{\_} \var{\_}
    \subseteq \powerset (\PrtclConsts \times \UTxO \times \Tx \times \UTxO)
  \end{equation*}
  \caption{UTxO with witness transition-system types}
  \label{fig:ts-types:utxow}
\end{figure}

\begin{figure}
  \begin{equation}
    \label{eq:utxo-witness-inductive}
    \inference[UTxO-wit]
    { \var{pc} \vdash \var{utxo} \trans{utxo}{tx} \var{utxo'}\\ ~ \\
      & \forall i \in \txins{tx} \cdot \exists (\var{vk}, \sigma) \in \wits{\var{tx}}
      \cdot
      \mathcal{V}_{\var{vk}}{\serialised{\txbody{tx}}}_{\sigma}
      \wedge  \fun{addr_h}~{utxo}~i = \hash{vk}\\
    }
    {\var{pc} \vdash \var{utxo} \trans{utxow}{tx} \var{utxo'}}
  \end{equation}
  \caption{UTxO with witnesses inference rules}
  \label{fig:rules:utxow}
\end{figure}

\section{Delegation}
\label{sec:delegation}
\section{Delegation}
\label{sec:delegation}

An agent owning a key that can sign new blocks can delegate its signing rights
to another key by means of \textit{delegation certificates}. These certificates
are included in the ledger, and therefore also included in the body of the
blocks in the blockchain.

There are several restrictions on a certificate posted on the blockchain:
\begin{enumerate}
\item Only genesis keys can delegate.
\item Certificates must be properly signed by the delegator.
\item Any given key can delegate at most once per-epoch.
\item Any given key can issue at most one certificate in a given slot.
\item The epochs in the certificates must refer to the current or to the next
  epoch. We do not want to allow certificates from past epochs so that a
  delegation certificate cannot be replayed. On the other hand if we allow
  certificates with arbitrary future epochs, then a malicious key can issue a
  delegation certificate per-slot, setting the epoch to a sufficiently large
  value. This will cause a blow up in the size of the ledger state since we
  will not be able to clean $\var{eks}$ (we only clean past epochs). Also note
  that we do not check the relation between the certificate epoch and the slot
  in which the certificate becomes active. This would bring additional
  complexity without any obvious benefit.
\item Certificates do not become active immediately, but they require a certain
  number of slots till they become stable in all the nodes.
\end{enumerate}
These conditions are formalized in \cref{fig:rules:delegation-scheduling}.
Rule~\ref{eq:rule:delegation-scheduling} determines when a certificate can
become ``scheduled''. The definitions used in this rules are presented in
\cref{fig:defs:delegation-scheduling}, and the types of the system induced by
$\trans{sdeleg}{\wcard}$ are presented in
\cref{fig:ts-types:delegation-scheduling}.

\begin{figure}[htb]
  \emph{Abstract types}
  \begin{equation*}
    \begin{array}{r@{~\in~}lr}
      c & \DCert & \text{delegation certificate}\\
      \var{vk_g} & \VKeyGen & \text{genesis verification key}\\
    \end{array}
  \end{equation*}

  \emph{Derived types}
  \begin{equation*}
    \begin{array}{r@{~\in~}l@{\qquad=\qquad}r@{~\in~}lr}
      \var{e} & \Epoch & n & \mathbb{N} & \text{epoch}\\
      \var{s} & \Slot & s & \mathbb{N} & \text{slot}\\
      \var{d} & \SlotCount & s & \mathbb{N} & \text{slot}
    \end{array}
  \end{equation*}

  \emph{Constraints}
  \begin{align*}
    \VKeyGen \subseteq \VKey
  \end{align*}

  \emph{Abstract functions}
  \begin{equation*}
    \begin{array}{r@{~\in~}lr}
      \fun{dbody} & \DCert \to (\VKey \times \Epoch)
      & \text{body of the delegation certificate}\\
      \fun{dwit} & \DCert \to (\VKeyGen \times \Sig)
      & \text{witness for the delegation certificate}\\
      \fun{dwho} & \DCert \mapsto (\VKeyGen \times \VKey)
      & \text{who delegates to whom in the certificate}\\
      \fun{depoch} & \DCert \mapsto \Epoch
      & \text{certificate epoch}
    \end{array}
  \end{equation*}
  \caption{Delegation scheduling definitions}
  \label{fig:defs:delegation-scheduling}
\end{figure}

\begin{figure}[htb]
  \emph{Delegation scheduling environments}
  \begin{equation*}
    \DSEnv =
    \left(
      \begin{array}{r@{~\in~}lr}
        \mathcal{K} & \powerset{\VKeyGen} & \text{allowed delegators}\\
        \var{e} & \Epoch & \text{epoch}\\
        \var{s} & \Slot & \text{slot}\\
        \var{k} & \SlotCount & \text{chain stability parameter}
      \end{array}
    \right)
  \end{equation*}

  \emph{Delegation scheduling states}
  \begin{equation*}
    \DSState
    = \left(
      \begin{array}{r@{~\in~}lr}
        \var{sds} & \seqof{(\Slot \times (\VKeyGen \times \VKey))} & \text{scheduled delegations}\\
        \var{eks} & \powerset{(\Epoch \times \VKeyGen)} & \text{key-epoch delegations}
      \end{array}
    \right)
  \end{equation*}

  \emph{Delegation scheduling transitions}
  \begin{equation*}
    \var{\_} \vdash
    \var{\_} \trans{sdeleg}{\_} \var{\_}
    \subseteq \powerset (\DSEnv \times \DSState \times \DCert \times \DSState)
  \end{equation*}
  \caption{Delegation scheduling transition-system types}
  \label{fig:ts-types:delegation-scheduling}
\end{figure}

\begin{figure}[htb]
  \begin{equation}
    \inference
    {
    }
    {
      {\begin{array}{l}
       \mathcal{K}\\
        e\\
        s\\
        d
      \end{array}}
      \vdash
      \left(
        \begin{array}{l}
          \epsilon\\
          \emptyset
        \end{array}
      \right)
    }
  \end{equation}
  \nextdef
  \begin{equation}
    \label{eq:rule:delegation-scheduling}
    \inference
    {
      (\var{vk_s},~ \sigma) \leteq \dwit{c}
      & \verify{vk_s}{\serialised{\dbody{c}}}{\sigma} & vk_s \in \mathcal{K}\\ ~ \\
      (\var{vk_s},~ \var{vk_d}) \leteq \dwho{c} & e_d \leteq \depoch{c}
      & (e_d,~ \var{vk_s}) \notin \var{eks} & 0 \leq e_d - e \leq 1 \\ ~ \\
      d \leteq 2 \cdot k & (s + d,~ (\var{vk_s},~ \wcard)) \notin \var{sds}\\
    }
    {
      {\begin{array}{l}
       \mathcal{K}\\
        e\\
        s\\
        k
      \end{array}}
      \vdash
      {
        \left(
          \begin{array}{l}
            \var{sds}\\
            \var{eks}
          \end{array}
        \right)
      }
      \trans{sdeleg}{c}
      {
        \left(
          \begin{array}{l}
            \var{sds}; (s + d,~ (\var{vk_s},~ \var{vk_d}))\\
            \var{eks} \cup \{(e_d,~ \var{vk_s})\}
          \end{array}
        \right)
      }
    }
  \end{equation}
  \caption{Delegation scheduling rules}
  \label{fig:rules:delegation-scheduling}
\end{figure}

\clearpage

The rules in Figure~\ref{fig:rules:delegation} model the activation of
delegation certificates. Once a scheduled certificate becomes active
(see~\cref{sec:delegation-interface-rules}), the delegation map is changed by
it only if:
\begin{itemize}
\item The delegating key ($\var{vk_s}$) did not activate a delegation
  certificate in a slot greater or equal than the certificate slot ($s$). This
  check is performed to avoid having the constraint that the delegation
  certificates have to be activated in slot order.
\item The key being delegated to ($\var{vk_d}$) has not been delegated by
  another key (injectivity constraint).
\end{itemize}
The reason why we check that the delegation map is injective is to avoid a
potential risk (during the OBFT era) in which a malicious node gets control of
a genesis key $\var{vk_m}$ that issued the maximum number of blocks in a given
window. By delegating to another key $\var{vk_d}$, which was already delegated to
by some other key $\var{vk_g}$, the malicious node could prevent $\var{vk_g}$
from issuing blocks. Even though the delegation certificates take several slots
to become effective, the malicious node could calculate when the certificate
would become active, and issue a delegation certificate at the right time.

As an additional advantage, by having an injective delegation map, we are able
to simplify our specification when it comes to counting the blocks issued by
(delegates of) genesis keys.

Note also, that we could not impose the injectivity constraint in
Rule~\ref{eq:rule:delegation-scheduling} since we do not have information about
the delegations that will become effective. We could of course detect a
violation in the injectivity constraint when scheduling a delegation
certificate, but this will lead to a complex computation and larger state in
said rule.

Finally, note that we do not want to reject a scheduled delegation that would
violate the injectivity constraint (since delegation might not have been
scheduled by the node issuing the block). Instead, we simply ignore the
delegation certificate (Rule~\ref{eq:rule:delegation-nop}).

\begin{figure}[htb]
  \begin{align*}
    & \unionoverrideRight \in (A \mapsto B) \to (A \mapsto B) \to (A \mapsto B)
    & \text{union override}\\
    & d_0 \unionoverrideRight d_1 = d_1 \cup (\dom d_1 \subtractdom d_0)
  \end{align*}
  \caption{Functions used in delegation rules}
  \label{fig:funcs:delegation}
\end{figure}

\begin{figure}[htb]
  \emph{Delegation environments}
  \begin{equation*}
    \DEnv =
    \left(
      \begin{array}{r@{~\in~}lr}
        \mathcal{K} & \powerset{\VKeyGen} & \text{allowed delegators}
      \end{array}
    \right)
  \end{equation*}

  \emph{Delegation states}
  \begin{align*}
    & \DState
      = \left(
        \begin{array}{r@{~\in~}lr}
          \var{dms} & \VKeyGen \mapsto \VKey & \text{delegation map}\\
          \var{dws} & \VKeyGen \mapsto \Slot & \text{when last delegation occurred}\\
        \end{array}\right)
  \end{align*}
  \emph{Delegation transitions}
  \begin{equation*}
    \_ \vdash \_ \trans{adeleg}{\_} \_ \in
    \powerset (\DEnv \times \DState \times (\Slot \times (\VKeyGen \times \VKey)) \times \DState)
    \end{equation*}
  \caption{Delegation transition-system types}
  \label{fig:ts-types:delegation}
\end{figure}

\begin{figure}[htb]
  \begin{equation}
    \inference
    {
      \var{dms_0} \leteq \Set{k \mapsto k}{k \in \mathcal{K}} &
      \var{dws_0} \leteq \Set{k \mapsto 0}{k \in \mathcal{K}}
    }
    {
      \mathcal{K}
      \vdash
      \left(
        \begin{array}{l}
          \var{dms_0}\\
          \var{dws_0}
        \end{array}
      \right)
    }
  \end{equation}
  \nextdef
  \begin{equation}\label{eq:rule:delegation-change}
    \inference
    {
      \var{vk_d} \notin \range~\var{dms} & (\var{vk_s} \mapsto s_p \in \var{dws} \Rightarrow s_p < s)
    }
    {
      \mathcal{K}
      \vdash
      \left(
      \begin{array}{r}
        \var{dms}\\
        \var{dws}
      \end{array}
      \right)
      \trans{adeleg}{(s,~ (vk_s,~ vk_d))}
      \left(
      \begin{array}{lcl}
        \var{dms} & \unionoverrideRight & \{\var{vk_s} \mapsto \var{vk_d}\}\\
        \var{dws} & \unionoverrideRight & \{\var{vk_s} \mapsto s \}
      \end{array}
      \right)
    }
  \end{equation}
  \nextdef
  \begin{equation}\label{eq:rule:delegation-nop}
    \inference
    {\var{vk_d} \in \range~\var{dms} \vee (\var{vk_s} \mapsto s_p  \in \var{dws}  \wedge s \leq s_p)
    }
    {
      \mathcal{K}
      \vdash
      \left(
      \begin{array}{r}
        \var{dms}\\
        \var{dws}
      \end{array}
      \right)
      \trans{adeleg}{(s,~ (\var{vk_s},~ \var{vk_d}))}
      \left(
      \begin{array}{lcl}
        \var{dms}\\
        \var{dws}
      \end{array}
      \right)
    }
  \end{equation}
  \caption{Delegation inference rules}
  \label{fig:rules:delegation}
\end{figure}

\clearpage

\subsection{Delegation sequences}
\label{sec:delegation-sequences}

This section presents the rules that model the effect that sequences of
delegations have on the ledger.

\begin{figure}[htb]
  \begin{equation}
    \inference[Initial-SDELEGS]
    {
    }
    {
      {\begin{array}{l}
       \mathcal{K}\\
        e\\
        s\\
        d
      \end{array}}
      \vdash
      \left(
        \begin{array}{l}
          \epsilon\\
          \emptyset
        \end{array}
      \right)
    }
  \end{equation}
  \nextdef
  \begin{equation}
    \label{eq:rule:delegation-scheduling-seq-base}
    \inference
    {
    }
    {
      {\begin{array}{l}
         \mathcal{K} \\
         e\\
         s\\
         d
       \end{array}}
      \vdash
      {
        \left(
          \begin{array}{l}
            \var{sds}\\
            \var{eks}
          \end{array}
        \right)
      }
      \trans{sdelegs}{\epsilon}
      {
        \left(
          \begin{array}{l}
            \var{sds}\\
            \var{eks}
          \end{array}
        \right)
      }
    }
  \end{equation}
  \nextdef
  \begin{equation}
    \label{eq:rule:delegation-scheduling-seq-ind}
    \inference
    {
      {\begin{array}{l}
         \mathcal{K} \\
         e\\
         s\\
         d
       \end{array}}
      \vdash
      {
        \left(
          \begin{array}{l}
            \var{sds}\\
            \var{eks}
          \end{array}
        \right)
      }
      \trans{sdelegs}{\Gamma}
      {
        \left(
          \begin{array}{l}
            \var{sds'}\\
            \var{eks'}
          \end{array}
        \right)
      }
      &
      {\begin{array}{l}
         \mathcal{K} \\
         e\\
         s\\
         d
       \end{array}}
      \vdash
      {
        \left(
          \begin{array}{l}
            \var{sds'}\\
            \var{eks'}
          \end{array}
        \right)
      }
      \trans{sdeleg}{c}
      {
        \left(
          \begin{array}{l}
            \var{sds''}\\
            \var{eks''}
          \end{array}
        \right)
      }
    }
    {
      {\begin{array}{l}
         \mathcal{K} \\
         e\\
         s\\
         d
       \end{array}}
      \vdash
      {
        \left(
          \begin{array}{l}
            \var{sds}\\
            \var{eks}
          \end{array}
        \right)
      }
      \trans{sdelegs}{\Gamma; c}
      {
        \left(
          \begin{array}{l}
            \var{sds''}\\
            \var{eks''}
          \end{array}
        \right)
      }
    }
  \end{equation}
  \caption{Delegation scheduling sequence rules}
  \label{fig:rules:delegation-scheduling-seq}
\end{figure}

\begin{figure}
  \begin{equation}
    \inference[Initial-ADELEGS]
    {
      \var{dms_0} \leteq \Set{k \mapsto k}{k \in \mathcal{K}} &
      \var{dws_0} \leteq \Set{k \mapsto 0}{k \in \mathcal{K}}
    }
    {
      \mathcal{K}
      \vdash
      \left(
        \begin{array}{l}
          \var{dms_0}\\
          \var{dws_0}
        \end{array}
      \right)
    }
  \end{equation}
  \nextdef
  \begin{equation}
    \label{eq:rule:delegation-seq-base}
    \inference
    {
    }
    {
      \mathcal{K}
      \vdash
      {
        \left(
          \begin{array}{l}
            \var{dms}\\
            \var{dws}
          \end{array}
        \right)
      }
      \trans{adelegs}{\epsilon}
      {
        \left(
          \begin{array}{l}
            \var{dms}\\
            \var{dws}
          \end{array}
        \right)
      }
    }
  \end{equation}
  \nextdef
  \begin{equation}
    \label{eq:rule:delegation-seq-ind}
    \inference
    {
      {
        \left(
          \begin{array}{l}
            \var{dms}\\
            \var{dws}
          \end{array}
        \right)
      }
      \trans{adelegs}{\Gamma}
      {
        \left(
          \begin{array}{l}
            \var{dms'}\\
            \var{dws'}
          \end{array}
        \right)
      }
      &
      {
        \left(
          \begin{array}{l}
            \var{dms'}\\
            \var{dws'}
          \end{array}
        \right)
      }
      \trans{adeleg}{c}
      {
        \left(
          \begin{array}{l}
            \var{dms''}\\
            \var{dws''}
          \end{array}
        \right)
      }
    }
    {
      \mathcal{K}
      \vdash
      {
        \left(
          \begin{array}{l}
            \var{dms}\\
            \var{dws}
          \end{array}
        \right)
      }
      \trans{adelegs}{\Gamma; c}
      {
        \left(
          \begin{array}{l}
            \var{dms''}\\
            \var{dws''}
          \end{array}
        \right)
      }
    }
  \end{equation}
  \caption{Delegations sequence rules }
  \label{fig:rules:delegation-seq}
\end{figure}


\addcontentsline{toc}{section}{References}
\bibliographystyle{plainnat}
\bibliography{references}

\end{document}
