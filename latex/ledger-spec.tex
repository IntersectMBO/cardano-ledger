\documentclass[11pt,a4paper]{article}
\usepackage[margin=2.5cm]{geometry}
\usepackage{iohk}
\usepackage{microtype}
\usepackage{mathpazo} % nice fonts
\usepackage{amsmath}
\usepackage{amssymb}
\usepackage{amsthm}
\usepackage{latexsym}
\usepackage{mathtools}
\usepackage{stmaryrd}
\usepackage{extarrows}
\usepackage{slashed}
\usepackage[colon]{natbib}
\usepackage[unicode=true,pdftex,pdfa]{hyperref}
\usepackage{xcolor}
\usepackage[capitalise,noabbrev,nameinlink]{cleveref}
\usepackage{float}
\floatstyle{boxed}
\restylefloat{figure}

%%
%% Package `semantic` can be used for writing inference rules.
%%
\usepackage{semantic}
%% Setup for the semantic package
\setpremisesspace{20pt}

%%
%% Types
%%
\newcommand{\Tx}{\type{Tx}}
\newcommand{\Ix}{\type{Ix}}
\newcommand{\TxId}{\type{TxId}}
\newcommand{\Addr}{\type{Addr}}
\newcommand{\UTxO}{\type{UTxO}}
\newcommand{\Value}{\type{Value}}
\newcommand{\Coin}{\type{Coin}}
\newcommand{\PrtclConsts}{\type{PrtclConsts}}
\newcommand{\Slot}{\type{Slot}}
\newcommand{\Duration}{\type{Duration}}
\newcommand{\Allocs}{\type{Allocs}}

\newcommand{\DCert}{\type{DCert}}
\newcommand{\DCertRegKey}{\type{DCert_{regkey}}}
\newcommand{\DCertDeRegKey}{\type{DCert_{deregkey}}}
\newcommand{\DCertDeleg}{\type{DCert_{delegate}}}
\newcommand{\DCertRegPool}{\type{DCert_{regpool}}}
\newcommand{\DCertRetirePool}{\type{DCert_{retirepool}}}
\newcommand{\ledgerState}{\ensuremath{\type{ledgerState}}}

\newcommand{\AddrRWD}{\type{Addr_{rwd}}}
\newcommand{\DState}{\type{DState}}
\newcommand{\DWState}{\type{DWState}}
\newcommand{\DWEnv}{\type{DWEnv}}
\newcommand{\PState}{\type{PState}}
\newcommand{\DCertBody}{\type{DCertBody}}
\newcommand{\DPoolReap}{\ensuremath{\type{poolreap}}}

%% Adding witnesses
\newcommand{\TxIn}{\type{TxIn}}
\newcommand{\TxOut}{\type{TxOut}}
\newcommand{\VKey}{\type{VKey}}
\newcommand{\SKey}{\type{SKey}}
\newcommand{\Hash}{\type{Hash}}
\newcommand{\SkVk}{\type{SkVk}}
\newcommand{\Sig}{\type{Sig}}
\newcommand{\Data}{\type{Data}}
%% Adding delegation
\newcommand{\Epoch}{\type{Epoch}}
\newcommand{\VKeyGen}{\type{VKeyGen}}
%% Blockchain
\newcommand{\Gkeys}{\var{G_{keys}}}
\newcommand{\Block}{\type{Block}}
\newcommand{\SlotId}{\type{SlotId}}
\newcommand{\UTxOEnv}{\type{UTxOEnv}}
\newcommand{\CEEnv}{\type{CEEnv}}
\newcommand{\CEState}{\type{CEState}}
\newcommand{\BDEnv}{\type{BDEnv}}
\newcommand{\BDState}{\type{BDState}}
\newcommand{\LEnv}{\type{LEnv}}
\newcommand{\LState}{\type{LState}}

%%
%% Functions
%%
\newcommand{\txins}[1]{\fun{txins}~ \var{#1}}
\newcommand{\txid}[1]{\fun{txid}~ \var{#1}}
\newcommand{\txouts}[1]{\fun{txouts}~ \var{#1}}
\newcommand{\values}[1]{\fun{values}~ #1}
\newcommand{\balance}[1]{\fun{balance}~ \var{#1}}
\newcommand{\ttl}[1]{\fun{ttl}~ \var{#1}}
\newcommand{\deposits}[2]{\fun{deposits}~ \var{#1} ~ \var{#2}}
\newcommand{\refund}[4]{\fun{refund}~ \var{#1}~ \var{#2}~ \var{#3}~ \var{#4}}
\newcommand{\refunds}[4]{\fun{refunds}~ \var{#1}~ \var{#2}~ \var{#3}~ \var{#4}}
\newcommand{\created}[5]{\fun{created}~ \var{#1}~ \var{#2}~ \var{#3}~ \var{#4}~ \var{#5}}
\newcommand{\destroyed}[2]{\fun{destroyed}~ \var{#1}~ \var{#2}}
\newcommand{\applyFun}[2]{\fun{#1}~\var{#2}}

\newcommand{\RegKey}[1]{\textsc{RegKey}(#1)}
\newcommand{\DeregKey}[1]{\textsc{DeregKey}(#1)}
\newcommand{\Delegate}[1]{\textsc{Delegate}(#1)}
\newcommand{\RegPool}[1]{\textsc{RegPool}(#1)}
\newcommand{\RetirePool}[1]{\textsc{RetirePool}(#1)}
\newcommand{\cauthor}[1]{\fun{author}~ \var{#1}}
\newcommand{\pool}[1]{\fun{pool}~ \var{#1}}
\newcommand{\retire}[1]{\fun{retire}~ \var{#1}}
\newcommand{\addrRw}[1]{\fun{addr_{rwd}}~ \var{#1}}
\newcommand{\epoch}[1]{\fun{epoch}~ \var{#1}}
\newcommand{\dcerts}[1]{\fun{dcerts}~ \var{#1}}

%% UTxO witnesses
\newcommand{\inputs}[1]{\fun{inputs}~ \var{#1}}
\newcommand{\wits}[1]{\fun{wits}~ \var{#1}}
\newcommand{\verify}[3]{\fun{verify} ~ #1 ~ #2 ~ #3}
\newcommand{\sign}[2]{\fun{sign} ~ #1 ~ #2}
\newcommand{\serialised}[1]{\llbracket \var{#1} \rrbracket}
\newcommand{\addr}[1]{\fun{addr}~ \var{#1}}
\newcommand{\hash}[1]{\fun{hash}~ \var{#1}}
\newcommand{\txbody}[1]{\fun{txbody}~ \var{#1}}
\newcommand{\txfee}[1]{\fun{txfee}~ \var{#1}}
\newcommand{\minfee}[2]{\fun{minfee}~ \var{#1}~ \var{#2}}
\newcommand{\slotminus}[2]{\var{#1}~-_{s}~\var{#2}}
\DeclarePairedDelimiter\floor{\lfloor}{\rfloor}
% wildcard parameter
\newcommand{\wcard}[0]{\underline{\phantom{a}}}
%% Adding ledgers...
\newcommand{\utxo}[1]{\fun{utxo}~ #1}
%% Delegation
\newcommand{\delegatesName}{\fun{delegates}}
\newcommand{\delegates}[3]{\delegatesName~#1~#2~#3}
\newcommand{\dwho}[1]{\fun{dwho}~\var{#1}}
\newcommand{\depoch}[1]{\fun{depoch}~\var{#1}}
%% Delegation witnesses
\newcommand{\dbody}[1]{\fun{dbody}~\var{#1}}
\newcommand{\dwit}[1]{\fun{dwit}~\var{#1}}
%% Blockchain
\newcommand{\bwit}[1]{\fun{bwit}~\var{#1}}
\newcommand{\bslot}[1]{\fun{bslot}~\var{#1}}
\newcommand{\bbody}[1]{\fun{bbody}~\var{#1}}
\newcommand{\bdlgs}[1]{\fun{bdlgs}~\var{#1}}
%% ledgerstate constants
\newcommand{\genesisId}{\ensuremath{Genesis_{Id}}}
\newcommand{\genesisTxOut}{\ensuremath{Genesis_{Out}}}
\newcommand{\genesisUTxO}{\ensuremath{Genesis_{UTxO}}}
\newcommand{\emax}{\mathsf{E_{max}}}

\theoremstyle{definition}
\newtheorem{definition}{Definition}[section]

\theoremstyle{definition}
\newtheorem{property}{Property}[section]

\begin{document}

\hypersetup{
  pdftitle={A Formal Specification of the Cardano Ledger},
  breaklinks=true,
  bookmarks=true,
  colorlinks=false,
  linkcolor={blue},
  citecolor={blue},
  urlcolor={blue},
  linkbordercolor={white},
  citebordercolor={white},
  urlbordercolor={white}
}


  \cleardoublepage%
  \tableofcontents%
  \listoffigures%
  \clearpage%

  \begin{changelog}
        \change{2019/10/08}{Jared Corduan, Polina Vinogradova and Matthias G\"udemann}{FM (IOHK)}{Initial version (0.1).}
        \change{2019/10/08}{Kevin Hammond}{FM (IOHK)}{Added cover page.}
      \end{changelog}
      \clearpage%
\begin{landscape}
\floatstyle{plain}
\restylefloat{figure}
\begin{figure*}
\includegraphics[scale=0.65]{d8-depends.pdf}
\caption{Positioning of this Deliverable.}
\end{figure*}
\end{landscape}
\floatstyle{boxed}
\restylefloat{figure}
\cleardoublepage
% \begin{center}
%   \large{Executive Summary}
% \end{center}

% \noindent
% This document provides a formal specification of the Cardano ledger for use in the Shelley implementation.
% It is intended to form the basis for an executable specification that will be the basis of the initial
% release.
% \cleardoublepage
\renewcommand{\thepage}{\arabic{page}}
\setcounter{page}{1}

\title{A Formal Specification of the Cardano Ledger}

\author{Jared Corduan  \\ {\small \texttt{jared.corduan@iohk.io}} \\
   \and Polina Vinogradova \\ {\small \texttt{polina.vinogradova@iohk.io}} \\
   \and Matthias G\"udemann  \\ {\small \texttt{matthias.gudemann@iohk.io}}}

%\date{}

\maketitle

\begin{abstract}
This document provides a formal specification of the Cardano ledger for use in the upcoming Shelley implementation.
It is intended to underpin a Haskell executable specification that will be the basis of the initial
Shelley release, and represents a core design and quality assurance document.
It will be used to define properties and tests, and to provide the basis for strong formal assurance
using mathematical proof techniques.
The document defines the rules for extending the ledger with transactions
that will affect both UTxO and stake delegation.
Key properties that have been identified include ...
% This will serve as the specification for random generators of transactions
% which adhere to the rules presented here, and that will.
\end{abstract}

\section*{List of Contributors}
\label{acknowledgements}

Nicol\'as Arqueros,
Nicholas Clarke,
Duncan Coutts,
Ruslan Dudin,
Sebastien Guillemot,
Vincent Hanquez,
Ru Horlick,
Michael Hueschen,
Yun Lu,
Philipp Kant,
Jean-Christophe Mincke,
Damian Nadales,
Nicolas Di Prima.


\tableofcontents
\listoffigures

\section{Introduction}
\label{sec:introduction}
\section{Introduction}
\label{sec:introduction-shelley}

This document is a formal specification of the functionality of the ledger
on the blockchain. The blockchain layer of the
protocol and the interaction between the ledger and the blockchain
layer is presented in a separate document, see~\cite{shelley_consensus}. The details of the
background and the larger context
for the design decisions formalized in this document are presented
in~\cite{delegation_design}

In this work,
we present important properties any implementation of the ledger must have.
Specifically, we model the following aspects
of the functionality of the ledger on the blockchain:

\begin{description}
\item[Preservation of value] Every coin in the system is accounted for,
  and the total amount is unchanged by every transaction and epoch change.
  In particular, every coin is accounted for by one of the following categories:
  \begin{itemize}
    \item Circulation (UTxO)
    \item Deposit pot
    \item Fee pot
    \item Reserves (monetary expansion)
    \item Rewards (account addresses)
    \item Reward pot (undistributed)
    \item Treasury
  \end{itemize}
\item[Witnesses] Authentication of parts of the transaction data by means of
  cryptographic entities (such as signatures and private keys) contained in
  these transactions.
\item[Delegation] Validity of delegation certificates, which delegate
  block-signing rights.
\item[Stake] Staking rights associated to an address.
\end{description}

While the blockchain protocol is a reactive system driven by the arrival
of blocks causing updates to the ledger, the formal description is a collection
of rules which is a
static description of what a \textit{valid ledger} is. The specifics of the
semantics we use to define and apply
the rules we present in this document are explained in detail in
\cite{small_step_semantics}. A valid ledger state can only
reached by applying a sequence of inference rules, and any valid ledger state
is reachable by applying some sequence of these rules.

The structure of the rules we give here is such that their application is
deterministic. That is, given a specific initial state and relevant environmental
constants, there is no ambiguity
about which rule should be applied at any given time (i.e.~which state
transition is allowed take place). This is an important property which reflects
the reality of the implementation --- the blockchain evolves in a particular way
given some user activity and the passage of time, and its behaviour is
never unexpected.


\section{Notation}\label{sec:notation}

The transition system is explained in \cite{small_step_semantics}.

\begin{description}
\item[Powerset] Given a set $\type{X}$, $\powerset{\type{X}}$ is the set of all
  the subsets of $X$.
\item[Sequences] Given a set $\type{X}$, $\seqof{\type{X}}$ is the set of
  sequences having elements taken from $\type{X}$. The empty sequence is
  denoted by $\epsilon$, and given a sequence $\Lambda$, $\Lambda; \type{x}$ is
  the sequence that results from appending $\type{x} \in \type{X}$ to
  $\Lambda$.
\item[Functions] $A \to B$ denotes a \textbf{total function} from $A$ to $B$.
  Given a function $f$ we write $f~a$ for the application of $f$ to argument
  $a$.
\item[Fibre] Given a function $f: A \to B$ and $b\in B$, we write
  $f^{-1}~b$ for the \textbf{fibre} of $f$ at $b$, which is defined by
  $\{a \mid\ f a =  b\}$.
\item[Maps and partial functions] $A \mapsto B$ denotes a \textbf{partial
    function} from $A$ to $B$, which can be seen as a map (dictionary) with
  keys in $A$ and values in $B$. Given a map $m \in A \mapsto B$, notation
  $a \mapsto b \in m$ is equivalent to $m~ a = b$.
\end{description}

\section{Cryptographic primitives}
\label{sec:crypto-primitives}

Figure~\ref{fig:crypto-defs} introduces the cryptographic abstractions used in
this document.

\begin{figure}
  \emph{Abstract types}
  %
  \begin{equation*}
    \begin{array}{r@{~\in~}lr}
      \var{vk} & \SKey & \text{private signing key}\\
      \var{vk} & \VKey & \text{public verifying key}\\
      \var{hk} & \Hash & \text{hash of a key}\\
      \sigma & \Sig  & \text{signature}\\
      \var{d} & \Data  & \text{data}\\
    \end{array}
  \end{equation*}
  \emph{Derived types}
  \begin{equation*}
    \begin{array}{r@{~\in~}lr}
      (sk, vk) & \SkVk & \text{signing-verifying key pairs}
    \end{array}
  \end{equation*}
  \emph{Abstract functions}
  %
  \begin{equation*}
    \begin{array}{r@{~\in~}lr}
      \hash{} & \VKey \to \Hash
      & \text{hash function} \\
      %
      \fun{verify} & \powerset{\left(\VKey \times \Data \times \Sig\right)}
      & \text{verification relation}\\
    \end{array}
  \end{equation*}
  \emph{Constraints}
  \begin{align*}
    & \forall (sk, vk) \in \SkVk,~ m \in \Data,~ \sigma \in \Sig \cdot
      \verify{vk}{m}{\sigma} \iff \sign{sk}{m} = \sigma
  \end{align*}
  \emph{Notation for serialized and verified data}
  \begin{align*}
    & \serialised{x} & \text{serialised representation of } x\\
    & \mathcal{V}_{\var{vk}}{\serialised{m}}_{\sigma} = \verify{vk}{m}{\sigma}
      & \text{shorthand notation for } \fun{verify}
  \end{align*}
  \caption{Cryptographic definitions}
  \label{fig:crypto-defs}
\end{figure}

\section{Serialization}
\label{sec:serialization}

\begin{todo}
  Discuss here serialization and
  \href{https://iohk.myjetbrains.com/youtrack/issue/CDEC-628}{composable
    serialization}
\end{todo}

\section{Delegation}
\label{sec:delegation}

We briefly describe the motivation and context for delegation.
The full context is contained in \cite{delegation_design}.

Stake is said to be \textit{active} in the blockchain protocol
when it is eligible for participation in the leader election. In order for
stake to become active,
the associated verification stake key must be registered
and its staking rights must be delegated to an active stake pool.
Individuals who wish to participate in the protocol can
register themselves as a stake pool.

Stake keys are registered (deregistered) through the use of
registration (deregistration) certificates.
Registered stake keys are delegated through the use of delegation certificates.
Finally, stake pools are registered (retired) through the use of
registration (retirement) certificates.

Stake pool retirement is handled a bit differently than stake key deregistration.
Stake keys are considered inactive as soon as a deregistration certificate
is applied to the ledger state.
Stake pool retirement certificates, however, specify the epoch in
which it will retire.

Delegation requires the following to be tracked by the ledger state:
the registered stake keys, the delegation map from registered stake keys to stake
pools, the registered stake pools, and upcoming stake pool retirements.
Additionally, the blockchain protocol rewards eligible stake, and so we must
also include a mapping from active stake keys to rewards.



\begin{note}
  The current rules allow for delegation to a non-existent pool.
  Such stake is not eligible for leader election.
  This allows for a clean separation between the rules in
  \cref{fig:delegation-rules} and \cref{fig:pool-rules}.
  We may, however, later choose to enforce that delegation certificates
  target a registered pool. It would then make sense to remove
  mappings in $\var{delegations}$ when stake pools retire.
\end{note}


\subsection{Delegation Definitions}
\label{sec:deleg-defs}

In \cref{fig:delegation-definitons} we give the delegation primitives.
The additional relevant primitive datatypes for delegation are addresses,
epochs, slot numbers, and duration (the difference between two slot numbers).
The constant $\emax$ gives the number of epochs a stake pool will take to retire.

The type $\DCert$ is a generic certificate type, which can be a registration,
deregistration or delegation certificate for a key, or a registration/retirement
 certificate for a stake pool. It is denoted as disjoint union in the figure,
one should, however, think of a term of this type as a term of a specific
one of these five subtypes.

The terms $\RegKey$, $\DeregKey$, $\Delegate$, $\RegPool$, $\RetirePool$ are
all predicates on the type $\DCert$. Each of these predicates is true
when applied to a certificate $c \in \DCert$ of the corresponding type.
Using these predicates in defining inference rules ensures $c$ is the correct
type of certificate for the transition
that a rule describes.

Note that the reason for combining the different types of
certificates into a common type is that it allows us to use that type to later
define a single type for all
ledger state transitions having to do with delegation (i.e. $\DState$
transitions),
and, in a very similarly
way, a type for transitions describing stake pool-related ledger updates
(i.e. $\PState$ transitions).

The maps $\fun{hash}$, $\fun{addr_{rwd}}$, $\fun{author}$, $\fun{pool}$,
$\fun{retire}$, and $\fun{epoch}$ are all used to
retrieve specific information about the origin type. The subtraction $(-)$
of slots gives the number of slots in between (referred to here as
$\Duration$ here by $\var{dur}$).



%%
%% Figure - Delegation Definitions
%%
\begin{figure}
  \emph{Abstract types}
  %
  \begin{equation*}
    \begin{array}{r@{~\in~}lr}
      a & \AddrRWD & \text{reward address} \\
      slot & \Slot & \text{slot}\\
      dur & \Duration & \text{duration}\\
      epoch & \Epoch & \text{epoch} \\
      stakepool & \StakePool & \text{stake pool constants}
    \end{array}
  \end{equation*}
  %
  \emph{Constants}
  \begin{equation*}
    \begin{array}{r@{~\in~}lr}
      \emax & \Epoch & \text{epoch bound on pool retirement}
    \end{array}
  \end{equation*}
  %
  \emph{Delegation Certificate types}
  %
  \begin{equation*}
  \begin{array}{r@{}c@{}l}
    \DCert &=& \DCertRegKey \uniondistinct \DCertDeRegKey \uniondistinct \DCertDeleg \\
                &\hfill\uniondistinct\;& \DCertRegPool \uniondistinct \DCertRetirePool \\
    \RegKey{c} \in \DCert &\iff& c \in \DCertRegKey \\
    \DeregKey{c} \in \DCert &\iff& c \in \DCertDeRegKey \\
    \Delegate{c} \in \DCert &\iff& c \in \DCertDeleg \\
    \RegPool{c} \in \DCert &\iff& c \in \DCertRegPool\\
    \RetirePool{c} \in \DCert &\iff& c \in \DCertRetirePool \\
  \end{array}
  \end{equation*}
  %
  \emph{Derived types}
  \begin{equation*}
    \begin{array}{r@{~\in~}l@{\qquad=\qquad}r@{~\in~}lr}
      \var{allocs}
      & \Allocs
      & hkeys \mapsto slot
      & \HashKey \mapsto \Slot
      & \text{resource allocations}
    \end{array}
  \end{equation*}
  %
  \emph{Abstract functions}
  %
  \begin{equation*}
  \begin{array}{r@{~\in~}lr}
  \fun{hashKey} & \VKey \to \HashKey
  & \text{hashKeying a key}
  \\
  \fun{addr_{rwd}} & \HashKey \to \AddrRWD
  & \text{address of a hashKeykey}
  \\
  \fun{author} & \DCert \to \HashKey
  & \text{certificate author}
  \\
  \fun{dpool} & \DCertDeleg \to \HashKey
  & \text{pool being delegated to}
  \\
  \fun{stakepool} & \DCertRegPool \to \StakePool
  & \text{stake pool constants}
  \\
  \fun{retire} & \DCertRetirePool \to \Epoch
  & \text{epoch of pool retirement}
  \\
  \fun{epoch} & \Slot \to \Epoch
  & \text{epoch of a slot}
  \\
    (\slotminus{}{}) & \Slot \to \Slot \to \Duration
  & \text{slot duration}
  \end{array}
  \end{equation*}
  %

  \caption{Delegation Definitions}
  \label{fig:delegation-definitons}
\end{figure}


\subsection{Delegation Transitions}
\label{sec:deleg-trans}


In \cref{fig:delegation-transitions} we give the delegation and stake pool
state transition types. We define two separate parts of the ledger state.

The part of the ledger state keeping track of current delegations, $\DState$,
consists of three variables. The first, $\var{stkeys}$, keeps track of stake
keys. It consists of the hash keys of registered stake keys paired with
slot number in which they are registered.
Note that individual key registration
and deregistration certificates contain no important data beyond a declaration
of registration, and thus are not stored as part of the state variables once
they have been processed.

The second, $\var{rewards}$, stores the
rewards accumulated by stakey keys. These are represented by
a finite
map that matches addresses with the rewards belonging to them. The third
variable in $\DState$ stores currently registered delegations.
Delegations ($\var{delegations}$) are also expressed by a finite map, which
associates a stake key with the hash key of the pool to which it delegates.

The ledger additionally keeps track of the stake pools in a separate state variable
$\PState$.
This state contains a list of registered stake pools, $\var{stpools}$.
These are pairs of registration certificates and slot numbers in which the
stake pool was registered, indexed by the hash key
of the author to whom they are registered. The stake pool certificates
do contain useful data that must be stored, and hence are included in the
ledger state.

It also keeps track of
stake pools scheduled to retire via the variable $\var{retiring}$,
which associates
a stake pool hash key with the epoch in which it is supposed to retire.

\begin{todo}
  Is it better to store this variable in the blockchain layer, and
  clean up $\var{retiring}$ the same way outdated scheduled delegations are
  cleaned up?

  Discuss POOLREAP and pparams!!!!!
\end{todo}

The environment for state transitions for both $\DState$ and $\PState$ contains
only the current slot number. The slot number is not strictly necessary for the
$\DState$ transitions, however, it makes the transition environments consistent
with each other for a more straighforward way to combine the transition
rules of $\DState$ and $\PState$ into one. Both transitions are triggered by a
certificate (contained in a signal transaction).


%%
%% Figure - Delegation Transitions
%%
\begin{figure}
  \emph{Delegation States}
  %
  \begin{equation*}
    \begin{array}{l}
    \DState =
    \left(\begin{array}{r@{~\in~}lr}
      \var{stkeys} & \Allocs & \text{registered stake keys to creation time}\\
      \var{rewards} & \AddrRWD \mapsto \Coin & \text{rewards}\\
      \var{delegations} & \HashKey \mapsto \HashKey & \text{delegations}\\
    \end{array}\right)
    \\
    \\
    \PState =
    \left(\begin{array}{r@{~\in~}lr}
      \var{stpools} & \Allocs & \text{registered pools to creation time}\\
      \var{pparams} & \HashKey \mapsto \StakePool
        & \text{registered pools to pool parameters}\\
      \var{retiring} & \HashKey \mapsto \Epoch & \text{retiring stake pools}\\
    \end{array}\right)
    \end{array}
  \end{equation*}
  %
  \emph{Delegation Transitions}
  \begin{equation*}
    \_ \vdash \_ \trans{deleg}{\_} \_ \in
      \powerset (\Slot \times \DState \times \DCert \times \DState)
  \end{equation*}
  %
  \begin{equation*}
    \_ \vdash \_ \trans{pool}{\_} \_ \in
      \powerset (\Slot \times \PState \times \DCert \times \PState)
  \end{equation*}
  %
  \begin{equation*}
    \_ \vdash \_ \trans{poolreap}{} \_ \in
    \powerset (\Slot \times \PState \times \PState)
  \end{equation*}
  %
  \caption{Delegation Transitions}
  \label{fig:delegation-transitions}
\end{figure}


\subsection{Delegation Rules}
\label{sec:deleg-rules}


The rules for registering and delegating stake keys are given in
\cref{fig:delegation-rules}. The preconditions for the registration of a stake
key ensure that a given certificate $c$ is of the correct type
(i.e. $\DCertRegKey$),
and that the hash key associated with the author of the certificate is not
already found in the current list of stake keys.

In order to delegate to a stake pool, a stake key must first be registered.
When registering a new stake key (the \cref{eq:deleg-reg} inference rule),
the key must be
added to the set of stake keys ($\var{stkeys}$), the rewards for the address
corresponding to that key set to 0, and no new delegations added. Note that
since the hash key corresponding to the address must not have been previously
registered,
it should not have any rewards in its associated address. Thus, it is safe
to set the rewards to 0 using union override (i.e. replace any value previously
associated with this address with 0).

\begin{todo}
Here we would add the rules for managing rewards beyond assigning
the initial 0 value upon registration. The formula that calculates rewards
takes into account stake pool
performance (ranked in the order of expected rewards).

Discuss POOLREAP and pparams
\end{todo}

When deregistering a key (the \cref{eq:deleg-dereg} rule), we again
require that the certificate is of the correct type. We also require
that the key of the author is indeed a registered stake key
in order to be able to retrieve its address for the rewards update.

As a result of this rule, the author's key must be removed from the $\var{stkeys}$ list,
and all the rewards and delegations associated with this key must be removed
from the $\var{rewards}$ and $\var{delegations}$ parameters also.

Finally, for creating a delegation (the \cref{eq:deleg-deleg} rule),
given that the certificate $c$ is of the correct type, we add to the
$\var{delegations}$ finite map the pair of
the author's hash key and hash key of the pool being delegated to.
Again, we require the author's key be a registered key, as it does not make
sense to allow delegation otherwise.
The $\var{stkeys}$ and $\var{rewards}$ parameters are kept constant by
this rule. The use of union override here allows us to use the same rule
to perform an update on an existing delegation while keeping the rewards
associated with the key accounted for.

We would like to point out, here, that this specification does not describe
how the wallet makes a decision about which stake pool a stake key will
delegate to. This decision, however, is influenced by some parameters in the
protocol. These parameters determine which stake pools are more profitable
to delegate to, as well as the optimal number of stake pools in the system,
by means of regulating reward distribution.
This avoids forming a monopoly of a a single large stake pool constantly
being delegated to.


%%
%% Figure - Delegation Rules
%%
\begin{figure}
  \centering
  \begin{equation}\label{eq:deleg-reg}
    \inference[Deleg-Reg]
    {
      \RegKey{c} & \cauthor{c} = hk & hk \notin \var{stkeys}
    }
    {
      \var{slot} \vdash
      \left(
      \begin{array}{r}
        \var{stkeys} \\
        \var{rewards} \\
        \var{delegations}
      \end{array}
      \right)
      \trans{deleg}{\var{c}}
      \left(
      \begin{array}{rcl}
        \var{stkeys} & \union & \{\var{hk} \mapsto slot\}\\
        \var{rewards} & \unionoverrideRight & \{\addrRw \var{hk} \mapsto 0\}\\
        \var{delegations}
      \end{array}
      \right)
    }
  \end{equation}

  \begin{equation}\label{eq:deleg-dereg}
    \inference[Deleg-Dereg]
    {
      \DeregKey{c} & \cauthor{c} = hk & hk \in \var{stkeys}
    }
    {
      \var{slot} \vdash
      \left(
      \begin{array}{r}
        \var{stkeys} \\
        \var{rewards} \\
        \var{delegations}
      \end{array}
      \right)
      \trans{deleg}{\var{c}}
      \left(
      \begin{array}{rcl}
        \{\var{hk}\} & \subtractdom & \var{stkeys}\\
        \{\addrRw \var{hk}\} & \subtractdom & \var{rewards} \\
        \{\var{hk}\} & \subtractdom & \var{delegations}
      \end{array}
      \right)
    }
  \end{equation}

  \begin{equation}\label{eq:deleg-deleg}
    \inference[Deleg-Deleg]
    {
      \Delegate{c} & \cauthor{c} = hk & hk \in \var{stkeys}
    }
    {
      \var{slot} \vdash
      \left(
      \begin{array}{r}
        \var{stkeys} \\
        \var{rewards} \\
        \var{delegations}
      \end{array}
      \right)
      \trans{deleg}{c}
      \left(
      \begin{array}{rcl}
        \var{stkeys} \\
        \var{rewards} \\
        \var{delegations} & \unionoverrideRight & \{\var{hk} \mapsto \dpool c\}
      \end{array}
      \right)
    }
  \end{equation}
  \caption{Delegation Inference Rules}
  \label{fig:delegation-rules}
\end{figure}



\subsection{Stake Pool Rules}
\label{sec:pool-rules}


The rules for updating the part of the ledger state defining the current stake
pools are given in \cref{fig:pool-rules}.
For each rule, again, we first check that a given certificate $c$ is of
the correct type using the
corresponding $\RegPool$ and $\RetirePool$ predicates.

The first rule, \cref{eq:pool-reg}, can be used to register a new stake pool, register a
stake pool with a new certificate ($\var{c}$), or stop the retirement
process for that key.  When this rule is invoked,
 the author's key and corresponding certificate along with the current slot
 number are added to the set of stake pools,
potentially overwriting a previous registration of a stake pool with
 that key. Then, the author's hash key and associated data
 will be removed from the $\var{retiring}$ parameter to cancel any pending
retirement for that key.

The second rule, \cref{eq:pool-ret}, starts the pool retirement process. Given a
slot number $\var{slot}$, the application of this rule requires that the
planned retirement epoch $\var{e}$ stated in the certificate is in the future,
i.e. after
$\var{cepoch}$, the epoch of the current slot number in this context, as well as
that it is
less than $\emax$ epochs after the current one. Another precondition for this
rule is that
the certificate author's key is in the set of the registered pools' keys.
This rule simply adds the pair of the
certificate author's key and the planned retirement epoch $e$ to the $\var{retiring}$
parameter, overrwiting any existing retirement schedule for the key.

Note that imposing the $\emax$ constraint on the system is not strictly necessary.
However, forcing stake pools to announce their retirement a shorter time in
advance will curb the growth of the $\var{retiring}$ list in the ledger state
(this is a finite map used to keep track of what stake pool is retiring when,
discussed in more detail later).
Allowing pools to make retirement announcements arbitrarily far in advance
could result in a linear (in the number of pools) growth of storage space needed
for this datatype.

The third rule, \cref{eq:pool-reap}, is invoked when the planned retirement epoch
$e$ for some non-empty set of
stake pools in the $\var{retiring}$ set is reached. In this case, all the
pairs associated with hash keys of pools which are scheduled to retire in this
epoch are removed
from both $\var{stpools}$ and $\var{retiring}$. Note that no certificate is
necessary here in order to apply this rule (the signal is empty).

\begin{todo}
  Discuss how signal $c$ works here
\end{todo}

\begin{todo}

  It seems like it would make sense to share data in the `retiring` variable
  and build a rule to manipulate it (through a ledger-blockchain interface)
\end{todo}

%%
%% Figure - Pool Rules
%%
\begin{figure}
  \begin{equation}\label{eq:pool-reg}
    \inference[Pool-Reg]
    {
      \RegPool{c}
      & \cauthor{c} = \var{hk}
      & \stakepool{c} = \var{stakepool}
    }
    {
      \var{slot} \vdash
      \left(
      \begin{array}{r}
        \var{stpools} \\
        \var{pparams} \\
        \var{retiring}
      \end{array}
      \right)
      \trans{pool}{c}
      \left(
      \begin{array}{rcl}
        \var{stpools} & \unionoverrideLeft
                      & \{\var{hk} \mapsto \var{slot}\} \\
        \var{pparams} & \unionoverrideRight
                      & \{\var{hk} \mapsto \var{stakepool}\} \\
        \{\var{hk}\} & \subtractdom & \var{retiring} \\
      \end{array}
      \right)
    }
  \end{equation}


  \begin{equation}\label{eq:pool-ret}
    \inference[Pool-Retire]
    {
    \RetirePool{c}
    & \cauthor{c} = hk
    & \var{hk} \in \dom \var{stpools} \\
    \var{e} = \retire{c}
    & \var{cepoch} = \epoch{slot}
    & \var{cepoch} < \var{e} < \var{cepoch} + \emax
  }
  {
    \var{slot} \vdash
    \left(
      \begin{array}{r}
        \var{stpools} \\
        \var{pparams} \\
        \var{retiring}
      \end{array}
    \right)
    \trans{pool}{c}
    \left(
      \begin{array}{rcl}
        \var{stpools} \\
        \var{pparams} \\
        \var{retiring} & \unionoverrideRight & \{\var{hk} \mapsto \var{e}\} \\
      \end{array}
    \right)
  }
  \end{equation}

  \begin{equation}\label{eq:pool-reap}
    \inference[Pool-Reap]
    {
      \var{retired} = \var{retiring}^{-1}~\var{(\epoch{slot})}
      & \var{retired} \neq \emptyset
    }
    {
      \var{slot} \vdash
      \left(
      \begin{array}{r}
        \var{stpools} \\
        \var{pparams} \\
        \var{retiring}
      \end{array}
      \right)
      \trans{poolreap}{}
      \left(
      \begin{array}{rcl}
        \var{retired} & \subtractdom & \var{stpools} \\
        \var{retired} & \subtractdom & \var{pparams} \\
        \var{retired} & \subtractdom & \var{retiring} \\
      \end{array}
      \right)
    }
  \end{equation}
  \caption{Pool Inference Rule}
  \label{fig:pool-rules}

\end{figure}

\subsection{Witnesses}
\label{sec:delegation-witnesses}


The new definitions needed for updating the ledger state with a given
certificate witnessed by some key-signature pair
are given in Figure~\ref{fig:defs:delegationw}. The $\fun{dwit}$ returns
the witness of a given certificate, and $\fun{dbody}$ gives the body of a
certificate. It is important to make this separation of certificate data
in order to avoid signing signatures.

We include $\var{slot}$ in the context ($\DWEnv$) because it is needed to update
$\PState$, as well as the transaction $\var{tx}$ which is being witnessed.
Note here that the transaction here is part of the context, not the signal.
(the signal is the certificate to be applied).

The type of the delegation witness
transition in this figure is meant to describe a complete ledger state
transition, including an update of both the delegations and the stake pools
(i.e. $\DState$ as well as the $\PState$, which together we call $\DWState$).
Now, such a transition is
composable with any other witnessed ledger state transition.

The rule for certificate witnesses is given in
Figure~\ref{fig:rules:delegationw}. The actual update of the ledger state is
consistent with the rules given in Figures~\ref{fig:delegation-rules}
and~\ref{fig:pool-rules}. However, we now add the following precondition for the
combined application of these rules:

\begin{itemize}
\item The signature of the transaction and certificate must be verified by the
  verification key with which the certificate is associated (i.e. witnessed).
\end{itemize}

That is, when this condition is met and we have valid $\DState$ and $\PState$
transitions in a given context $\DWEnv$, both triggered by certificate $\var{c}$,
according to this inference rule, the combination of these transitions is a
valid witnessed delegation transition.


\begin{figure}
  \emph{Abstract types}
  \begin{equation*}
    \begin{array}{r@{~\in~}lr}
      tx & \Tx & \text{transaction}\\
      dcb & \DCertBody & \text{certificate body}\\
    \end{array}
  \end{equation*}
  %
  \emph{Abstract functions}
  \begin{equation*}
    \begin{array}{r@{~\in~}lr}
      \fun{dbody} & \DCert \mapsto \DCertBody
      & \text{body of the delegation certificate}\\
      \fun{dwit} & \DCert \mapsto (\VKey \times \Sig)
      & \text{witness for the delegation certificate}\\
      \fun{dcerts} & \Tx \to \DCert
      & \text{delegation certificate in a transaction}\\
    \end{array}
  \end{equation*}
  %
  \emph{Delegation Witnesses environment}
  \begin{equation*}
    \DWEnv =
    \left(
      \begin{array}{r@{~\in~}lr}
        \var{tx} & \Tx & \text{transaction}\\
        \var{slot} & \Slot & \text{slot}\\
      \end{array}
    \right)
  \end{equation*}
  %
  \emph{Delegation Witnesses state}
  \begin{equation*}
    \DWState =
    \left(
      \begin{array}{r@{~\in~}lr}
        \var{dstate} & \DState & \text{delegation state}\\
        \var{pstate} & \PState & \text{pool state}\\
      \end{array}
    \right)
  \end{equation*}
  %
  \emph{Delegation Witnesses Transition}
  \begin{equation*}
    \_ \vdash \_ \trans{delegw}{\_} \_ \in
      \powerset (
        \DWEnv \times \DWState \times \DCert \times \DWState)
  \end{equation*}
  \caption{Delegation witnesses definitions}
  \label{fig:defs:delegationw}
\end{figure}

\begin{figure}
  \emph{Delegation with witness rule}
  \begin{equation}
    \label{eq:deleg-witnesses-keys}
    \inference[Deleg-wit-keys]
    { \dwit{c} = (\var{vk_s}, \sigma)
      & \var{slot} \vdash \var{dstate} \trans{deleg}{c} \var{dstate'}
      \\ ~ \\
      \verify{vk_s}{\serialised{(\dbody{c},~\txbody \var{tx})}}{\sigma}
    }
    { \left(
      \begin{array}{r}
        \var{tx} \\
        \var{slot}
      \end{array}
      \right)
      \vdash
      \left(
      \begin{array}{r}
        \var{dstate} \\
        \var{pstate}
      \end{array}
      \right)
      \trans{delegw}{c}
      \left(
      \begin{array}{rcl}
        \var{dstate'} \\
        \var{pstate'}
      \end{array}
      \right)
    }
  \end{equation}

  \begin{equation}
    \label{eq:deleg-witnesses-pools}
    \inference[Deleg-wit-pools]
    { \dwit{c} = (\var{vk_s}, \sigma)
      & \var{slot} \vdash \var{pstate} \trans{pool}{c} \var{pstate'}
      \\ ~ \\
      \verify{vk_s}{\serialised{(\dbody{c},~\txbody \var{tx})}}{\sigma}
    }
    { \left(
      \begin{array}{r}
        \var{tx} \\
        \var{slot}
      \end{array}
      \right)
      \vdash
      \left(
      \begin{array}{r}
        \var{dstate} \\
        \var{pstate}
      \end{array}
      \right)
      \trans{delegw}{c}
      \left(
      \begin{array}{rcl}
        \var{dstate'} \\
        \var{pstate'}
      \end{array}
      \right)
    }
  \end{equation}
  \caption{Delegation witnesses inference rules}
  \label{fig:rules:delegationw}
\end{figure}




\begin{figure}
  \begin{equation}
    \inference[Seq-delg-base]
    {}
    {
      \begin{array}{r}
        \var{tx}\\
        \var{slot}
      \end{array}
      \vdash
      \var{dwstate}
      \trans{delegs}{\epsilon}
      \var{dwstate'}
    }
    \label{eq:rule:sequence-delegation-base}
  \end{equation}

  \begin{equation}
    \inference[Seq-delg-ind]
    {
      {
        \begin{array}{r}
          \var{tx}\\
          \var{slot}\\
        \end{array}
      }
      \vdash
      \var{dwstate}
      \trans{delegs}{\Gamma}
      \var{dwstate'}
    \\ ~ \\
    {
      \begin{array}{r}
        \var{tx}\\
        \var{slot}\\
      \end{array}
    }
    \vdash
      \var{dwstate'}
      \trans{delegw}{c}
      \var{dwstate''}
    }
    {
      \var{dwstate}
      \trans{delegs}{\Gamma; c}
      \var{dwstate''}
    }
    \label{eq:rule:sequence-delegation-inductive}
  \end{equation}
  \caption{Delegation sequence rules}
  \label{fig:rules:delegation-sequence}
\end{figure}

\begin{todo}
  How should we integrate the \textsc{poolreap} transition? It is conceptually
  different from the others, as it does not require a certificate, but is only
  slot / epoch dependent.
\end{todo}


\section{UTxO}
\label{sec:state-trans-utxo-1}

In order to define the \textit{preservation of value} conditon,
we define the calculations for deposits and refunds in
Figure~\ref{fig:defs:deposits}.

\begin{note}
  We define $\fun{refund}$ by cases on whether or not
  the refunding certificate has a corresponding
  resource creating certificate.
  If our rules are correct, then $\fun{refund}$
  is only called in the case where such a matching
  certificate exists.
\end{note}

The transition rules for unspent outputs are presented in
Figure~\ref{fig:rules:utxo}. The states of the UTxO transition system,
along with their types are defined in Figure~\ref{fig:defs:utxo}.
Functions on these types are defined in Figure~\ref{fig:derived-defs:utxo}.



\begin{figure*}
  \emph{Abstract types}
  \begin{equation*}
    \begin{array}{r@{~\in~}lr}
      pc & \PrtclConsts & \text{protocol constants}
    \end{array}
  \end{equation*}
  %
  \emph{Derived types}
  \begin{equation*}
    \begin{array}{r@{~\in~}l@{\qquad=\qquad}r@{~\in~}lr}
      \var{allocs}
      & \Allocs
      & hkeys \mapsto slot
      & \Hash \to \Slot
      & \text{resource allocations}
    \end{array}
  \end{equation*}
  %
  \emph{Abstract Functions}
  \begin{equation*}
    \begin{array}{r@{~\in~}lr}
      \fun{dvalue} & \PrtclConsts \to \DCert \to \Coin
        & \text{deposit amount of a certificate}\\

      \fun{decay} & \PrtclConsts \to \mathbb{N}\times\mathbb{Q}^{+}
        & \text{decay constants}\\

      \fun{dresource} & \Tx \to \powerset{(\DCertRegKey \uniondistinct \DCertRegPool)}
        & \text{resource allocating certificates}\\

      \fun{dderegister} & \Tx \to \powerset{\DCertDeRegKey}
        & \text{resource releasing certificates}\\

      \fun{dretire} & \Tx \to \powerset{\DCertRetirePool}
        & \text{resource releasing certificates}\\

      \fun{ttl} & \Tx \to \Slot
        & \text{time to live}\\
    \end{array}
  \end{equation*}
  \caption{Definitions used in Deposits}
  \label{fig:defs:deposits}
\end{figure*}

\begin{figure}
  \begin{align*}
      & \fun{deposits} \in \PrtclConsts \to \Tx \to \Coin
      & \text{total deposits for transaction} \\
      & \fun{deposits}~{pc}~{tx} = \sum\limits_{c \in \fun{dresource}~tx} (\fun{d}~pc~c)
      \nextdef
      & \fun{poolAllocs} \in (\Hash \mapsto (\DCertRegPool \times \Slot)) \to \Allocs
      & \text{pool allocations} \\
      & \fun{poolAllocs}~\var{stpool} =
          \{hash \mapsto slot \mid hash \mapsto (\_, slot) \in \var{stpool}\}
      \nextdef
      & \fun{refund} \in \PrtclConsts \to \Allocs \to \Slot \to \DCert \to \Coin
      & \text{total refund for a certificate} \\
      & \refund{allocs}{pc}{slot}{c} =\\
      & \begin{cases}
        0 & \text{if not}~(\fun{releasing}~c)\\
            0 & \text{if}~\cauthor c \notin allocs\\
            \floor*{
              \left(\fun{d}~pc~c\right) \cdot
            \left(d_{\min}+(1-d_{\min})\cdot e^{-\lambda\cdot\delta}\right)}
            & \text{otherwise}
        \end{cases}\\
      &
      \begin{array}{lr@{~=~}l}
        \where &\fun{releasing}~\var{c} & \DeregKey{c} \lor \RetirePool{c}\\
        & d_{\min},~\lambda & \fun{decay}~pc\\
        &\delta & \slotminus{slot}{(allocs~(\cauthor c))}\\
      \end{array}\\
      \nextdef
      & \fun{refunds} \in \PrtclConsts \to \Allocs \to \Allocs \to \Tx \to \Coin
      & \text{total refunds for transaction} \\
      & \refunds{pc}{dallocs}{pallocs}{tx} =\\
      &   \sum\limits_{c \in \fun{dderegister}~tx} \refund{pc}{dallocs}{(\ttl{tx})}{c}\\
      &   ~~~+ \sum\limits_{c \in \fun{dretire}~tx} \refund{pc}{pallocs}{(\retire{c})}{c}
  \end{align*}
  \caption{Functions used in Deposits}
  \label{fig:functions:deposits}
\end{figure}


\begin{figure*}
  \emph{Abstract types}
  %
  \begin{equation*}
    \begin{array}{r@{~\in~}lr}
      \var{txid} & \TxId & \text{transaction id}\\
      %
      ix & \Ix & \text{index}\\
      %
      \var{addr} & \Addr & \text{address}\\
      %
      c & \Coin & \text{currency value}\\
      %
      slot & \Slot & \text{slot}
    \end{array}
  \end{equation*}
  \emph{Derived types}
  %
  \begin{equation*}
    \begin{array}{r@{~\in~}l@{\qquad=\qquad}r@{~\in~}lr}
      \var{txin}
      & \TxIn
      & (\var{txid}, \var{ix})
      & \TxId \times \Ix
      & \text{transaction input}
      \\
      \var{txout}
      & \type{TxOut}
      & (\var{addr}, c)
      & \Addr \times \Coin
      & \text{transaction output}
      \\
      \var{utxo}
      & \UTxO
      & \var{txin} \mapsto \var{txout}
      & \TxIn \mapsto \TxOut
      & \text{unspent tx outputs}
    \end{array}
  \end{equation*}
  %
  \emph{Abstract Functions}
  \begin{equation*}
    \begin{array}{r@{~\in~}lr}
      \txid{} & \Tx \to \TxId & \text{compute transaction id}\\
      %
      \fun{txbody} & \Tx \to \powerset{\TxIn} \times (\Ix \mapsto \TxOut)
                                  & \text{transaction body}\\
      %
      \fun{txfee} & \Tx \to \Coin & \text{transaction fee}\\
      %
      \fun{minfee} & \PrtclConsts \to \Tx \to \Coin & \text{minimum fee}
    \end{array}
  \end{equation*}
  \caption{Definitions used in the UTxO transition system}
  \label{fig:defs:utxo}
\end{figure*}

\begin{figure}
  \begin{align*}
    & \fun{txins} \in \Tx \to \powerset{\TxIn}
    & \text{transaction inputs} \\
    & \txins{tx} = \var{inputs} \where \txbody{tx} = (\var{inputs}, ~\wcard)
    \nextdef
    & \fun{txouts} \in \Tx \to \UTxO
    & \text{transaction outputs as UTxO} \\
    & \fun{txouts} ~ \var{tx} =
      \left\{ (\fun{txid} ~ \var{tx}, \var{ix}) \mapsto \var{txout} ~
      \middle| \begin{array}{l@{~}c@{~}l}
                 (\_, \var{outputs}) & = & \txbody{tx} \\
                 \var{ix} \mapsto \var{txout} & \in & \var{outputs}
               \end{array}
      \right\}
    \nextdef
    & \fun{balance} \in \UTxO \to \Coin
    & \text{UTxO balance} \\
    & \fun{balance} ~ utxo = \sum_{(~\wcard ~ \mapsto (\wcard, ~c)) \in \var{utxo}} c
    \nextdef
    & \fun{created} \in \PrtclConsts \to \UTxO \to \Allocs \to \Allocs \to \Tx \to \Coin
    & \text{value created} \\
    & \created{pc}{utxo}{dallocs}{pallocs}{tx} = \\
    & ~~\balance{(\txins{tx} \restrictdom \var{utxo})} + \refunds{pc}{dallocs}{pallocs}{tx}
    \nextdef
    & \fun{destroyed} \in \PrtclConsts \to \Tx \to \Coin
    & \text{value destroyed} \\
    & \fun{destroyed} ~ pc ~ tx =
      \balance{(\txouts{tx})}  + \txfee{tx} + \deposits{pc}{tx}\\
  \end{align*}

  \begin{align*}
    \var{ins} \restrictdom \var{utxo}
    & = \{ i \mapsto o \mid i \mapsto o \in \var{utxo}, ~ i \in \var{ins} \}
    & \text{domain restriction}
    \\
    \var{ins} \subtractdom \var{utxo}
    & = \{ i \mapsto o \mid i \mapsto o \in \var{utxo}, ~ i \notin \var{ins} \}
    & \text{domain exclusion}
    \\
    \var{utxo} \restrictrange \var{outs}
    & = \{ i \mapsto o \mid i \mapsto o \in \var{utxo}, ~ o \in \var{outs} \}
    & \text{range restriction}
  \end{align*}
  \caption{Functions used in UTxO rules}
  \label{fig:derived-defs:utxo}
\end{figure}

\begin{figure}
  \emph{UTxO environment}
  \begin{equation*}
    \UTxOEnv =
    \left(
      \begin{array}{r@{~\in~}lr}
        \var{slot} & \Slot & \text{current slot}\\
        \var{pc} & \PrtclConsts & \text{protocol constants}\\
        \var{dallocs} & \Allocs & \text{stake key allocations}\\
        \var{pallocs} & \Allocs & \text{stake pool allocations}\\
      \end{array}
    \right)
  \end{equation*}
  %
  \emph{UTxO transitions}
  \begin{equation*}
    \_ \vdash
    \var{\_} \trans{utxo}{\_} \var{\_}
    \subseteq \powerset (\UTxOEnv \times \UTxO \times \Tx \times \UTxO)
  \end{equation*}
  \caption{UTxO transition-system types}
  \label{fig:ts-types:utxo}
\end{figure}

\begin{figure}
  \begin{equation}\label{eq:utxo-inductive}
    \inference[UTxO-inductive]
    { \ttl tx \leq \var{slot}
      & \txins{tx} \neq \emptyset
      \\
      \txins{tx} \subseteq \dom \var{utxo}
      & \minfee{pc}{tx} \leq \txfee{tx}
      \\
      \created{pc}{utxo}{dallocs}{pallocs}{tx} = \destroyed{pc}{tx}
    }
    {
      \begin{array}{l}
        \var{slot}\\
        \var{pc}\\
        \var{dallocs}\\
        \var{pallocs}\\
      \end{array}
      \vdash \var{utxo} \trans{utxo}{tx}
      (\txins{tx} \subtractdom \var{utxo}) \cup \txouts{tx}
    }
  \end{equation}
  \caption{UTxO inference rules}
  \label{fig:rules:utxo}
\end{figure}

Rule~\ref{eq:utxo-inductive} specifies under which conditions a transaction can
be applied to a set of unspent outputs, and how the set of unspent output
changes with a transaction:
\begin{itemize}
\item Each input spent in the transaction must be in the set of unspent
  outputs.
\item The balance of the unspent outputs in a transaction (i.e. the total
  amount paid in a transaction) must be equal or less than the amount of spent
  inputs.
\item If the above conditions hold, then the new state will not have the inputs
  spent in transaction $\var{tx}$ and it will have the new outputs in
  $\var{tx}$.
\end{itemize}

\begin{note}
  $\Coin$ is defined as a primitive type, but there is a difference
  between implementing it with $\mathbb{N}$ versus $\mathbb{Z}$.
  Since this is a pure UTxO ledger, $\mathbb{N}$ suffices.
  If, however, $\mathbb{Z}$ is used, then extra validation is required
  to ensure that all $\TxOut$ are non-negative.
  This extra condition would be added to \cref{eq:utxo-inductive}.
\end{note}

\subsection{Properties}
\label{sec:utxo-properties}

\begin{todo}
  Can we prove properties of the transition system of this section? For
  instance we might like to formalize ``double spending'' and prove that these
  rules prevent it. Do we want it?
\end{todo}

\subsection{Witnesses}
\label{sec:witnesses}

The rules for witnesses are presented in Figure~\ref{fig:rules:utxow}.
The definitions used in Rule~\ref{eq:utxo-witness-inductive} are given in
Figure~\ref{fig:defs:utxow}. Note that
Rule~\ref{eq:utxo-witness-inductive} uses the transition relation defined in
Figure~\ref{fig:rules:utxo}. The main reason for doing this is to define
the rules incrementally, modeling different aspects in isolation to keep the
rules as simple as possible. Also note that the $\trans{utxo}{}$ relation could
have been defined in terms of $\trans{utxow}{}$ (thus composing the rules in a
different order). The choice here is arbitrary.

\begin{figure}
  \emph{Abstract functions}
  %
  \begin{equation*}
    \begin{array}{r@{~\in~}lr}
      \fun{wits} & \Tx \to \powerset{(\VKey \times \Sig)}
      & \text{witnesses of a transaction}\\
      \fun{hash_{spend}} & \Addr \mapsto \Hash
      & \text{hash of a spending key in an address}\\
    \end{array}
  \end{equation*}
  \caption{Definitions used in the UTxO transition system with witnesses}
  \label{fig:defs:utxow}
\end{figure}

\begin{figure}
  \begin{align*}
    & \addr{}{} \in \UTxO \to \TxIn \mapsto \Addr & \text{address of an input}\\
    & \addr{utxo} = \{ i \mapsto a \mid i \mapsto (a, \wcard) \in \var{utxo} \} \\
    \nextdef
    & \fun{addr_h} \in \UTxO \to \TxIn \mapsto \Hash & \text{hash of an input address}\\
    & \fun{addr_h}~utxo = \{ i \mapsto h \mid i \mapsto (a, \wcard) \in \var{utxo}
      \wedge a \mapsto h \in \fun{hash_{spend}} \}
  \end{align*}
  \caption{Functions used in rules witnesses}
  \label{fig:derived-defs:utxow}
\end{figure}

\begin{figure}
  \emph{UTxO with witness transitions}
  \begin{equation*}
    \var{\_} \vdash
    \var{\_} \trans{utxow}{\_} \var{\_}
    \subseteq \powerset (\UTxOEnv \times \UTxO \times \Tx \times \UTxO)
  \end{equation*}
  \caption{UTxO with witness transition-system types}
  \label{fig:ts-types:utxow}
\end{figure}

\begin{figure}
  \begin{equation}
    \label{eq:utxo-witness-inductive}
    \inference[UTxO-wit]
    {
      {
        \begin{array}{l}
        \var{slot}\\
        \var{pc}
        \end{array}
      }
      \vdash \var{utxo} \trans{utxo}{tx} \var{utxo'}\\ ~ \\
      & \forall i \in \txins{tx} \cdot \exists (\var{vk}, \sigma) \in \wits{\var{tx}}
      \cdot
      \mathcal{V}_{\var{vk}}{\serialised{\txbody{tx}}}_{\sigma}
      \wedge  \fun{addr_h}~{utxo}~i = \hash{vk}\\
    }
    {
      \begin{array}{l}
        \var{slot}\\
        \var{pc}\\
      \end{array}
      \vdash \var{utxo} \trans{utxow}{tx} \var{utxo'}
    }
  \end{equation}
  \caption{UTxO with witnesses inference rules}
  \label{fig:rules:utxow}
\end{figure}

\begin{figure}
  \emph{Ledger environment}
  \begin{equation*}
    \LEnv =
    \left(
      \begin{array}{r@{~\in~}lr}
        \var{slot} & \Slot & \text{current slot}\\
        \var{pc} & \PrtclConsts & \text{protocol constants}\\
      \end{array}
    \right)
  \end{equation*}
  %
  \emph{Ledger state}
  \begin{equation*}
    \LState =
    \left(
      \begin{array}{r@{~\in~}lr}
        \var{utxo} & \UTxO & \text{UTxO}\\
        \var{dwstate} & \DWState & \text{delegation witnesess state}\\
      \end{array}
    \right)
  \end{equation*}
  %
  \emph{Ledger transitions}
  \begin{equation*}
    \_ \vdash
    \var{\_} \trans{ledger}{\_} \var{\_}
    \subseteq \powerset (\LEnv \times \LState \times \Tx \times \LState)
  \end{equation*}
  \caption{Ledger transition-system types}
  \label{fig:ts-types:ledger}
\end{figure}

\begin{figure}
  \begin{equation}
    \label{eq:ledger}
    \inference[ledger]
    {
      \Gamma = \dcerts{tx}
      & pallocs = \fun{poolAllocs}~stpools\\~\\
      {
        \begin{array}{r}
        slot\\
        pc\\
        stkeys\\
        pallocs\\
        \end{array}
      }
      \vdash \var{utxo} \trans{utxow}{tx} \var{utxo'}\\~\\~\\
      %
      {
        \begin{array}{l}
          tx \\
          slot \\
        \end{array}
      }
      \vdash
      dwstate \trans{delegs}{\Gamma} dwstate'
    }
    {
      \begin{array}{l}
        \var{slot}\\
        \var{pc}\\
      \end{array}
      \vdash
      \left(
        \begin{array}{ll}
          utxo \\
          dwstate \\
        \end{array}
      \right)
      \trans{ledger}{certs}
      \left(
        \begin{array}{ll}
          utxo' \\
          dwstate' \\
        \end{array}
      \right)
    }
  \end{equation}
  \caption{Ledger inference rule}
  \label{fig:rules:ledger}
\end{figure}

\section{Properties}
\label{sec:properties}
In this section we discuss the properties which we want the ledger to have. One
goal is to include these properties in the executable specification for doing
property-based testing or formal verification.

\subsection{Validity of a Ledger State}
\label{sec:valid-ledg-state}

Many properties only make sense when applied to a valid ledger state. In
informal terms, a valid ledger state $l$ can only be reached when starting from
an initial state $l_{0}$ (genesis state) and only executing state transition
rules as specified in Section~\ref{sec:state-trans-utxo-1} for UTxO or
Section~\ref{sec:delegation} for delegation.

\begin{figure}[ht]
  \centering
  \begin{align*}
    \genesisId & \in & \TxId \\
    \genesisTxOut & \in & \TxOut \\
    \genesisUTxO & \coloneqq & \{\genesisId, \emptyset\} \mapsto \genesisTxOut
    \\
    \ledgerState & \in & \left(
                         \begin{array}{c}
                           \UTxO \\
                           \DState \\
                           \PState
                         \end{array}
    \right)\\
               && \\
    \fun{getUTxO} & \in & \ledgerState \to \UTxO \\
    \fun{getUTxO} & \coloneqq & (\var{utxo}, \wcard, \wcard) \to \var{utxo}
  \end{align*}
  \caption{Valid Ledger State}
  \label{fig:valid-ledger}
\end{figure}

In Figure~\ref{fig:valid-ledger} \genesisId{} marks the transaction identifier
of the initial coin distribution, where \genesisTxOut{} represents the initial
UTxO. It should be noted that no corresponding inputs exists, i.e., the
transaction inputs are the empty set for the initial transaction. An element of
\ledgerState{} is a triplet of UTxO, stake delegation state (\DState) and
delegation pool state (\PState).

\begin{definition}[\textbf{Valid Ledger State}]
  \begin{multline*}
    \label{eq:2}
    \forall l_{0},..,l_{n} \in \ledgerState, l_{0} =
    \left(
      \begin{array}{c}
        \left\{
        \genesisUTxO
        \right\} \\
        \emptyset\\
        \emptyset\\
      \end{array}
    \right)  \\
    \implies \forall 0 < i \leq n, ((\exists c \in \DCert, l_{i-1}
    \trans{delegw}{c} l_{i}) \vee (\exists tx \in \Tx: l_{i-1} \trans{utxow}{tx}
    l_{i}))\\ \implies \applyFun{validLedgerState}(l_{n})
  \end{multline*}
  \label{def:valid-ledger-state}
\end{definition}

Definition~\ref{def:valid-ledger-state} defines a valid ledger state reachable
from the genesis state via valid UTxO, stake delegation or stake pool
transactions. This gives a constructive rule how to reach a valid ledger state.

\subsection{Ledger Properties}
\label{sec:ledger-properties}

The following properties state the desired features of updating a valid ledger
state.

\begin{property}[\textbf{Preserve Balance Modulo Fee}]
  \begin{multline*}
    \forall \var{l}, \var{l'} \in \ledgerState: \applyFun{validLedgerstate}{l}\\
    \implies \forall \var{tx} \in \Tx, \var{l} \trans{utxo}{tx} \var{l'}
    \implies \fun{balance}(\applyFun{getUTxO}{l}) =
    \fun{balance}(\applyFun{getUTxO}{l'}) + \applyFun{txfee}{tx}
  \end{multline*}
  \label{prop:ledger-properties-1}
\end{property}

Property~\ref{prop:ledger-properties-1} states that for each valid ledger $l$,
if a transaction $tx$ is added to the ledger via the state transition rule
$utxow$ to the new ledger state $l'$, the balance of the UTxOs in $l$ equals the
balance of the UTxOs in $l'$ minus the transaction fees.

\begin{property}[\textbf{Preserve Balance Restricted to TxIns in Balance of
    TxOuts}]
  \begin{multline*}
    \forall \var{l}, \var{l'} \in \ledgerState: \applyFun{validLedgerstate}{l}\\
    \implies \forall \var{tx} \in \Tx, \var{l} \trans{utxo}{tx} \var{l'}
    \implies \fun{balance}(\applyFun{txins}{tx} \restrictdom
    \applyFun{getUTxO}{l}) = \fun{balance}(\applyFun{txouts}{tx}) +
    \applyFun{txfee}{tx}
  \end{multline*}
  \label{prop:ledger-properties-2}
\end{property}

Property~\ref{prop:ledger-properties-2} states the more detailed relation of the
balances change. For ledgers $l, l'$ and a transaction $tx$ as above, the
balance of the UTxOs of $l$ restricted to those whose domain is in the set of
transaction inputs of $tx$ equals the balance of the transaction outputs of $tx$
minus the transaction fees.

\begin{property}[\textbf{Preserve Outputs of Transaction}]
  \begin{multline*}
    \forall \var{l}, \var{l'} \in \ledgerState: \applyFun{validLedgerstate}{l}\\
    \implies \forall \var{tx} \in \Tx, \var{l} \trans{utxo}{tx} \var{l'}
    \implies \forall \var{out} \in \applyFun{txouts}{tx}, out \in
    \applyFun{getUTxO}{l'}
  \end{multline*}
  \label{prop:ledger-properties-3}
\end{property}

Property~\ref{prop:ledger-properties-3} states that for every ledger states
$l, l'$ and transaction $tx$ as above, all output UTxOs of $tx$ are in the UTxO
set of $l'$, i.e., they are now available as unspent transaction output.

\begin{property}[\textbf{Eliminate Inputs of Transaction}]
  \begin{multline*}
    \forall \var{l}, \var{l'} \in \ledgerState: \applyFun{validLedgerstate}{l}\\
    \implies \forall \var{tx} \in \Tx, \var{l} \trans{utxo}{tx} \var{l'}
    \implies \forall \var{in} \in \applyFun{txins}{tx}, in \not\in
    \fun{dom}(\applyFun{getUTxO}{l'})
  \end{multline*}
  \label{prop:ledger-properties-4}
\end{property}

Property~\ref{prop:ledger-properties-4} states that for every ledger states
$l, l'$ and transaction $tx$ as above, all transaction inputs $in$ of $tx$ are
not in the domain of the UTxO set of $l'$, i.e., these are no longer available
to spend.

\begin{property}[\textbf{Completeness and Collision-Freeness of new Transaction
    Ids}]
  \begin{multline*}
    \forall \var{l}, \var{l'} \in \ledgerState: \applyFun{validLedgerstate}{l}\\
    \implies \forall \var{tx} \in \Tx, \var{l} \trans{utxo}{tx} \var{l'}
    \implies \forall utxo' \in \applyFun{txouts}{tx}, \var{utxo'} \in
    \applyFun{getUTxO}{l'} \wedge \\(\var{utxo'} = ((\var{txId'}, \wcard) \mapsto
    \wcard) \implies \forall \var{utxo} \in \applyFun{getUTxO}{l}, \var{utxo} =
    ((\var{txId}, \wcard) \mapsto \wcard) \implies \var{txId'} \neq \var{txId}
  \end{multline*}
  \label{prop:ledger-properties-5}
\end{property}

Property~\ref{prop:ledger-properties-5} states that for ledger states $l, l'$
and a transaction $tx$ as above, the UTxOs of $l'$ contain all newly created
UTxOs and the referred transaction id of each new UTxO is not used in the UTxO
set of $l$.

%%% Local Variables:
%%% mode: latex
%%% TeX-master: "ledger-spec"
%%% End:


\addcontentsline{toc}{section}{References}
\bibliographystyle{plainnat}
\bibliography{references}

\end{document}
