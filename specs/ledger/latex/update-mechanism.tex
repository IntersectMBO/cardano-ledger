\newcommand{\UProp}{\ensuremath{\type{UProp}}}
\newcommand{\UPropId}{\ensuremath{\type{UPropId}}}
\newcommand{\UPropSD}{\ensuremath{\type{UPropSD}}}
\newcommand{\ProtVer}{\ensuremath{\type{ProtVer}}}
\newcommand{\ProtAtt}{\ensuremath{\type{ProtAtt}}}
\newcommand{\ProtParams}{\ensuremath{\type{ProtParams}}}
\newcommand{\UPVEnv}{\ensuremath{\type{UPVEnv}}}
\newcommand{\UPVState}{\ensuremath{\type{UPVState}}}
\newcommand{\UPLEnv}{\ensuremath{\type{UPLEnv}}}
\newcommand{\UPLState}{\ensuremath{\type{UPLState}}}
\newcommand{\UPAEnv}{\ensuremath{\type{UPAEnv}}}
\newcommand{\UPRState}{\ensuremath{\type{UPRState}}}
\newcommand{\Vote}{\ensuremath{\type{Vote}}}
\newcommand{\VEnv}{\ensuremath{\type{VEnv}}}
\newcommand{\VState}{\ensuremath{\type{VState}}}

\newcommand{\upSize}[1]{\ensuremath{\fun{upSize}~\var{#1}}}
\newcommand{\upPV}[1]{\ensuremath{\fun{upPV}~\var{#1}}}
\newcommand{\upId}[1]{\ensuremath{\fun{upId}~\var{#1}}}
\newcommand{\upSig}[1]{\ensuremath{\fun{upSig}~\var{#1}}}
\newcommand{\upSigData}[1]{\ensuremath{\fun{upSigData}~\var{#1}}}
\newcommand{\upIssuer}[1]{\ensuremath{\fun{upIssuer}~\var{#1}}}
\newcommand{\upParams}[1]{\ensuremath{\fun{upParams}~\var{#1}}}
\newcommand{\vCaster}[1]{\ensuremath{\fun{vCaster}~\var{#1}}}
\newcommand{\vPropId}[1]{\ensuremath{\fun{vPropId}~\var{#1}}}
\newcommand{\vSig}[1]{\ensuremath{\fun{vSig}~\var{#1}}}

\lstset{ frame=tb,
       , language=Haskell
       , basicstyle=\footnotesize\ttfamily,
       , keywordstyle=\color{blue!80},
       , commentstyle=\itshape\color{purple!40!black},
       , identifierstyle=\bfseries\color{green!40!black},
       , stringstyle=\color{orange},
       }

\lstMakeShortInline[columns=fixed]`

\section{Update mechanism}
\label{sec:update}

This section formalizes the update mechanism by which the protocol parameters get
updated.

\subsection{Update proposals}
\label{sec:update-proposals}

\begin{figure}[htb]
  \emph{Abstract types}
  %
  \begin{equation*}
    \begin{array}{r@{~\in~}lr}
      \var{up} & \UProp & \text{(protocol) update proposal}\\
      \var{pa} & \ProtAtt & \text{update attributes}\\
      \var{swv} & \type{SWVer} & \text{update software version}\\
      \var{upd} & \type{UpdData} & \text{update data}\\
      \var{upa} & \type{UpdAttr} & \text{update attributes}\\
    \end{array}
  \end{equation*}
  %
  \emph{Derived types}
  \begin{equation*}
    \begin{array}{r@{~\in~}l@{~=~}r@{~\in~}lr}
      \var{pv} & \ProtVer & (\var{maj}, \var{min}, \var{alt})
      & (\mathbb{N}, \mathbb{N}, \mathbb{N}) & \text{protocol version}\\
      \var{pps} & \ProtParams & \var{pps} & \ProtAtt \mapsto \Value
                                             & \text{protocol parameters}\\
      \var{pb} & \UPropSD
      &
        {\left(\begin{array}{r l}
                 \var{pv}\\
                 \var{pps}\\
                 \var{swv}\\
                 \var{upd}\\
                 \var{upa}\\
               \end{array}\right)}
      & {
        \left(
        \begin{array}{l}
          \ProtVer\\
          \ProtParams\\
          \type{SWVer}\\
          \type{UpdData}\\
          \type{UpdAttr}\\
        \end{array}
                   \right)
                   }
                          & \text{protocol update signed data}
    \end{array}
  \end{equation*}
  \emph{Abstract functions}
  %
  \begin{equation*}
    \begin{array}{r@{~\in~}lr}
      \fun{upIssuer} & \UProp \to \VKeyGen & \text{issuer of the update proposal}\\
      \fun{upSize} & \UProp \to \mathbb{N} & \text{update proposal size}\\
      \fun{upPV} & \UProp \to \ProtVer & \text{update proposal protocol version}\\
      \fun{upId} & \UProp \to \UPropId & \text{update proposal id}\\
      \fun{upParams} & \UProp \to \mathbb{\ProtParams}
                                           & \text{new parameters that are proposed }\\
      \fun{upSig} & \UProp \to \Sig & \text{update proposal signature}\\
      \fun{upSigdata} & \UProp \to \UPropSD & \text{update proposal signed data}
    \end{array}
  \end{equation*}
  \caption{Update proposals definitions}
  \label{fig:defs:update-proposals}
\end{figure}

\subsection{Update proposals registration}
\label{sec:update-proposals-registration}

First we model the validity of a proposal.

\begin{figure}[htb]
  \emph{Update proposals validity environments}
  \begin{equation*}
    \UPVEnv =
    \left(
      \begin{array}{r@{~\in~}lr}
        \var{pv} & \ProtVer & \text{adopted (current) protocol version}\\
        \var{pps} & \ProtParams & \text{adopted protocol parameters}\\
      \end{array}
    \right)
  \end{equation*}
  %
  \emph{Update proposals validity states}
  \begin{equation*}
    \UPVState
    = \left(
      \begin{array}{r@{~\in~}lr}
        \var{rups} & \powerset{\UPropId \mapsto (\ProtVer \times \ProtParams)}
        & \text{registered update proposals}\\
      \end{array}
    \right)
  \end{equation*}
  %
  \emph{Update proposals validity transitions}
    \begin{equation*}
    \var{\_} \vdash
    \var{\_} \trans{upv}{\_} \var{\_}
    \subseteq \powerset (\UPVEnv \times \UPVState \times \UProp \times \UPVState)
  \end{equation*}
  \caption{Update proposals validity transition-system types}
  \label{fig:ts-types:up-validity}
\end{figure}

Terse explanation of Rule~\ref{eq:rule:up-validity}: a new proposal:
\begin{itemize}
\item must increase one of the (major, minor, or alternative) of the
  current version in a consistent manner:
  \begin{itemize}
  \item The proposed version must be lexicographically bigger than the current
    version.
  \item The major versions of the proposed and current version must differ in
    at most one.
  \item If the proposed major version is equal to the current major
    version, then the proposed minor version must be incremented by one.
  \item If the proposed major version is larger than the current major, then
    the proposed minor version must be zero.
  \end{itemize}
\item must be consistent with the current protocol parameters parameters:
  \begin{itemize}
  \item the proposal size must not exceed the maximum size specified by
    the current protocol parameters,
  \item the proposed new maximum block should be not greater than twice current
    maximum block size, and
  \item the proposed new script version can be incremented by at most 1.
  \end{itemize}
\item must be new:
  \begin{itemize}
  \item must not exist in the set of registered proposals, and
  \item must have a unique version among the current active proposals. This
    implies that a proposal is uniquely determined by the protocol version it
    proposes.
  \end{itemize}
\item the protocol version and parameters, along with some other data (see the
  definition of $\fun{upSigdata}$), must be signed by the proposal issuer.
\end{itemize}

\begin{figure}[htb]
  \begin{equation}
    \label{eq:func:can-follow}
    \begin{array}{r c l}
      \fun{canFollow}~(\var{mj_n}, \var{mi_n}, \var{a_n})~(\var{mj_p}, \var{mi_p}, \var{a_p})
      & = & (\var{mj_p}, \var{mi_p}, \var{a_p}) < (\var{mj_n}, \var{mi_n}, \var{a_n})\\
      & \wedge & 0 \leq \var{mj_n} - \var{mj_p} \leq 1\\
      & \wedge & (\var{mj_p} = \var{mj_n} \Rightarrow \var{mi_p} + 1 = \var{mi_n}))\\
      & \wedge & (\var{mj_p} + 1 = \var{mj_n} \Rightarrow \var{mi_n} = 0)
    \end{array}
  \end{equation}
  \nextdef
  \begin{equation}
    \label{eq:func:can-update}
    \begin{array}{r c l}
      \fun{canUpdate}~\var{pps}~\var{up}
      & = & \var{maxUpSize} \mapsto \var{mus} \in \var{pps}\\
      & \wedge & \upSize{up} \leq \var{mus}\\
      & \wedge & \var{maxBlockSize} \mapsto \var{bszm} \in \var{pps}\\
      & \wedge & \var{maxBlockSize} \mapsto \var{bsm_{up}} \in (\upParams{up})\\
      & \wedge & \var{bszm_{up}} \leq 2\var{bszm}\\
      & \wedge & \var{scriptVer} \mapsto \var{sv} \in \var{pps} \\
      & \wedge & \var{scriptVer} \mapsto \var{sv_{up}} \in (\upParams{up}) \\
      & \wedge &  0 \leq \var{sv_{up}} - \var{sv} \leq 1
    \end{array}
  \end{equation}
  \nextdef
  \begin{equation}
    \label{eq:func:is-new}
    \begin{array}{r c l}
      \fun{isNew}~\var{rups}~(\var{pid}, \var{nv})
      & = &  \var{pid} \notin \dom \var{rups}
            \wedge \var{nv} \notin \dom (\range \var{rups})\\
    \end{array}
  \end{equation}
  \caption{Update validity functions}
\end{figure}

\begin{figure}[htb]
  \begin{equation}
    \label{eq:rule:up-validity}
    \inference
    {
      \upIssuer{up} = \var{vk}
      & \upId{up} = \var{pid}
      & \upPV{up} = \var{nv}
      & \upParams{up} = \var{pps_n}\\
      & \fun{canFollow}~\var{nv}~\var{pv}
      & \fun{canUpdate}~\var{pps}~\var{up}
      & \fun{isNew}~\var{rups}~(\var{pid}, \var{nv})\\
      \mathcal{V}_{\var{vk}}\serialised{\upSigData{up}}_{(\upSig{up})}
    }
    {
      {
        \begin{array}{l}
          \var{pv}\\
          \var{pps}
        \end{array}
      }
      \vdash
      {
        \left(
          \begin{array}{l}
            \var{rups}
          \end{array}
        \right)
      }
      \trans{upv}{\var{up}}
      {
        \left(
          \begin{array}{l}
            \var{rups} \unionoverride \{ \var{pid} \mapsto (\var{nv}, \var{pps_n}) \}
          \end{array}
        \right)
      }
    }
  \end{equation}
  \caption{Update proposals validity rules}
  \label{fig:rules:up-validity}
\end{figure}

\clearpage

\begin{figure}[htb]
  \emph{Update proposals limits  environments}
    \begin{equation*}
    \UPLEnv =
    \left(
      \begin{array}{r@{~\in~}lr}
        \var{e_c} & \Epoch & \text{current epoch}\\
        \var{dms} & \VKeyGen \mapsto \VKey & \text{delegation map}\\
      \end{array}
    \right)
  \end{equation*}
  %
  \emph{Update proposals limits states}
  \begin{equation*}
    \UPLState
    = \left(
      \begin{array}{r@{~\in~}lr}
        \var{eps} & \powerset{(\Epoch \times \VKeyGen)} & \text{proposals per-epoch per-key}\\
      \end{array}
    \right)
  \end{equation*}
  %
  \emph{Update proposals limits transitions}
  \begin{equation*}
    \var{\_} \vdash
    \var{\_} \trans{upl}{\_} \var{\_}
    \subseteq \powerset (\UPLEnv \times \UPLState \times \UProp \times \UPLState)
  \end{equation*}
  \caption{Update proposals limits transition-system types}
  \label{fig:ts-types:up-limits}
\end{figure}

Terse explanation of Rule~\ref{eq:rule:up-limits}:
\begin{itemize}
\item We consider the update proposal issuers to be the delegators of the key
  ($\var{vk}$) that is associated with the proposal under consideration
  ($\var{up}$).
\item We check that no key that delegated to $\var{vk}$ has issued a proposal
  in the current epoch $\var{e_c}$.
\item If the above check succeeds, then all the delegators of $\var{vk}$ are
  added to the set of keys that proposed in this epoch.
\end{itemize}

\begin{figure}[htb]
  \begin{equation}
    \label{eq:rule:up-limits}
    \inference
    {\upIssuer{up} = \var{vk}
      & \var{eps_{vk}} = \{ (e_c, \var{vk_s})
      \mid \var{vk_s} \mapsto \var{vk} \in \var{dms}\}
      & \var{eps_{vk}} \cap \var{eps} = \emptyset
    }
    {
      {\begin{array}{l}
         \var{e_c}\\
         \var{dms}
       \end{array}
      }
      \vdash
      {
        \left(
          \begin{array}{l}
            \var{eps}
          \end{array}
        \right)
      }
      \trans{upl}{\var{up}}
      {
        \left(
          \begin{array}{l}
            \var{eps} \cup \var{eps_{vk}}
          \end{array}
        \right)
      }
    }
  \end{equation}
  \caption{Update proposals limits rules}
  \label{fig:rules:up-limits}
\end{figure}

\begin{figure}[htb]
  \emph{Update proposals registration  environments}
    \begin{equation*}
    \UPAEnv =
    \left(
      \begin{array}{r@{~\in~}lr}
        \var{pv} & \ProtVer & \text{adopted (current) protocol version}\\
        \var{pps} & \ProtParams & \text{adopted protocol parameters}\\
        \var{e_c} & \Epoch & \text{current epoch}\\
        \var{dms} & \VKeyGen \mapsto \VKey & \text{delegation map}\\
      \end{array}
    \right)
  \end{equation*}
  %
  \emph{Update proposals registration states}
  \begin{equation*}
    \UPRState
    = \left(
      \begin{array}{r@{~\in~}lr}
        \var{rups} & \powerset{\UPropId \mapsto (\ProtVer \times \ProtParams)}
        & \text{registered update proposals}\\
        \var{eps} & \powerset{(\Epoch \times \VKeyGen)} & \text{proposals per-epoch per-key}\\
      \end{array}
    \right)
  \end{equation*}
  %
  \emph{Update proposals registration transitions}
  \begin{equation*}
    \var{\_} \vdash
    \var{\_} \trans{upr}{\_} \var{\_}
    \subseteq \powerset (\UPAEnv \times \UPRState \times \UProp \times \UPRState)
  \end{equation*}
  \caption{Update proposals registration transition-system types}
  \label{fig:ts-types:up-registration}
\end{figure}

\begin{figure}[htb]
  \begin{equation}
    \label{eq:rule:up-registration}
    \inference
    {
      {
        \begin{array}{l}
          \var{pv}\\
          \var{pps}
        \end{array}
      }
      \vdash
      {
        \left(
          \begin{array}{l}
            \var{rups}\\
          \end{array}
        \right)
      }
      \trans{upv}{\var{up}}
      {
        \left(
          \begin{array}{l}
            \var{rups'}\\
          \end{array}
        \right)
      }
      &
      {\begin{array}{l}
          \var{e_c}\\
          \var{dms}
        \end{array}
      }
      \vdash
      {
        \left(
          \begin{array}{l}
            \var{eps}
          \end{array}
        \right)
      }
      \trans{upl}{\var{up}}
      {
        \left(
          \begin{array}{l}
            \var{eps'}
          \end{array}
        \right)
      }
    }
    {
      {
        \begin{array}{l}
          \var{pv}\\
          \var{pps}\\
          \var{e_c}\\
          \var{dms}
        \end{array}
      }
      \vdash
      {
        \left(
          \begin{array}{l}
            \var{rups}\\
            \var{eps}
          \end{array}
        \right)
      }
      \trans{upr}{\var{up}}
      {
        \left(
          \begin{array}{l}
            \var{rups'}\\
            \var{eps'}
          \end{array}
        \right)
      }
    }
  \end{equation}
  \caption{Update registration rules}
  \label{fig:rules:up-registration}
\end{figure}

\clearpage

\subsection{Voting on update proposals}
\label{sec:voting-on-update-proposals}

\begin{figure}[htb]
  \emph{Abstract types}
  %
  \begin{equation*}
    \begin{array}{r@{~\in~}lr}
      \var{v} & \Vote & \text{vote on an update proposal}
    \end{array}
  \end{equation*}
  %
  \emph{Abstract functions}
  \begin{align*}
    & \fun{vCaster} \in \Vote \to \VKey & \text{caster of a vote}\\
    & \fun{vPropId} \in \Vote \to \UPropId & \text{proposal id that is being voted}\\
    & \fun{vSig} \in \Vote \to \Sig & \text{vote signature}
  \end{align*}
  \caption{Voting definitions}
  \label{fig:defs:voting}
\end{figure}

\begin{figure}[htb]
  \emph{Voting environments}
  \begin{align*}
    & \VEnv
      = \left(
      \begin{array}{r@{~\in~}lr}
        \var{rups} & \powerset{\UPropId \mapsto (\ProtVer \times \ProtParams)}
        & \text{registered update proposals}\\
        \var{dms} & \VKeyGen \mapsto \VKey & \text{delegation map}
      \end{array}\right)
  \end{align*}
  %
  \emph{Voting states}
  \begin{align*}
    & \VState
      = \left(
      \begin{array}{r@{~\in~}lr}
        \var{vts} & \powerset{(\UPropId \times \VKeyGen)} & \text{votes}
      \end{array}\right)
  \end{align*}
  %
  \emph{Voting transitions}
    \begin{equation*}
    \_ \vdash \_ \trans{vote}{\_} \_ \in
    \powerset (\VEnv \times \VState \times \Vote \times \VState)
    \end{equation*}
  \caption{Voting transition-system types}
  \label{fig:ts-types:voting}
\end{figure}

Terse explanation of Rule~\ref{eq:rule:voting}:
\begin{itemize}
\item Only genesis keys can vote on an update proposal, although votes can be
  cast by delegates of these genesis keys.
\item We count one vote per genesis key that delegated to the key that is
  casting the vote.
\item The vote must refer to a registered update proposal.
\item The proposal id must be signed by the key that is casting the vote.
\item It is possible for the same genesis key to vote multiple times for
  the same proposal, however this vote will be counted once (note that we're
  taking the union of the key-proposal-id pairs).
\end{itemize}

\begin{figure}[htb]
  \begin{equation}
    \label{eq:rule:voting}
    \inference
    {
      \vPropId{v} = \var{pid} &  \vCaster{v} = \var{vk} &
      \var{vts}_{\var{pid}} =
      \{ (\var{pid}, \var{vk_s}) \mid \var{vk_s} \mapsto \var{vk} \in \var{dms} \}\\
      & \var{pid} \in \dom \var{rups} &
      \mathcal{V}_{\var{vk}}\serialised{\var{pid}}_{(\vSig{v})}\\
    }
    {
      {
        \begin{array}{l}
          \var{rups}\\
          \var{dms}
        \end{array}
      }
      \vdash
      {
        \left(
          \begin{array}{l}
            \var{vts}
          \end{array}
        \right)
      }
      \trans{addvote}{\var{v}}
      {
        \left(
          \begin{array}{l}
            \var{vts} \cup \var{vts}_{\var{pid}}\\
          \end{array}
        \right)
      }
    }
  \end{equation}
  \caption{Update voting rules}
  \label{fig:rules:voting}
\end{figure}

\clearpage

The rules in Figure~\ref{fig:rules:up-confirmation} model the registration of a vote:
\begin{itemize}
\item The vote gets added to the list set of votes per-proposal ($\var{vts}$),
  via transition $\trans{addvote}{}$.
\item If the number of votes for the proposal $v$ refers to exceeds the
  confirmation threshold then the proposal gets added to the set of confirmed
  proposals ($\var{cps}$).
\end{itemize}

\begin{figure}[htb]
  \begin{equation}
    \label{eq:rule:up-no-confirmation}
    \inference
    {
      {
        \begin{array}{l}
          \var{rups}\\
          \var{dms}
        \end{array}
      }
      \vdash
      {
        \left(
          \begin{array}{l}
            \var{vts}
          \end{array}
        \right)
      }
      \trans{addvote}{\var{v}}
      {
        \left(
          \begin{array}{l}
            \var{vts'}
          \end{array}
        \right)
      }\\
      \vPropId{v} = \var{pid}
      & \var{pcThr} \mapsto t \in \var{pps}
      & \size{\{\var{pid}\} \restrictdom \var{vts'}} < t
    }
    {
      {
        \begin{array}{l}
          b_n\\
          \var{pps}\\
          \var{rups}\\
          \var{dms}
        \end{array}
      }
      \vdash
      {
        \left(
          \begin{array}{l}
            \var{cps}\\
            \var{vts}
          \end{array}
        \right)
      }
      \trans{upvote}{\var{v}}
      {
        \left(
          \begin{array}{l}
            \var{cps}\\
            \var{vts'}
          \end{array}
        \right)
      }
    }
  \end{equation}
  \nextdef
  \begin{equation}
    \label{eq:rule:up-confirmation}
    \inference
    {
      {
        \begin{array}{l}
          \var{rups}\\
          \var{dms}
        \end{array}
      }
      \vdash
      {
        \left(
          \begin{array}{l}
            \var{vts}
          \end{array}
        \right)
      }
      \trans{addvote}{\var{v}}
      {
        \left(
          \begin{array}{l}
            \var{vts'}
          \end{array}
        \right)
      }\\
      \vPropId{v} = \var{pid}
      & \var{pcThr} \mapsto t \in \var{pps}
      & t \leq \size{\{\var{pid}\} \restrictdom \var{vts'}}
    }
    {
      {
        \begin{array}{l}
          \var{b_n}\\
          \var{pps}\\
          \var{rups}\\
          \var{dms}
        \end{array}
      }
      \vdash
      {
        \left(
          \begin{array}{l}
            \var{cps}\\
            \var{vts}
          \end{array}
        \right)
      }
      \trans{upvote}{\var{v}}
      {
        \left(
          \begin{array}{l}
            \var{cps} \unionoverride  \{\var{pid} \mapsto b_n\} \\
            \var{vts'}
          \end{array}
        \right)
      }
    }
  \end{equation}
  \caption{Update-proposals confirmation rules}
  \label{fig:rules:up-confirmation}
\end{figure}

\clearpage

\subsection{Block protocol version registration}
\label{sec:block-protocol-version-reg}

Rules in \cref{fig:rules:up-bv-reg} specify what happens when a block issuer
signals that it is ready to upgrade to a new protocol version, given in the
rule by $\var{bv}$:
\begin{itemize}
\item The list of the last $w$ blocks, $\var{bvs}$ is appended with $\var{bv}$.
  Here $w$ is the window-size parameter determines how many candidate protocol
  versions we keep in $\var{bvs}$.
\item If the candidate protocol version $\var{bv}$ is greater than the current
  candidate version $\var{pv_c}$, and there is a significant number of blocks
  signed with that version (condition formalized in
  \cref{eq:predicate:canadopt}), then the new candidate version becomes
  $\var{bv}$. Note that we only compare candidates block versions, and not the
  current protocol version $\var{pv}$. If this rule is used in an initial state
  in which $\var{pv} \leq \var{pv_c}$, then this invariant is maintained by
  these rules.
\item The protocol version $\var{bv}$ must refer to a registered update
  proposal (which are contained in $\var{rups}$), and this update proposal must
  have been confirmed $k$ blocks ago, to ensure stability of the confirmation.
\end{itemize}

\begin{equation}
  \label{eq:predicate:canadopt}
  \begin{array}{r c l}
    \fun{canAdopt}~\var{pps}~\var{bvs}~\var{bv}
    & =
    & \var{upAdptThr} \mapsto t \in pps \wedge
    t \leq \dfrac{\size{\var{bvs} \restrictrange \{\var{bv}\}}}{\size{\var{bvs}}}\\
  \end{array}
\end{equation}

\begin{figure}[htb]
  \begin{equation}
    \label{eq:rule:snocbv}
    \inference
    {
      \var{bvs'} = \var{bvs};\var{bv} & m = \size{bvs'}
    }
    {
      {
        \begin{array}{l}
          w\\
          \var{pps}
        \end{array}
      }
      \vdash
      {
        \left(
          \begin{array}{l}
            bvs
          \end{array}
        \right)
      }
      \trans{snocbv}{bv}
      {
        \left(
          \begin{array}{l}
            {[m - w, ..]} \restrictdom \var{bvs'}
          \end{array}
        \right)
      }
    }
  \end{equation}
  %
  \nextdef
  %
  \begin{equation}
    \label{eq:rule:up-adopted}
    \inference
    {
      \var{bv} \leq \var{pv_c}
      &
      {
        \begin{array}{l}
          w\\
          \var{pps}
        \end{array}
      }
      \vdash
      {
        \left(
          \begin{array}{l}
            bvs
          \end{array}
        \right)
      }
      \trans{snocbv}{bv}
      {
        \left(
          \begin{array}{l}
            bvs'
          \end{array}
        \right)
      }
    }
    {
      {
        \begin{array}{l}
          k\\
          w\\
          b_n\\
          (\var{pv}, \var{pps})\\
          \var{cps}\\
          \var{rups}
        \end{array}
      }
      \vdash
      {
        \left(
          \begin{array}{l}
            (\var{pv_c}, \var{pps_c})\\
            \var{bvs}
          \end{array}
        \right)
      }
      \trans{upbvreg}{\var{bv}}
      {
        \left(
          \begin{array}{l}
            (\var{pv_c}, \var{pps_c})\\
            \var{bvs'}
          \end{array}
        \right)
      }
    }
  \end{equation}
  %
  \nextdef
  %
  \begin{equation}
    \label{eq:rule:up-no-adoption}
    \inference
    {
      \var{pv_c} < \var{bv}
      &
      {
        \begin{array}{l}
          w\\
          \var{pps}
        \end{array}
      }
      \vdash
      {
        \left(
          \begin{array}{l}
            bvs
          \end{array}
        \right)
      }
      \trans{snocbv}{bv}
      {
        \left(
          \begin{array}{l}
            bvs'
          \end{array}
        \right)
      }
      & \neg (\fun{canAdopt}~\var{pps}~\var{bvs'}~\var{bv})\\
      \var{pid} \mapsto (\var{bv}, \wcard) \in \var{rups}
      & \var{pid} \in \dom~(\var{cps} \restrictrange [.., b_n - k])
    }
    {
      {
        \begin{array}{l}
          k\\
          w\\
          b_n\\
          (\var{pv}, \var{pps})\\
          \var{cps}\\
          \var{rups}
        \end{array}
      }
      \vdash
      {
        \left(
          \begin{array}{l}
            (\var{pv_c}, \var{pps_c})\\
            \var{bvs}
          \end{array}
        \right)
      }
      \trans{upbvreg}{\var{bv}}
      {
        \left(
          \begin{array}{l}
            (\var{pv_c}, \var{pps_c})\\
            \var{bvs'}
          \end{array}
        \right)
      }
    }
  \end{equation}
  %
  \nextdef
  %
  \begin{equation}
    \label{eq:rule:up-adoption}
    \inference
    {
      \var{pv_c} < \var{bv}
      &
      {
        \begin{array}{l}
          w\\
          \var{pps}
        \end{array}
      }
      \vdash
      {
        \left(
          \begin{array}{l}
            bvs
          \end{array}
        \right)
      }
      \trans{snocbv}{bv}
      {
        \left(
          \begin{array}{l}
            bvs'
          \end{array}
        \right)
      }
      &
      \fun{canAdopt}~\var{pps}~\var{bvs'}~\var{bv}\\
      \var{pid} \mapsto (\var{bv}, \var{pps_c'}) \in \var{rups}
      & \var{pid} \in \dom~(\var{cps} \restrictrange [.., b_n - k])\\
    }
    {
      {
        \begin{array}{l}
          k\\
          w\\
          b_n\\
          (\var{pv}, \var{pps})\\
          \var{cps}\\
          \var{rups}
        \end{array}
      }
      \vdash
      {
        \left(
          \begin{array}{l}
            (\var{pv_c}, \var{pps_c})\\
            \var{bvs}
          \end{array}
        \right)
      }
      \trans{upbvreg}{\var{bv}}
      {
        \left(
          \begin{array}{l}
            (\var{bv}, \var{pps_c'})\\
            \var{bvs'}
          \end{array}
        \right)
      }
    }
  \end{equation}
  \caption{Update-proposals adoption rules}
  \label{fig:rules:up-bv-reg}
\end{figure}

\clearpage

\subsection{Deviations from the actual implementation}
\label{sec:deviation-actual-impl}

The current specification of the voting mechanism deviates from the actual
implementation, although it should be backwards compatible the latter. These
deviations are required to simplify the voting and update mechanism removing
unnecessary features and reducing accidental complexity. The following
subsections highlight the differences between the this specification and the
current implementation.

\subsubsection{Positive votes}
\label{sec:only-positive-votes}

Votes are only positive. Genesis keys can only vote positively for an update
proposal. In the current implementation stakeholders can vote for or against a
proposal, which makes the voting logic more complex:
\begin{itemize}
\item there are more cases to consider
\item the current voting validation rules allow voters to change their minds
  (by flipping their vote) at most once, which requires to keep track how a
  stake holder voted and how many times. Contrast this with
  Rule~\ref{eq:rule:voting} where we only need to keep track of the set of
  key-proposal-id's pairs.
\end{itemize}

\subsubsection{Alternative version numbers}
\label{sec:alt-version-numbers-constraints}

Alternative version numbers are only lexicographically constrained. The current
implementation seems to be dependent on the order in which the update proposals
arrive: given a new update proposal $\var{up}$, if a set $X$ of update
proposals with the same minor and major versions than $\var{up}$ exist, then
the alternative version of $\var{up}$ has to be one more than the maximum
alternative number of $X$. Not only this logic seems to be brittle since it
depends on the order of arrival of the update proposals, but it requires a more
complex check (which depends on state) to determine if a proposed version is
consistent. By being more lenient on the alternative versions of update
proposals we can simplify the version checking logic considerably.

\subsubsection{Cleanup}
\label{sec:up-cleanup}

Update proposals that are older than $u$ blocks w.r.t. the current block are
discarded from the state, along with its information. The current
implementation makes use of the implicit agreement rule, and the epoch boundary
checks: this leads to plenty of different states for a proposal: active,
adopted, confirmed, competing, never-to-become-adopted, rejected, discarded. If
the cleanup of proposals can be done in the way specified here we will avoid a
great deal of cognitive complexity when reasoning about the update system.

\subsubsection{Adoption threshold}
\label{sec:adoption-threshold}

The current implementation adopts a proposal with version $\var{pv}$ if the
portion of block issuers' stakes, which issued blocks with this version, is
greater than the threshold given by:

\begin{lstlisting}
max spMinThd (spInitThd - (t - s) * spThdDecrement)
\end{lstlisting}

where:
\begin{itemize}
\item \lstinline{spMinThd} is a minimum threshold required for adoption.
\item \lstinline{spInitThd} is an initial threshold.
\item \lstinline{spThdDecrement} is the decrement constant of the initial
  threshold.
\end{itemize}

In this specification we consider a fixed threshold, called $\var{upAdptThr}$
until it becomes clear why a dynamic alternative is needed.

\subsubsection{No voting deadlines}
\label{sec:no-voting-deadlines}

\subsubsection{No checks on unlock-stake-epoch parameter}
\label{sec:no-unlock-stake-epoch-check}

The rule of Figure~\ref{eq:rule:up-validity} does not check the
\lstinline{bvdUnlockStakeEpoch} parameter, since it will have a different
meaning in the handover phase: its use will be reserved for unlocking the
Ouroboros-BFT logic in the software.

\subsubsection{No grouping of proposals per-application name}
\label{sec:no-app-up-grouping}

It is unclear at this moment whether we need to model the applications names
and versions, so this aspect is not modeled in the present rules.

\subsubsection{No model of software-version}
\label{sec:no-model-software-version}

We do not model the
\href{https://github.com/input-output-hk/cardano-sl/blob/develop/docs/block-processing/us.md#software-version}{application
  name and the software version} of this application. It remains to be seen
whether this is part of the scope of this specification.


\subsubsection{Protocol-only vs software updates}
\label{sec:protocol-vs-software-updates}

We do not distinguish between protocol or software updates. The ledger only
cares about the mechanisms by which the protocol parameters are changed.

\subsubsection{Ignored attributes of proposals}

In Figure~\ref{fig:defs:update-proposals} the types $\type{SWVer}$,
$\type{UpdData}$, and $\type{UpdAttr}$ are only needed to model the fact that
an update proposal must sign such data, however, we do not use them for any
other purpose in this formalization.

\subsection{Information in the ledger state}
\label{sec:information-in-ledger-state}

The ledger state has to expose some parameters of the protocol version to its
clients. In the current implementation these parameters are kept in the
`BlockVersionData` structure.

\begin{lstlisting}
  data BlockVersionData = BlockVersionData { ... }
\end{lstlisting}

The following parameters are likely to be needed by the consumers of the ledger
layer:

\begin{itemize}
\item Slot duration (`bvdSlotDuration`)
\item Maximum block size (`bvdMaxBlockSize`)
\item Maximum header size (`bvdMaxHeaderSize`)
\item Maximum transaction size (`bvdMaxTxSize`)
\item Maximum proposal size (`bvdMaxProposalSize`)
\item Transaction fee policy (`bvdTxFeePolicy`)
\end{itemize}

The following parameters will be assumed to be constant:
\begin{itemize}
\item MPC threshold (`bvdMpcThd`)
\item Heavy delegation threshold (`bvdHeavyDelThd`)
\item Update vote threshold (`bvdUpdateVoteThd`)
\item Update proposal threshold (`bvdUpdateProposalThd`)
\item Soft fork rule parameters (`bvdSoftforkRule`)
\end{itemize}

At the moment we don't know whether we need these:

\begin{itemize}
\item Script Version (`bvdScriptVersion`)
\item Update implicit (`bvdUpdateImplicit`)
\end{itemize}

Finally, the `bvdUnlockStakeEpoch` field of `BlockVersionData` does not need to
be modeled.
