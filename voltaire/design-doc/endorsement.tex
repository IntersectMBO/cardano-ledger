\newpage
\section{Endorsement of Proposals (Endorsers, Needed only for Protocol Version Changes)}
\label{sect:endorsement}

Proposals that change either or both the major or minor protocol version number (``hard forks'') need confirmation from the block producing nodes that they are able
to deal with the new protocol.  We term this ``endorsement''.   In previous Cardano eras, endorsement has been handled manually, as a judgement call that
is made by the blockchain operators, who are in control of the genesis keys and the core block producing nodes.  As we move away from federated control and eliminate these
mechanisms, it is necessary to automate this process.  Major protocol versions will never decrease.  Minor protocol versions will always either increase or be set to zero
(with a corresponding increase in the major protocol version number).

\subsection{Automating Endorsement}

Every block producing node (``stake pool'') that wishes either to continue minting
blocks or to support the distributed verification of the Cardano blockchain must upgrade its
software to a node version that is compatible with the new protocol.  So that it
can follow the blockchain from its inception (that is, from the genesis block),
the new node software must also be prepared to handle any previous version of
the protocol, that is from version 1.0 (Byron) through all intermediate versions to the current protocol version.

There are three obvious ways that could enable an \emph{endorser} to endorse a protocol version upgrade.  A decision needs
to be made on which of these to use.

\begin{description}
\item
  [Manual endorsement.]  The endorser submits a signed transaction on-chain that confirms the maximum protocol version that they are
  able to accept (or submits an endorsement transaction for a specific proposal id).
\item
  [Automatic endorsement.]  The new software submits a signed transaction on-chain that confirms the maximum protocol version that it is
  able to accept.
\item
  [Automatic endorsement by block.]  Each block includes the protocol version that is used by the pool that minted it.
\end{description}

% In  cases, the transaction will need to be signed by the stake pool operational key.
Automatic endorsement has the advantage that it will always record the current
status of the pools (including any downgrades).
% However, it is not obvious that a specific proposal (identified by the proposal id) could be endorsed
% automatically.
Manual endorsement allows endorsers to have better control over
the timing of their endorsement (for example, when testing a new node version)
and to endorse specific proposals rather than a protocol version (so avoiding issues if two proposals
mistakenly refer to the same protocol version).  However, some
endorsements may be missed, and consequently some upgrade proposals may not be enacted
even though nodes are actually able to handle the new protocol.
Automating endorsements by block production reduces the number of transactions that are needed (and so reduces overhead costs for pool operators),
but unless the stake snapshot is used, this will only give a statistical measure of the upgrade status (some percentage of blocks have been produced by
pools that have upgraded).
% It may be necessary to restrict the period of endorsement to the time between proposal submission and vote tallying.
% \khcomment{Are there situations where it might make sense to upgrade pre-emptively before a proposal is submitted?}

\subsection{Tallying the Endorsements}

Endorsements are tallied following the deadline that is given in the proposal.
If there are sufficient endorsements for the proposal, then it will be carried
forward for enactment.  While the endorsement deadline is technically
independent of the voting deadline, in practice it will usually be later than the
voting deadline. That is, first the vote will be taken, then pools will indicate
their readiness to proceed with the protocol version upgrade, and if both thresholds are met,
then the proposal will be enacted.

The endorsement threshold is separate from the voting threshold, and also
may be set per-proposal.  % If the endorsement threshold is met, then the proposal is passed for enactment.
In contrast to voting, the total amount of stake that has been delegated for block production purposes is used to determine this
threshold. \khcomment{I think it makes sense to
  use the same snapshot that is used for normal block production rather than the
  most recent delegation snapshot, but this should be discussed.}
To avoid chain forks, the minimum threshold for endorsement should be no less than 50\%
of the block producing stake.
If statistical block production is used, then the threshold should be set as a percentage of blocks that have been produced.
This threshold should be set conservatively (e.g. a minimum of 60\% of the blocks produced).
