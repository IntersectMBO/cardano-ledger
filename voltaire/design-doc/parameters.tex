\section{New Protocol Parameters}
\label{sect:parameters}

\begin{tabular}{||p{2in}|p{2.8in}|p{0.5in}|p{0.4in}||}
  \hline\hline
  \textbf{Parameter} & \textbf{Description} & \textbf{Initial Setting} & \textbf{Updat\-able}
  \\\hline
  \texttt{MinVote} & Minimum Allowable Threshold for Voting & 50\% & Y
  \\\hline
  \texttt{MinEndorse} & Minimum Allowable Threshold for Endorsement & 75\% & Y
  \\\hline
  \texttt{MaxEnact} & How far in Advance of Enactment can a Proposal be Submitted & 3 epochs & Y
  \\\hline
  \texttt{ProposalQuorum} & What is the Minimum Number of Signatories that is Required on a Proposal & ? & Y
  \\\hline
  \texttt{MaxDelegateRetire} & How far in Advance can a Delegate Announce their Retirement & 5 epochs & Y
  \\\hline
  \texttt{MinDelegateRetire} & How far in Advance must a Delegate Announce their Retirement & 0 epochs & N
  \\\hline
  \texttt{LatestEndorsementDeadline} & By Which Slot in an Epoch must a Proposal be Endorsed if it is to be Enacted & ? & N
  \\\hline
  \texttt{LatestVotingDeadline} & By Which Slot in an Epoch must a Proposal be Voted on if it is to be Enacted & ? & N
  \\\hline
  \texttt{LatestProposalDeadline} & By Which Slot in an Epoch must a Proposal be Submitted if it is to be Enacted in that epoch  & ? & N
  \\\hline
  \hline
\end{tabular}

The \texttt{ProposalQuorum} setting will need to be a sensible proportion of the submitter group. Note that \texttt{MinDelegateRetire} may be built in to the protocol rather than a parameter.


\section{New Genesis File Settings}
\label{sec:genesis}

Some additions need to be made to the genesis file.


\begin{tabular}{||p{1in}|p{4.7in}||}
  \hline\hline
  \textbf{Setting} & \textbf{Description}
  \\\hline
  \texttt{submitters} & List of hashes of public keys for the submitter group
  \\\hline
  \hline
\end{tabular}

The \texttt{submitters} entry gives a list of public keys for the initial submitter set.  This can be modified by registering additional submitters.
