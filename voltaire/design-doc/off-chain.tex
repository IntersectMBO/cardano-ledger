\section{Off-Chain Process}
\label{sect:off-chain}

This appendix outlines possible off-chain processes, and how they would interact with the on-chain process that is the focus of this design document.

\subsection{Voter Registration}

The Catalyst system requires voters to register prior to casting their votes.  This is not needed for the on-chain process, since it is possible to take a
stake snapshot directly for each voting address. Votes may then be delegated simply by issuing an on-chain transaction, without requiring any explicit registration.
\khcomment{There seems to be no good reason to exclude ada holders from on-chain voting simply because they have not registered for off-chain Catalyst voting?}
As discussed in Section~\ref{sect:rewards}, rewards are only issued to those who do register for off-chain voting, so there is already an incentive to participate
in the voting process.

\subsection{Proposal Implementation}

On-chain proposals must follow the precise format that is given in Section~\ref{sect:submission}.

\paragraph{Proposal Editors.}  May be involved to confirm that the proposal is documented according to the required procedures, and that the change is properly recorded.

\paragraph{Technical Experts.}  May be involved to formulate the proposal and ensure that it is processed according to the procedures.

\paragraph{Software Developers.} Will be involved to implement new node and other software that is required by a new protocol version.

\paragraph{Security Experts.} Will be involved to advise on the effects of parameter changes, and to advise on the security implications of new software.

It is important that all deadlines are specified precisely in any on-chain proposal, and that sensible deadlines are chosen, based on the type of proposal and any
voting decision that has been made.  It is also essential that all requirements are followed in terms of submission times, stability windows signing etc, so that a proposal is not
rejected unnecessarily.


\subsection{Relating on-chain and off-chain proposals}
\label{sect:relating-off-and-on-chain}

It may be necessary to relate off-chain discussion and decisions to the corresponding on-chain proposal and enactment.  The unique proposal
identifier allows a proposal to be tracked on-chain, but does not link it directly to off-chain discussion.

\begin{description}
\item [Manual.]
  A member of the community (eg the proposal submitter), creates the link.  The proposal is tracked automatically through an on-chain explorer, and the result is
  displayed in a form where it can be tracked by the original voting group or other interested parties.
\item [Automatic.]
  An identifier is constructed following a successful off-chain vote.  This could be a hash or some other form.  The identifier is used by the proposal implementors and editors, and is embedded
  in the proposal implementation.  The identifier is linked to the on-chain identifier, and the progress of the proposal is then tracked automatically on-chain, as described above.
\end{description}

In both cases, it may be possible or even necessary for multiple on-chain proposals to refer to the same off-chain identifier.  This could happen where, for example:

\begin{itemize}
\item
  An incorrect proposal is submitted on-chain.
\item
  A proposal fails to be enacted for some reason, and a new version is therefore submitted on-chain.
\item
  An off-chain proposal gives rise to multiple on-chain proposals.  For example, multiple independent funds transfers might be submitted as the result of a single
  decision; independent proposals might be submitted for each of two parameter changes; a protocol version might be upgraded and some specific parameter settings made independently;
  parameters might be changed over repeated epochs (e.g. to gradually increase/decrease some parameter); parameters might be changed in one epoch, in preparation for a protocol version
  change in a subsequent epoch.
\end{itemize}

Tracking proposals in this way helps preserve continuity, and provides explanations for enacting specific on-chain proposals.  This will be especially helpful where many proposals
are simultaneously under consideratation.


\subsection{Security Threats.}

In most cases, security threats should not be made public until a fix has been prepared, and ideally until it has been enacted on-chain.  This mitigates the risk of
security vulnerabilities affecting the continuity of the blockchain. However, it creates a conflict with the
principle of full decentralisation and transparency of government.  An acceptable process needs to be developed that will meet the required security and openness goals.
\emph{In particular, it may be necessary to prioritise security over transparency, and to accept lower levels of decentralisation.  It may also be necessary to consider
  the legal implications of specific actions or governance structures.}
While vulnerabilities in the node or other software may be fixed out-of-band, without needing any governance mechanism,
security vulnerabilities in the protocol will, of course, usually require a protocol version change (``hard fork'').

\TODO{Survey the literature on blockchain security and open governance, and consider solutions that have been proposed.  Priviledge does not directly consider security, as far as I am aware?  It allows
  proposal prioritisation and deadlines, which provide one way to help deal with security problems.}

\TODO{Take input from the IOG security group.}

Possible approaches include:

\begin{itemize}
\item
  Allow security issues to completely bypass the off-chain discussion, ideation and voting phases.  Appoint a committee of experts to assess security threats.
  Ensure that security threats are classified, and prioritise the implementation of solutions to those threats.  Where the protocol needs to be changed, ensure that this
  proposal has sufficient priority.  Proposals should still comply with implementation requirements, but there may be a moratorium on documentation until the proposal has
  been enacted.
\item
  Allow security issues to bypass discussion and ideation, but then follow the usual voting, implementation, checking etc process.  This may be appropriate for low threats,
  or ones that cannot be easily enacted without rapid detection and response, but raises significant risks where a major or easily exploited vulnerability has been found.
\end{itemize}


\subsection{Legal Issues}

Legal input needs to be taken on the governance structure.  In particular, care needs to be taken over group liability.  Indemnity insurance may be necessary or advisable when
undertaking some roles.

\TODO{Obtain legal input before setting up the procedures.}


\subsection{Scope of Governance Issues}

The PUP mechanism is confined to:

\begin{enumerate}
\item
  changes in updatable protocol parameters;
\item
  protocol version upgrades (``hard forks'');
\item
  central funds transfers (``MIRs'').
\end{enumerate}

Decisions that are taken on parameter and protocol version updates will generally be highly technical.  Discussion may be confined to expert groups, implementation made by
technical experts, and delegates will usually be drawn from those with significant technical expertise.  Decisions must be made in a timely manner, with on-chain enactment.
It may be necessary to establish a separate governance track, different voting thresholds, and specific procedures for such decisions, including some way to prioritise and
filter business.
