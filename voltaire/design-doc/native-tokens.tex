\pagebreak
\section{Extensions to Enable Governance for Native Tokens}
\label{sect:native-tokens}

This appendix considers how these mechanisms could potentially be extended to enable
governance mechanisms for native tokens, for example to govern the supply of specific tokens.
Major changes might be needed compared with the general PUP system.  These issues are discussed below.

\subsection{Integration with Off-Chain Mechanisms}

Any desired mechanism, including informal ones or side-chains, could be used to initiate ideation and discussion or for
initial voting.  This might include Catalyst, provided that a suitable way to obtain snapshots could be devised.
A purely on-chain mechanism would probably be sufficient for most native tokens governance, however.

\subsection{Proposals}

The proposal submission mechanism would need to be adapted for native tokens.

\begin{itemize}
\item
  Proposals would need to be restricted to the correct native token.  This could be done by embedding the monetary policy script hash in the proposal.
\item
  Proposals would be restricted to relevant governance issues rather than protocol parameters, protocol updates or central funds transfers.
  These issues could include:
  \begin{inparaenum}
  \item
    parameter settings related to the monetary policy script (which could be a smart script once Plutus core is deployed);
  \item
    decisions on increasing/reducing the supply of one or more tokens;
  \item
    decisions on transferring tokens to specific accounts (e.g. a ``treasury'');
  \item
    decisions on who can authorise the minting or burning of new tokens (and, of what types), including new signing policies (if this is possible in future);
  \item
    decisions on whether to issue new tokens.
  \end{inparaenum}
\item
  Proposals would need to be signed by a quorum of submitters.  The obvious approach is to use the minting policy script.
\end{itemize}


\subsection{Voter Registration}

Since native tokens are always associated with UTxOs, the same vote key
registration mechanism could be used as for Cardano.  There might be no need to
register for each different kind of native token.  However, such an approach
would only allow one delegation from each address (so the same delegation would
cover Cardano and native tokens votes), which could create
confusion/opportunities for subversion.  A solution might be to hold native
tokens separately from large ada holdings, so creating granular voting rights.
The alternative of allowing multiple voter registrations potentially creates
clutter, and may require some internal fields to be arbitrary sized.

\subsection{Vote Delegation}

The standard delegate system might be overly complicated for most native tokens.
However, voters could register as delegates so that they could cast their votes
directly.  This would carry some fee for the registration and vote transactions.
Alternatively, the token issuer could register standard Yes/No delegates, and
ensure that these always voted in the required way.  A voting centre could hide
the complexities of delegate registration.  Care would need to be taken to
distinguish delegates for Cardano and non-Cardano purposes.  Conceivably, a
single delegate could vote for different native tokens on behalf of a single
token holder.

\subsection{Snapshots, Tallies and Thresholds}

\paragraph{Snapshots.}
Unlike ada, native tokens can be minted or burnt at any point in time, meaning
that the token supply for any native token could fluctuate significantly.  At
present, the total circulation of each native token is never calculated.
Real-time circulation calculations in the ledger are probably not practical from
an efficiency (and therefore security) perspective.  The simplest solution would
be to take voting snapshots for native tokens at the same time as for ada (at
each epoch boundary).

\paragraph{Circulation.}
Native tokens will typically be minted en-masse, and then issued piecemeal.
This needs to be considered in calculating the total supply and voting
thresholds for each token.  Tokens that are held by the token issuer and have
never been issued should probably not count towards the circulation (they are
equivalent to the Cardano reserve).  Similarly tokens that have been returned to
the issuer should not be counted in the circulation, even if they have not been
burnt (they have been returned to the reserve).  Finally, tokens might be held
within special-purpose accounts (e.g. as the equivalent of the Cardano
treasury), and should not be counted towards circulation.  This implies either
imposing some structure on token accounts, or else providing customisable rules for
each asset class.

\paragraph{Performance Issues.}
One major issue that must be addressed is the need to maintain multiple voting
credential maps (one for each native token).  This could have significant memory
implications as well as increased execution time.  In order to avoid slow-downs,
it might be necessary to create/update credential maps incrementally (e.g. when
tokens were issued/transferred/burnt).

\paragraph{Differential  Voting Rights.}
Different kinds of tokens may have different voting rights.  For example, a
currency might have Class A and Class B tokens, where Class A tokens have full
voting rights and Class B tokens have partial or no voting rights.  This could
create complications in calculating voting thresholds.  One solution might be to
separate tokens into different asset class, where a vote on one asset class
could govern decisions on another asset class.  So Class A tokens might be able
to take governance decisions on Class B, but not vice-versa, for example.
Similar considerations apply to Non-Fungible Tokens, adding specific kinds of
value to these tokens.

\paragraph{Thresholds.}
Unlike Cardano, there is no need to set minimum voting thresholds.  These can be
set on each proposal as required.

\subsection{Endorsement.}

Since native tokens cannot affect the Cardano protocol itself, there is no need for an endorsement mechanism
for native token proposals.

\subsection{Automated Enactment.}

Automated enactment might be implemented by calling suitable Plutus smart contracts, for example.  Unlike
Cardano, it is not necessary for automated enactment to be tied to an epoch boundary.
