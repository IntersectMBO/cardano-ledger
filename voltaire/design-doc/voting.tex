\newpage
\section{Voting on Proposals (On-Chain, Delegates)}
\label{sect:voting}

Once a properly signed proposal has been submitted on-chain, delegates may vote on whether or not to accept it.  A threshold is set for each vote.

\subsection{Voting}

Each registered delegate may vote either in favour or against a proposal by submitting the corresponding transaction.  They may change their vote up to the point
where the vote is tallied.  Any changes after that point in time are not considered.  If a delegate chooses not to vote on a proposal, that is considered to be
a vote against the proposal.

\subsection{Tallying Votes}

Votes are tailied according to a snapshot of delegated vote that is taken at the specific point given in the proposal submission. All stake that has been delegated to the delegate address
prior to that point in time counts towards the voting outcome.

\subsection{Voting Outcomes and Thresholds}

Following the tally, a proposal may be either accepted or rejected, based on whether it has achieved the required voting threshold.  Any proposal that achieves at least
the specified percentage of votes is accepted, and passed forward for future enactment.
The threshold for accepting a specific vote is specified in the proposal\khcomment{there may also be minima that are set in the protocol parameters.}.
% All thresholds
