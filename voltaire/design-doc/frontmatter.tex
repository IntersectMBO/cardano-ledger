\hypersetup{
  pdftitle={Design of the Cardano Ledger with Automated Parameter Updates},
  breaklinks=true,
  bookmarks=true,
  colorlinks=false,
  linkcolor={blue},
  citecolor={blue},
  urlcolor={blue},
  linkbordercolor={white},
  citebordercolor={white},
  urlbordercolor={white}
}


\floatstyle{boxed}
\restylefloat{figure}
\cleardoublepage
\renewcommand{\thepage}{\arabic{page}}
\setcounter{page}{1}

\title{Design of the Cardano Ledger with Automated Parameter Updates and Central Fund Transfers}

\author{
   Kevin Hammond \\ {\small \texttt{kevin.hammond@iohk.io}} \\
   Philipp Kant \\ {\small \texttt{philipp.kant@iohk.io}} \\
   Andre Knispel \\ {\small \texttt{andre.knispel@iohk.io}} \\
   }

\date{}

\maketitle

\begin{abstract}
  This document outlines the design of the mechanisms that are needed to support
  automated parameter updates and central funds transfers (collectively PUP).  The PUP mechanism enables a key part of the Cardano governance structure by
  eliminating the use of genesis keys or delegates to control the operation of the Cardano blockchain.  This enables decentralised governance of the blockchain.
  The overall governance process includes both
  off-chain and on-chain components.  This design document is concerned primarily with the on-chain component, but also references the corresponding off-chain
  process, and discusses how the two components interact to ensure decentralised governance.  It covers all parts of the on-chain process including proposal submission,
  vote delegation, on-chain voting/endorsement and automated enactment of proposals.
\end{abstract}

\section*{List of Contributors}
\label{acknowledgements}

\begin{changelog}
\change{2021-02-16}{Kevin Hammond}{FM (IOHK)}{Initial version. }
\change{2021-02-17}{Kevin Hammond}{FM (IOHK)}{Description of goals and submission process.}
\change{2021-02-19}{Kevin Hammond}{FM (IOHK)}{Groups involved.  Vote delegation.}
\change{2021-02-19}{Kevin Hammond}{FM (IOHK)}{Added various outline sections.  Voting, sidechains, short Priviledge comparison.}
\change{2021-02-22}{Kevin Hammond}{FM (IOHK)}{Added workflows, plus textual improvements.}
\change{2021-02-25}{Kevin Hammond}{FM (IOHK)}{Added transition process, updated diagrams, described endorsement and enactment, cleaned up text.}
\change{2021-03-01}{Kevin Hammond}{FM (IOHK)}{Reviewed and updated text.  Added missing diagram, section on transition.}
\change{2021-03-03}{Kevin Hammond}{FM (IOHK)}{Added appendix on user experience.}
\change{2021-03-05}{Kevin Hammond}{FM (IOHK)}{Revised and tidied.  Added address structure and other diagrams.  Checked against delegation design document.}
\change{2021-03-10}{Kevin Hammond}{FM (IOHK)}{Clarified automated endorsement.  Decision needs to be taken on whether endorsement is stake-based or block-based.}
\change{2021-03-12}{Kevin Hammond}{FM (IOHK)}{Added appendix on protocol parameters.  Updated text.}
\change{2021-03-12}{Kevin Hammond}{FM (IOHK)}{Incorporated initial feedback from Andre and Philipp.}
\change{2021-03-15}{Kevin Hammond}{FM (IOHK)}{Added Section on Design Notes to record discussion.}
\change{2021-04-01}{Kevin Hammond}{FM (IOHK)}{Added Appendix on Native Token Governance}
\change{2021-04-01}{Kevin Hammond}{FM (IOHK)}{Added Appendix on Expert Ballots}
\change{2021-04-09}{Kevin Hammond}{FM (IOHK)}{Added Section on Differentiated Submission Groups}
\change{2021-04-09}{Kevin Hammond}{FM (IOHK)}{Added Appendix on Efficiency Concerns}
\end{changelog}

