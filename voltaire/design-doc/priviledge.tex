\section{Comparison with the approach taken by the Priviledge Project}

The EU Priviledge project aims to produce a blockchain-agnostic fully decentralised voting system.
Its design is described in~\ref{Priviledge}, and has been used to inform the design that is described here
(PUP).  The key differences between the PUP mechanism and the one that has been produced for the EU Priviledge project are:

\begin{tabular}{||p{3in}|p{3in}||}
  \hline\hline
  \textbf{Priviledge} & \textbf{PUP}
  \\\hline
  Completely on-chain & Uses off-chain processes as well as on-chain ones \\\hline
  Does not restrict proposal submission & Restricts on-chain submission to a specific group \\\hline
  Proposals are tracked throughout & Proposals are tracked once they are on-chain, mechanisms are provided to tie on-chain to off-chain proposals \\\hline
  Proposals are completely independent & A single off-chain proposal may initiate multiple on-chain proposals for enactment \\\hline
  Security-related proposals follow the standard process\khcomment{Confirm this.}  & Security-related proposals may bypass some off-chain processes \\\hline
  A single vote may only be used for one proposal\khcomment{Confirm this.} & A single vote may be used for any active proposals \\\hline
  Does not identify different actor groups & Clearly identifies different actor groups \\\hline
  Supports explicit vote abstention & Treats abstention as rejection \\\hline
  Allows votes to be removed prior to tally & Allows votes and delegations to be changed prior to tally (equivalent effect) \\\hline
  Allows multiple voting rounds & Allows only one on-chain voting round \\\hline
  Allows  vote stake snapshots to be taken at arbitrary points & Takes vote snapshots at fixed points \\\hline
  Explicit proposal prioritisation & Submission-based prioritisation \\\hline
  Proposals can be cancelled & Proposals cannot be cancelled once they are approved and endorsed \\\hline
  Includes software upgrades (not just protocol upgrades) & Does not consider software upgrades, except where these are protocol version changes \\\hline
  Does not handle central funds transfers & Handles central funds transfers \\\hline
  Does not support vote delegation & Supports vote delegation \\\hline
  Does not consider blockchain stability windows & Explicitly considers blockchain stability windows \\\hline
  Requires version numbering in each proposal & No version numbering required  \\\hline
  Single proposal enactment at an epoch boundary\khcomment{From memory.  Check this.}  & Multiple proposal enactment at each epoch boundary \\\hline
  \hline
\end{tabular}

Most of these differences reflect either differences in requirements, or a desire to make
the new mechanism generally consistent with the existing update mechanism.
Others are simplifications that are aimed at improving peformance or security (e.g. vote
stake snapshots, restricting submission) or reducing development and testing time (e.g. version numbering, time-based prioritisation).
Most of the Priviledge mechanisms could be added at a later date, if required, but some adaptation would generally be required.
