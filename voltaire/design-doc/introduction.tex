\section{Introduction}

The aim of this design is to eliminate the use of the genesis keys (or delegates) within the formal ledger rules.  This will enable
decentralised governance of the blockchain protocol by supporting automated submission and enactment of proposals (PUP).
The process is designed to meet current and future national and international regulatory concerns by removing control of the blockchain protocol
from a small centralised group of actors.

There are three places where genesis keys are currently used.


\begin{itemize}
\item
  \textbf{Parameter Updates:}
  Any updatable protocol parameter may be changed.  Multiple parameters may be changed as part of a single update proposal.
\item
  \textbf{Protocol Version Changes:}
  A change may be made to a major or minor protocol version (a ``hard fork'').  The change must be accompanied by upgrades to
  the software that is being used by block producing nodes (``stake pools''), and acknowledged by sufficient pools upgrading to the new software version.
\item
  \textbf{Transfers from/to Reserves/Treasury:}
  A funds transfer may be made directly from eiher the reserves or treasury pots to a normal address, between the reserves and treasury pots (in either direction), or from a normal address to the treasury pot.  This is referred to in the existing Cardano documentation as an ``MIR'' transfer (Move Instantaneous Rewards).
\end{itemize}

All of these issues must be devolved to a decentralised governance process.

\subsection{Outline Governance Process}

The proposed governance process involves both off-chain and on-chain components.

\begin{enumerate}
\item
  \textbf{Off-Chain:}
  An issue is discussed off-chain.  A vote is taken on a specific proposal using the Catalyst system.  The proposal must be sufficiently unambiguous that its intention is clear, and must include dates by which the proposal is to be submitted and enacted on chain.
\item
  \textbf{On-Chain:}
  A formal proposal is constructed that captures the intention of the off-chain proposal.  It is submitted, verified, and then enacted on-chain.  The proposal will specify the date at which it will be enacted if it is
  properly accepted and endorsed.
\end{enumerate}


The outline process for enacting a proposal is shown in Figure~\ref{fig:workflows}.   The process is split into two main parts: the initial off-chain process, which is expected to be largely manual; and the second, on-chain process, which will largely be
automated.  Only the latter needs to be captured by formal ledger rules.  In order to establish confidence in the governance process,
it is necessary to link the off-chain and on-chain processes.

\begin{figure}
  \begin{center}
  \includegraphics[trim=30 250 100 150,clip,width=\textwidth]{StoryBoard}
  \end{center}
  \caption{Outline Process: Ideas are discussed off-chain and voted on using the Catalyst mechanism.  Successful proposals are implemented, and then submitted for on-chain voting and enactment.}
\label{fig:workflows}
\end{figure}

\emph{Note that the design of the on-chain submission and enactment process allows for the first part of this process to be changed for an on-chain one in future
  (eg one based on the Priviledge design -- see comparison in Appendix~\ref{sect:priviledge}) or a side-chain (see Appendix~\ref{side-chain}) by making small technical
  changes to the submission process to link proposals and by adding additional on-chain mechanisms to record discussions, initial votes, formal approvals etc.}


\newpage
\subsection{Off-Chain Process}

The off-chain process involves five key stages (which may overlap or involve iterations):

\begin{enumerate}
\item
  \textbf{Discussion.}  An issue is discussed.  This may be either formal or informal discussion, using a mechanism such as IdeaScale, a community forum such as a TeleGram channel or the Cardano Forum,
  or any other appropriate mechanism.
\item
  \textbf{Ideation.}  A concrete proposal that addresses the issue is formulated by an interested party.
\item
  \textbf{Submission.}  A proposal is submitted for voting.  The submission of proposals must follow the defined process, and may be restricted in some way.
\item
  \textbf{Voting.}  Voters may vote on whether a proposal should be progressed to implementation, requires amendment etc.  Outline or exact deadlines should be agreed as part of this process.
\item
  \textbf{Implementation.}  Accepted proposals are implemented.  For a simple proposal (eg a change to an updatable parameter), this may be as simple as capturing the intentions of the original proposal in a corresponding JSON file.
  Proposals to change protocol versions may involve significant software development, and therefore need to be funded by the Catalyst fund or some other means.
\end{enumerate}

Each of these off-chain stages may follow its own formally defined proces, and may be partially automated.  Generally, the processes will be driven by convention and rules.  A detailed governance structure is necessary to ensure that business can
be transacted in an efficient and effective way, without risking loss of control to a small non-representative group, and so negating the purpose of decentralised governance.
The processes and structures may evolve over time.  Some of the issues are discussed in Appendix~\ref{sect:off-chain}.  The PUP design assumes that these general governance
issues have been satisfactorily resolved, and that a mandate has been obtained to submit, ratify, and enact specific formal proposals on-chain.


\newpage
\subsection{On-Chain Process}

Once a proposal has been implemented in the required form, it may be submitted for enactment on-chain.  The on-chain process involves four key stages that are managed automatically in accordance with formal ledger rules.  Extreme care needs to be taken with these processes,
since it may not always be possible to stop or reverse actions once they are initiated.

\begin{enumerate}
\item
  \textbf{Submission.}  A formal proposal is \emph{submitted} on-chain by one of a group of designated \emph{submitters}.  The proposal must be in the agreed format, and must include deadlines for voting, endorsement, and enactment.  It must be counter-signed, as described below.  Invalid proposals will be rejected.
\item
  \textbf{Voting.}  The proposal is voted on and the votes are \emph{tallied}.  If sufficient votes are obtained by the stated deadline, then the proposal is \emph{approved} for enactment.
\item
  \textbf{Endorsement.}  If the proposal involves a change to the protocol, then stake pool operators must upgrade their nodes to a version of the software that will accept the new protocol.  This will signal their readiness to run the new protocol, thereby endorsing the proposal.
  Assuming that sufficient endorsement is obtained by the deadline, then the proposal is \emph{endorsed} for enactment.
\item
  \textbf{Enactment.}  Proposals that have been approved (and also endorsed, if necessary) are processed fully automatically, so \emph{enacting} the proposal.
  Valid parameter changes and protocol version upgrades come into effect at the corresponding epoch boundary.  Valid funds transfers may be enacted at a different
  point in time.
\end{enumerate}

\newpage
\subsection{Key Stakeholder Groups}

As described in Section~\ref{sect:groups}, a number of stakeholder groups are involved in the governance process.  The key ones are:

\begin{itemize}
\item
  \textbf{Normal Ada Holders:}
  Normal Ada holders are invested in the Cardano ecosystem in financial and other ways.
   In order to ensure fully decentralised governance, the goal of the PUP design is that normal Ada holders should have the ultimate governance power. They may, however, choose to delegate this to specialists when dealing with more technical issues, for example.   In order to ensure long-term stability of the system, other groups may also act as checks/balances.
%  This ensures fully decentralised governance.
  \item
  \textbf{Stake Pool Operators:}
  Stake pool operators have the responsibility for maintaining the Cardano network.  They are heavily invested in the Cardano ecosystem through the commitment of their time, money and other resources.
  In addition to other voting rights, they need to maintain software compatibility with the latest version of the Cardano blockchain protocol.
  They will have the ultimate decision on whether to progress protocol upgrades.
  \item
  \textbf{End-Users:}
  These are individuals or organisations who use the Cardano blockchain to process transactions etc. (which may be e.g. DeFi applications).
  They may hold minimal ada, perhaps only on a transient basis.
  This will give them limited or no governance rights.  Their needs should, however, be considered when changes are made to protocol parameters or
  protocol upgrades are made.
  \item
  \textbf{Exchanges and other Proxy Ada Holders:}
  Exchanges and similar organisations hold Ada on behalf of other parties.  They require a safe, secure and robust blockchain so that they may operate successfully on
  behalf of their clients.  It is important from both a regulatory and a technical perspective that they do not hold excessive
  power, since this will work against the principle of decentralisation.\footnote{They could voluntarily declare their intention via eg ``enterprise'' addresses, which would not have voting rights. However, experience shows that this mechanism is essentially unused for block production, so it might be similarly ineffective for voting.}
\end{itemize}

The network should be able to evolve in a way that meets the best long-term needs of all these stakeholders, while maintaining the integrity, stability and longevity of
the Cardano blockchain.
